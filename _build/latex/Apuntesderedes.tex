%% Generated by Sphinx.
\def\sphinxdocclass{report}
\documentclass[letterpaper,10pt,spanish]{sphinxmanual}
\ifdefined\pdfpxdimen
   \let\sphinxpxdimen\pdfpxdimen\else\newdimen\sphinxpxdimen
\fi \sphinxpxdimen=.75bp\relax
\ifdefined\pdfimageresolution
    \pdfimageresolution= \numexpr \dimexpr1in\relax/\sphinxpxdimen\relax
\fi
%% let collapsible pdf bookmarks panel have high depth per default
\PassOptionsToPackage{bookmarksdepth=5}{hyperref}

\PassOptionsToPackage{warn}{textcomp}
\usepackage[utf8]{inputenc}
\ifdefined\DeclareUnicodeCharacter
% support both utf8 and utf8x syntaxes
  \ifdefined\DeclareUnicodeCharacterAsOptional
    \def\sphinxDUC#1{\DeclareUnicodeCharacter{"#1}}
  \else
    \let\sphinxDUC\DeclareUnicodeCharacter
  \fi
  \sphinxDUC{00A0}{\nobreakspace}
  \sphinxDUC{2500}{\sphinxunichar{2500}}
  \sphinxDUC{2502}{\sphinxunichar{2502}}
  \sphinxDUC{2514}{\sphinxunichar{2514}}
  \sphinxDUC{251C}{\sphinxunichar{251C}}
  \sphinxDUC{2572}{\textbackslash}
\fi
\usepackage{cmap}
\usepackage[T1]{fontenc}
\usepackage{amsmath,amssymb,amstext}
\usepackage{babel}



\usepackage{tgtermes}
\usepackage{tgheros}
\renewcommand{\ttdefault}{txtt}



\usepackage[Sonny]{fncychap}
\ChNameVar{\Large\normalfont\sffamily}
\ChTitleVar{\Large\normalfont\sffamily}
\usepackage{sphinx}

\fvset{fontsize=auto}
\usepackage{geometry}


% Include hyperref last.
\usepackage{hyperref}
% Fix anchor placement for figures with captions.
\usepackage{hypcap}% it must be loaded after hyperref.
% Set up styles of URL: it should be placed after hyperref.
\urlstyle{same}

\addto\captionsspanish{\renewcommand{\contentsname}{Índice:}}

\usepackage{sphinxmessages}
\setcounter{tocdepth}{2}



\title{Apuntes de redes Documentation}
\date{09 de noviembre de 2021}
\release{1.0}
\author{Oscar G. G.}
\newcommand{\sphinxlogo}{\vbox{}}
\renewcommand{\releasename}{Versión}
\makeindex
\begin{document}

\ifdefined\shorthandoff
  \ifnum\catcode`\=\string=\active\shorthandoff{=}\fi
  \ifnum\catcode`\"=\active\shorthandoff{"}\fi
\fi

\pagestyle{empty}
\sphinxmaketitle
\pagestyle{plain}
\sphinxtableofcontents
\pagestyle{normal}
\phantomsection\label{\detokenize{index::doc}}



\chapter{Caracterización de redes}
\label{\detokenize{t1_caracterizacion_redes/apuntes_t1:caracterizacion-de-redes}}\label{\detokenize{t1_caracterizacion_redes/apuntes_t1::doc}}

\section{Terminología: redes LAN, MAN y WAN, topologías, arquitecturas, protocolos.}
\label{\detokenize{t1_caracterizacion_redes/apuntes_t1:terminologia-redes-lan-man-y-wan-topologias-arquitecturas-protocolos}}
\sphinxAtStartPar
Apartados 1.2.2.2 hasta 1.2.3.2


\section{Sistemas de numeración decimal, binario y hexadecimal.}
\label{\detokenize{t1_caracterizacion_redes/apuntes_t1:sistemas-de-numeracion-decimal-binario-y-hexadecimal}}
\sphinxAtStartPar
Apartado 8.1.1.5


\subsection{Ejercicio: convertir a binario}
\label{\detokenize{t1_caracterizacion_redes/apuntes_t1:ejercicio-convertir-a-binario}}
\sphinxAtStartPar
Convertir a binario:
\begin{itemize}
\item {} 
\sphinxAtStartPar
43

\item {} 
\sphinxAtStartPar
67

\item {} 
\sphinxAtStartPar
95

\item {} 
\sphinxAtStartPar
121

\item {} 
\sphinxAtStartPar
193

\item {} 
\sphinxAtStartPar
217

\item {} 
\sphinxAtStartPar
675

\end{itemize}


\subsection{Ejercicio: convertir a hexadecimal}
\label{\detokenize{t1_caracterizacion_redes/apuntes_t1:ejercicio-convertir-a-hexadecimal}}\begin{itemize}
\item {} 
\sphinxAtStartPar
5191

\item {} 
\sphinxAtStartPar
2193

\item {} 
\sphinxAtStartPar
21430

\item {} 
\sphinxAtStartPar
39810

\item {} 
\sphinxAtStartPar
46712

\end{itemize}


\section{Conversión entre sistemas.}
\label{\detokenize{t1_caracterizacion_redes/apuntes_t1:conversion-entre-sistemas}}
\sphinxAtStartPar
Apartado 8.1.1.5


\section{Arquitectura de redes.}
\label{\detokenize{t1_caracterizacion_redes/apuntes_t1:arquitectura-de-redes}}
\sphinxAtStartPar
Apartados 1.3.1.1 hasta 1.3.2.4


\section{Encapsulamiento de la información.}
\label{\detokenize{t1_caracterizacion_redes/apuntes_t1:encapsulamiento-de-la-informacion}}
\sphinxAtStartPar
Apartado 3.1.14


\section{El modelo OSI.El modelo TCP/IP.}
\label{\detokenize{t1_caracterizacion_redes/apuntes_t1:el-modelo-osi-el-modelo-tcp-ip}}
\sphinxAtStartPar
Apartado 3.2.4.2


\section{Las tecnologías «Ethernet».}
\label{\detokenize{t1_caracterizacion_redes/apuntes_t1:las-tecnologias-ethernet}}
\sphinxAtStartPar
Apartados 5.1.1.1 hasta 5.3.1.2
Apartados 3.3.2.3


\section{El protocolo ARP}
\label{\detokenize{t1_caracterizacion_redes/apuntes_t1:el-protocolo-arp}}
\sphinxAtStartPar
Tenemos esta situación:
\begin{itemize}
\item {} 
\sphinxAtStartPar
Un ordenador tiene la IP 192.168.1.10

\item {} 
\sphinxAtStartPar
Otro ordenador tiene la IP 192.168.1.11

\item {} 
\sphinxAtStartPar
Se desea enviar un bloque de datos desde el 192.168.1.10 hacia el  192.168.1.11

\end{itemize}

\sphinxAtStartPar
Los bloques de datos \sphinxstylestrong{NO SE PUEDEN ENVIAR DIRECTAMENTE DE UNA IP A OTRA IP}. El ordenador 192.168.1.10 no puede enviar directamente a ese ordenador que le han dicho. PRIMERO HAY QUE AVERIGUAR LA DIRECCIÓN ETHERNET DEL 192.168.1.11.

\sphinxAtStartPar
Podrían pasar dos cosas
1. El ordenador 192.168.1.10 ya sabía de alguna manera la MAC del 192.168.1.11
2. Si no la sabe TIENE QUE PREGUNTAR.

\sphinxAtStartPar
Al protocolo que hace preguntas y respuestas traduciendo de direcciones IP a MACs se le llama \sphinxstylestrong{protocolo ARP (Address Resolution Protocol)}

\sphinxAtStartPar
ARP funciona así:
\begin{enumerate}
\sphinxsetlistlabels{\arabic}{enumi}{enumii}{}{.}%
\item {} 
\sphinxAtStartPar
El ordenador 192.168.1.10 envia los datos a su capa Ethernet. Esta capa no sabe cual es la MAC de ese destinatario 192.168.1.11, así que GENERA UN PAQUETE CON UNA PREGUNTA EN LA QUE PONE ESTO
\begin{quote}

\sphinxAtStartPar
Dirección origen: La MAC del 192.168.1.10 (por ejemplo  00\sphinxhyphen{}01\sphinxhyphen{}96\sphinxhyphen{}cc\sphinxhyphen{}59\sphinxhyphen{}43)
Dirección destino: FF\sphinxhyphen{}FF\sphinxhyphen{}FF\sphinxhyphen{}FF\sphinxhyphen{}FF\sphinxhyphen{}FF
\end{quote}

\item {} 
\sphinxAtStartPar
El paquete llega a todo el mundo, y por supuesto también al 192.168.1.11, que lo abre, ve la pregunta y por tanto construye la respuesta:
\begin{quote}

\sphinxAtStartPar
Dirección de origen: La MAC del 192.168.1.11 (por ejemplo 00.30.f2.d.6e.34)
Dirección de destino: La MAC del 192.168.1.10 (00\sphinxhyphen{}01\sphinxhyphen{}96\sphinxhyphen{}cc\sphinxhyphen{}59\sphinxhyphen{}43)
\end{quote}

\item {} 
\sphinxAtStartPar
El paquete de respuesta llega al que preguntó, el cual ya puede enviar los datos a la dirección Ethernet correcta.

\end{enumerate}


\section{El modelo OSI y «Ethernet».}
\label{\detokenize{t1_caracterizacion_redes/apuntes_t1:el-modelo-osi-y-ethernet}}
\sphinxAtStartPar
Apartado 3.2.4.4 página 142


\section{Tipos de cableado «Ethernet».}
\label{\detokenize{t1_caracterizacion_redes/apuntes_t1:tipos-de-cableado-ethernet}}
\sphinxAtStartPar
Cable UTP: apartado 4.2.1.3
Cable STP: apartado 4.2.1.4


\section{Cableado estructurado: subsistemas troncales y horizontales.}
\label{\detokenize{t1_caracterizacion_redes/apuntes_t1:cableado-estructurado-subsistemas-troncales-y-horizontales}}

\section{Algoritmo de acceso al medio CSMA/CD.}
\label{\detokenize{t1_caracterizacion_redes/apuntes_t1:algoritmo-de-acceso-al-medio-csma-cd}}
\sphinxAtStartPar
Apartado 4.4.3.3 página 198


\section{Estructura de la trama «Ethernet».}
\label{\detokenize{t1_caracterizacion_redes/apuntes_t1:estructura-de-la-trama-ethernet}}
\sphinxAtStartPar
Apartado 4.4.4.6, página 209


\chapter{Integración de elementos en una red}
\label{\detokenize{t2_integracion_elementos/apuntes_t2:integracion-de-elementos-en-una-red}}\label{\detokenize{t2_integracion_elementos/apuntes_t2::doc}}

\section{Los medios físicos.}
\label{\detokenize{t2_integracion_elementos/apuntes_t2:los-medios-fisicos}}
\sphinxAtStartPar
Apartado 1.2.1.4 Página 29


\subsection{Los cables metálicos (coaxial, STP y UTP).}
\label{\detokenize{t2_integracion_elementos/apuntes_t2:los-cables-metalicos-coaxial-stp-y-utp}}\begin{itemize}
\item {} 
\sphinxAtStartPar
Cable coaxial: apartado 4.2.1.5, página 168.

\item {} 
\sphinxAtStartPar
Cable UTP: apartado 4.2.2.1, página 170.

\item {} 
\sphinxAtStartPar
Cable STP: apartado 4.2.1.4, página 167.

\end{itemize}


\subsection{Fibra óptica y tipos de fibra.}
\label{\detokenize{t2_integracion_elementos/apuntes_t2:fibra-optica-y-tipos-de-fibra}}
\sphinxAtStartPar
Apartado 4.2.3.3, página 176.


\section{Ancho de banda y tasa de transferencia.}
\label{\detokenize{t2_integracion_elementos/apuntes_t2:ancho-de-banda-y-tasa-de-transferencia}}
\sphinxAtStartPar
Apartado 4.1.3.2 página 162

\sphinxAtStartPar
En realidad antes de comprender el ancho de banda necesitamos comprender algunos conceptos básicos y que indicamos en los siguientes apartados


\subsection{Analógico vs digital}
\label{\detokenize{t2_integracion_elementos/apuntes_t2:analogico-vs-digital}}\begin{itemize}
\item {} 
\sphinxAtStartPar
Una señal analógica es una señal en la que aceptamos cualquier valor.

\item {} 
\sphinxAtStartPar
Una señal digital es una en la que solo se aceptan ciertos valores.

\end{itemize}


\subsection{Parámetros de una señal.}
\label{\detokenize{t2_integracion_elementos/apuntes_t2:parametros-de-una-senal}}\begin{itemize}
\item {} 
\sphinxAtStartPar
Se llama amplitud a la altura de una onda. Cabe destacar que por altura nos referimos a la distancia entre el punto más alto y más bajo de una señal.

\item {} 
\sphinxAtStartPar
Se llama longitud de onda a la distancia que recorre una señal entera. Se mide en metros (o mm, o hasta nanómetros)

\item {} 
\sphinxAtStartPar
Se llama fase al punto donde empieza y acaba la onda, que no tiene por qué ser siempre el punto 0 o el punto más alto.

\end{itemize}


\subsection{Modulación}
\label{\detokenize{t2_integracion_elementos/apuntes_t2:modulacion}}
\sphinxAtStartPar
Modificar parámetros de una señal para enviar 0 y 1. Si combinamos la modificación de muchos parámetros conseguiremos enviar más bits por segundo, pero la recepción se vuelve algo muy complejo.


\subsection{Medidas}
\label{\detokenize{t2_integracion_elementos/apuntes_t2:medidas}}
\sphinxAtStartPar
En informática, en general 1K=1024. Sin embargo, en redes, las medidas como 1=1000. Por tanto si nos hablan de una conexión de 300Mbps, se refieren a 300*10e6. Además se debe recordar que:
\begin{itemize}
\item {} 
\sphinxAtStartPar
«b» (en minúscula) se refiere a \sphinxstyleemphasis{bits}.

\item {} 
\sphinxAtStartPar
«B» (en mayúscula) se refiere a \sphinxstyleemphasis{bytes}.

\end{itemize}

\sphinxAtStartPar
Pero ¿entonces qué es el ancho de banda? Se define como la diferencia entre la frecuencia máxima que se acepta y la frecuencia mínima. Cuanto más ancha sea esa banda, más datos podremos meter. No es lo mismo que la \sphinxstyleemphasis{velocidad}.

\sphinxAtStartPar
La diferencia entre el ancho de banda y la velocidad, es lo que llamamos \sphinxstyleemphasis{rendimiento}, que se mide en porcentaje y nunca es del 100\%
\begin{itemize}
\item {} 
\sphinxAtStartPar
Si tenemos una fibra de 600Mbps, y perdemos el 8\% en protocolos, ¿a qué velocidad nos descargaremos un archivo?

\end{itemize}

\sphinxAtStartPar
Si pierdo el 8\%, conservamos el 92, es decir 0.92*600=552Mbps. Si un archivo ocupa 400MB, entonces ocupa 400*8Mb, es decir 3200Mb, que en realidad es 3200*1024 Kb, o lo que es lo mismo 3200*1024*1024 bits, o sea 3.355.443.200 bits. Si queremos descargar esos 3.355.443.200 bits en una fibra de 552Mbps, aún tendremos que convertir 552 «medidas estándar», es decir 552.000.000 bits por segundo.

\sphinxAtStartPar
Conclusión: si dividimos 3.355.443.200 bits por 552.000.000 obtenemos 6,078 segundos.


\section{Factores físicos que afectan a la transmisión.}
\label{\detokenize{t2_integracion_elementos/apuntes_t2:factores-fisicos-que-afectan-a-la-transmision}}

\section{La conexión inalámbrica.}
\label{\detokenize{t2_integracion_elementos/apuntes_t2:la-conexion-inalambrica}}
\sphinxAtStartPar
Se han popularizado mucho por ofrecer una ventaja inexistente en otros medios: la movilidad. Las redes Wifi usan el estándar 802.11, del cual ha habido muchas variantes:
\begin{itemize}
\item {} 
\sphinxAtStartPar
802.11a), fue el primero, que ofrecía un máximo de 11Mbps, un alcance de unos pocos metros.

\item {} 
\sphinxAtStartPar
802.11n) ofrece mucha más velocidad y alcance.

\end{itemize}

\sphinxAtStartPar
Aparte de eso, las redes Wifi son más inseguras.

\sphinxAtStartPar
Un detalle muy sutil es que en ocasiones los usuarios usan la clave correcta en la red equivocada.

\sphinxAtStartPar
Si nuestro portátil tiene una tarjeta WiFi 802.11n) y nuestro router wifi resulta ser 802.11 a) ambos dispositivos cambian automáticamente al protocolo más compatible, que será el más lento.

\sphinxAtStartPar
Toda red Wifi tiene un identificador llamado SSID. La costumbre es que los nodos difundan el nombre. Sin embargo no es obligatorio, puede activarse una opción con un nombre parecido a este «Not broadcast SSID»

\sphinxAtStartPar
Hay muchos tipos de conexiones:
\begin{itemize}
\item {} 
\sphinxAtStartPar
Wifi: conexión doméstica, con alcance alto y una velocidad alta.

\item {} 
\sphinxAtStartPar
Bluetooth:  punto a punto, velocidad baja y un alcance bajo, consume muy poca energía.

\item {} 
\sphinxAtStartPar
4G, 5G.

\end{itemize}


\section{Los espectros de onda de microondas y radio.}
\label{\detokenize{t2_integracion_elementos/apuntes_t2:los-espectros-de-onda-de-microondas-y-radio}}
\sphinxAtStartPar
Apartado….


\section{Topologías.}
\label{\detokenize{t2_integracion_elementos/apuntes_t2:topologias}}\begin{itemize}
\item {} 
\sphinxAtStartPar
Bus: los equipos forman una línea y cada equipo tiene que averiguar al principio qué ordenadores están a su izquierda y cuales a su derecha. \sphinxstylestrong{Obsoleto}

\item {} 
\sphinxAtStartPar
Anillo: antiguo,  los equipos se conectaban en círculo y había un sentido de giro en el envío de paquetes, el sistema era un poco mejor, pero los cortes en el cable producían errores en toda la red. \sphinxstylestrong{Muy improbable que sigan usándose.}

\item {} 
\sphinxAtStartPar
Estrella: la conexión de los cables implica conectar todos los dispositivos a un punto central que retransmite los datos el equipo correcto. \sphinxstylestrong{Es prácticamente el único sistema que queda en uso}

\end{itemize}


\section{Asociación y autenticación en la WLAN.}
\label{\detokenize{t2_integracion_elementos/apuntes_t2:asociacion-y-autenticacion-en-la-wlan}}
\sphinxAtStartPar
Se llama «autenticación» al proceso seguido por un punto de acceso para ver si un equipo va a tener permiso para enviar y recibir datos a través de ese punto de acceso.

\sphinxAtStartPar
Se llama «asociación» al proceso por el cual un dispositivo utiliza el permiso concedido en el punto anterior para enviar y recibir datos.

\sphinxAtStartPar
Dentro de los sistemas de autenticación:
\begin{itemize}
\item {} 
\sphinxAtStartPar
Deshabilitado: cualquier puede asociarse al punto de acceso y transmitir y recibir.

\item {} 
\sphinxAtStartPar
WEP (Wire Equivalent Privacy) usa un sistema de cifrado y un sistema de claves. Quien proporcione la clave correcta podrá asociarse al punto de acceso y enviar y recibir datos cifrados con una clave del router.

\item {} 
\sphinxAtStartPar
WPA: usa un cifrado más potente y mucho más difícil de romper que WEP.

\item {} 
\sphinxAtStartPar
WPA2: va aún más lejos y ofrece una seguridad mucho mayor.

\end{itemize}

\sphinxAtStartPar
WEP, WPA y WPA2 suelen basarse un sistema llamado PSK (Pre\sphinxhyphen{}Shared Key o clave pre\sphinxhyphen{}compartida). En estos casos ponemos una clave en los router/puntos de acceso que luego también pondremos en los ordenadores. Estos sistemas suelen llamarse WPA\sphinxhyphen{}PSK y WPA2\sphinxhyphen{}PSK.

\sphinxAtStartPar
Existe una variante: WPA,WPA2 usan un tercer equipo que actúa de servidor de autenticación.

\sphinxAtStartPar
En todos los sistemas de autenticación ocurre lo siguiente:
* Los sistemas de cifrado pueden ser más potentes o más débiles. Los más potentes implican velocidades más lentas al gastar más tiempo en el cifrado y descifrado.
* Una vez que un dispositivo envía una petición de conexión el router/punto de acceso envía una petición de clave.
* Si el dispositivo envía una clave correcta, el router envía una clave de cifrado que se usará durante toda la sesión.


\section{Dispositivos hardware en redes: hubs, APs, switches y routers}
\label{\detokenize{t2_integracion_elementos/apuntes_t2:dispositivos-hardware-en-redes-hubs-aps-switches-y-routers}}
\sphinxAtStartPar
Dispositivos hay muchos, pero no todos ellos trabajan en la misma capa de red.


\subsection{Hub}
\label{\detokenize{t2_integracion_elementos/apuntes_t2:hub}}
\sphinxAtStartPar
Un hub o concentrador es un dispositivo «tonto», cualquier paquete que reciba lo difunde por todos los puertos Ethernet. Por lo tanto es un dispositivo de capa de enlace.


\subsection{Switch}
\label{\detokenize{t2_integracion_elementos/apuntes_t2:switch}}
\sphinxAtStartPar
Un switch es un dispositivo con un software incorporado que ejecuta un programa que apoyándose en una memoria RAM interna consigue enviar los paquetes \sphinxstylestrong{solo al destinatario correcto}. En un pequeño número sí generará colisiones, pero su número es muchísimo menor que el de un hub.


\subsection{Ejemplo de simulación con switch}
\label{\detokenize{t2_integracion_elementos/apuntes_t2:ejemplo-de-simulacion-con-switch}}
\sphinxAtStartPar
Supongamos que tenemos un switch. Supongamos que tenemos tres ordenadores:
\begin{itemize}
\item {} 
\sphinxAtStartPar
Ordenador con IP 192.168.1.20 con MAC 0A conectado al puerto 2 del switch.

\item {} 
\sphinxAtStartPar
Ordenador con IP 192.168.1.21 con MAC 0B conectado al puerto 5 del switch.

\item {} 
\sphinxAtStartPar
Ordenador con IP 192.168.1.22 con MAC 0C conectado al puerto 9 del switch.

\end{itemize}

\sphinxAtStartPar
Al principio la tabla del switch está en este estado:


\begin{savenotes}\sphinxattablestart
\centering
\begin{tabulary}{\linewidth}[t]{|T|T|}
\hline

\sphinxAtStartPar
Puerto
&
\sphinxAtStartPar
Mac
\\
\hline
\sphinxAtStartPar
0
&\\
\hline
\sphinxAtStartPar
1
&\\
\hline
\sphinxAtStartPar
2
&\\
\hline
\sphinxAtStartPar
3
&\\
\hline
\sphinxAtStartPar
4
&\\
\hline
\sphinxAtStartPar
5
&\\
\hline
\sphinxAtStartPar
6
&\\
\hline
\sphinxAtStartPar
7
&\\
\hline
\sphinxAtStartPar
8
&\\
\hline
\sphinxAtStartPar
9
&\\
\hline
\end{tabulary}
\par
\sphinxattableend\end{savenotes}

\sphinxAtStartPar
Ahora supongamos que en el 192.168.1.20 envía un ping al 192.168.1.22. El 192.168.1.20 mete el mensaje (que llevará la MAC de origen 0A dentro) en su cable que llega al switch.

\sphinxAtStartPar
El switch se encuentra con dos cosas:
\begin{enumerate}
\sphinxsetlistlabels{\arabic}{enumi}{enumii}{}{.}%
\item {} 
\sphinxAtStartPar
No sabe en qué puerto está el ordenador 0C que es el destinatario final: \sphinxstylestrong{no tendrá más remedio que enviar ese paquete por todos los puertos menos por donde vino}

\item {} 
\sphinxAtStartPar
El switch acaba de aprender y apuntar en su tabla que el ordenador con la MAC 0A está en el puerto 2, así que ahora la tabla del switch queda como sigue:

\end{enumerate}


\begin{savenotes}\sphinxattablestart
\centering
\begin{tabulary}{\linewidth}[t]{|T|T|}
\hline

\sphinxAtStartPar
Puerto
&
\sphinxAtStartPar
Mac
\\
\hline
\sphinxAtStartPar
0
&\\
\hline
\sphinxAtStartPar
1
&\\
\hline
\sphinxAtStartPar
2
&
\sphinxAtStartPar
0A
\\
\hline
\sphinxAtStartPar
3
&\\
\hline
\sphinxAtStartPar
4
&\\
\hline
\sphinxAtStartPar
5
&\\
\hline
\sphinxAtStartPar
6
&\\
\hline
\sphinxAtStartPar
7
&\\
\hline
\sphinxAtStartPar
8
&\\
\hline
\sphinxAtStartPar
9
&\\
\hline
\end{tabulary}
\par
\sphinxattableend\end{savenotes}

\sphinxAtStartPar
El mensaje llegará a todos los ordenadores y casi todos lo descartarán pero el «ping» llegará correctamente al 0C el cual enviará un mensaje de respuesta usando como MAC de origen 0C. Ese mensaje llega al switch que ahora se encuentra con dos cosas:
\begin{enumerate}
\sphinxsetlistlabels{\arabic}{enumi}{enumii}{}{.}%
\item {} 
\sphinxAtStartPar
El ordenador con la MAC 0C está conectado al puerto 9, así que ese conocimiento nuevo se apunta en la tabla quedará como sigue:

\end{enumerate}


\begin{savenotes}\sphinxattablestart
\centering
\begin{tabulary}{\linewidth}[t]{|T|T|}
\hline

\sphinxAtStartPar
Puerto
&
\sphinxAtStartPar
Mac
\\
\hline
\sphinxAtStartPar
0
&\\
\hline
\sphinxAtStartPar
1
&\\
\hline
\sphinxAtStartPar
2
&
\sphinxAtStartPar
0A
\\
\hline
\sphinxAtStartPar
3
&\\
\hline
\sphinxAtStartPar
4
&\\
\hline
\sphinxAtStartPar
5
&\\
\hline
\sphinxAtStartPar
6
&\\
\hline
\sphinxAtStartPar
7
&\\
\hline
\sphinxAtStartPar
8
&\\
\hline
\sphinxAtStartPar
9
&
\sphinxAtStartPar
0C
\\
\hline
\end{tabulary}
\par
\sphinxattableend\end{savenotes}
\begin{enumerate}
\sphinxsetlistlabels{\arabic}{enumi}{enumii}{}{.}%
\setcounter{enumi}{1}
\item {} 
\sphinxAtStartPar
El switch sabe que tiene que enviar un paquete al 0A así que analiza su tabla de direcciones. Al analizar su tabla y observar que tiene apunta que ese destinatario 0A está en el puerto 2 \sphinxstylestrong{EL PAQUETE SE ENVÍA SOLO POR EL PUERTO CORRECTO} sin generar colisiones en otros puntos de la red.

\end{enumerate}


\section{Direccionamiento.}
\label{\detokenize{t2_integracion_elementos/apuntes_t2:direccionamiento}}
\sphinxAtStartPar
Hasta ahora hemos visto que hay muchas capas de red: enlace, red, transporte, aplicación. Cada capa tiene su propio sistema de direcciones:
\begin{itemize}
\item {} 
\sphinxAtStartPar
En Ethernet hemos aprendido que las direcciones son de 48 bits, que se escriben como parejas de números hexadecimales, por ejemplo 3a:d1:f3:55:a8:10. Se debe recordar que hay una dirección Ethernet especial llamada «dirección de broadcast» o «dirección de difusión». Cuando un dispositivo quiere enviar un mensaje a toda la red, pone la dirección ff:ff:ff:ff:ff:ff como dirección de destino. Esto se hace por ejemplo en ARP cuando un ordenador quiere averiguar la MAC teniendo solo su IP. Los switches SIEMPRE OBEDECEN ESAS DIFUSIONES.

\item {} 
\sphinxAtStartPar
Si hay muchos sistemas de direcciones siempre va a ser necesario «traducir entre ellos». Y por ejemplo ya conocemos ARP (Address Resolution Protocol), el cual dada una IP usa difusiones para averiguar la MAC de dicho ordenador con esa IP.

\end{itemize}

\sphinxAtStartPar
Ethernet en realidad divide la MAC en dos partes: los tres primeros pares son el código de fabricante. Los tres últimos son el número de la tarjeta.


\section{Dominios de colisión y de «broadcast».}
\label{\detokenize{t2_integracion_elementos/apuntes_t2:dominios-de-colision-y-de-broadcast}}\begin{itemize}
\item {} 
\sphinxAtStartPar
Se llama «dominio de colisión» al conjunto de equipos que son susceptibles de provocarse colisiones mutuamente. En general es mucho mejor para el rendimiento el tener muchos dominios pequeños en lugar de uno grande.

\item {} 
\sphinxAtStartPar
Dominio de broadcast o dominio de difusión es el conjunto de ordenadores que reciben las difusiones de un ordenador.

\end{itemize}

\sphinxAtStartPar
Un dominio de difusón \sphinxstylestrong{NO TIENE POR QUÉ COINCIDIR} con el dominio de colisiones en una red.


\section{Direcciones IPv4 y máscaras de red.}
\label{\detokenize{t2_integracion_elementos/apuntes_t2:direcciones-ipv4-y-mascaras-de-red}}
\sphinxAtStartPar
Las direcciones IP son las direcciones de la capa software de red más extendida. La capa de red va a ser capaz de enviar datos a sitios remotos. Las direcciones IP están pensadas para poder distinguir un dispositivo cualquiera de cualquier otro del mundo.

\sphinxAtStartPar
Las direcciones IP son software, son un parámetro de configuración. La capa de red sirve como «abstracción» de la capa de enlace Ethernet.

\sphinxAtStartPar
En esencia una dirección IP es una secuencia binaria de 32 bits, como esta:

\sphinxAtStartPar
10010000.11110001.01110011.10101011

\sphinxAtStartPar
Como son muy poco prácticas de manejar y recordar, se suele permitir el escribirlas como números en decimal separados por un punto.
\begin{itemize}
\item {} 
\sphinxAtStartPar
El primer byte de la dirección dada es 10010000, que en decimal es 144.

\item {} 
\sphinxAtStartPar
El segundo byte es 11110001, que en decimal es 241.

\item {} 
\sphinxAtStartPar
El tercer byte es 01110011, que en decimal es 115.

\item {} 
\sphinxAtStartPar
El cuarto byte es 10101011, que en decimal es 171.

\end{itemize}

\sphinxAtStartPar
Por tanto esa IP era 144.241.115.171.

\sphinxAtStartPar
La idea original era que con 32 bits se podrían tener 2 a la 32 equipos, es decir 4.294.967.296 ordenadores.

\sphinxAtStartPar
Como aparentemente había direcciones de sobra, se decidió asignarlas en bloques. Como una IP debe servir para poner número a las redes, y dentro de las redes poner número a cada equipo de esa red, se decidió utilizar siempre una secuencia llamada máscara para poder distinguir cual es el número de red y cual es el número de host.

\sphinxAtStartPar
Supongamos que en un cierto sitio se tiene una red. Si en un ordenador nos han dado una secuencia de bits como la de arriba 10010000.11110001.01110011.10101011 (que en decimal era 144.241.115.171 ) ¿como saber cual es la parte de red y la parte de host. La clave es mirar ese parámetro llamado máscara. Supongamos que esa máscara es 255.0.0.0. Si la pasamos a binario sale que la máscara es 11111111.00000000.00000000.00000000.


\begin{savenotes}\sphinxattablestart
\centering
\begin{tabulary}{\linewidth}[t]{|T|T|}
\hline

\sphinxAtStartPar
Num de red
&
\sphinxAtStartPar
Número de host
\\
\hline
\sphinxAtStartPar
10010000 (144)
&
\sphinxAtStartPar
11110001.01110011.10101011      (241.115.171)
\\
\hline
\sphinxAtStartPar
11111111 (255)
&
\sphinxAtStartPar
00000000.00000000.00000000.     (0.0.0.0)
\\
\hline
\end{tabulary}
\par
\sphinxattableend\end{savenotes}

\sphinxAtStartPar
Por desgracia en los comienzos de Internet las direcciones IP \sphinxstylestrong{se dividieron en bloques muy desiguales} lo que dió lugar a un tremendo desperdicio de direcciones. Estos bloques fueros llamados «clases de direcciones» y en los puntos siguientes desglosamos su composición:


\begin{savenotes}\sphinxattablestart
\centering
\begin{tabulary}{\linewidth}[t]{|T|T|T|T|T|T|}
\hline
\sphinxstyletheadfamily 
\sphinxAtStartPar
Patrón de clase A
&\sphinxstyletheadfamily 
\sphinxAtStartPar
0RRRRRRR
&\sphinxstyletheadfamily 
\sphinxAtStartPar
HHHHHHHH
&\sphinxstyletheadfamily 
\sphinxAtStartPar
HHHHHHHH
&\sphinxstyletheadfamily 
\sphinxAtStartPar
HHHHHHHH
&\sphinxstyletheadfamily \\
\hline
\sphinxAtStartPar
Primera
de clase A
&
\sphinxAtStartPar
00000000
&
\sphinxAtStartPar
00000000
&
\sphinxAtStartPar
00000000
&
\sphinxAtStartPar
00000000
&
\sphinxAtStartPar
0.0.0.
\\
\hline
\sphinxAtStartPar
Última de
clase A
&
\sphinxAtStartPar
01111111
&
\sphinxAtStartPar
11111111
&
\sphinxAtStartPar
11111111
&
\sphinxAtStartPar
11111111
&
\sphinxAtStartPar
127.255.255.255
\\
\hline
\sphinxAtStartPar
Máscara
de clase A
&
\sphinxAtStartPar
11111111
&
\sphinxAtStartPar
00000000
&
\sphinxAtStartPar
00000000
&
\sphinxAtStartPar
00000000
&
\sphinxAtStartPar
255.0.0.0
\\
\hline
\sphinxAtStartPar
Patrón clase B
&
\sphinxAtStartPar
10RRRRRR
&
\sphinxAtStartPar
RRRRRRRR
&
\sphinxAtStartPar
HHHHHHHH
&
\sphinxAtStartPar
HHHHHHHH
&\\
\hline
\sphinxAtStartPar
Primera de clase B
&
\sphinxAtStartPar
10000000
&
\sphinxAtStartPar
00000000
&
\sphinxAtStartPar
00000000
&
\sphinxAtStartPar
00000000
&
\sphinxAtStartPar
128.0.0.0
\\
\hline
\sphinxAtStartPar
Última de clase B
&
\sphinxAtStartPar
10111111
&
\sphinxAtStartPar
11111111
&
\sphinxAtStartPar
11111111
&
\sphinxAtStartPar
11111111
&
\sphinxAtStartPar
191.255.255.255
\\
\hline
\sphinxAtStartPar
Máscara de clase B
&
\sphinxAtStartPar
11111111
&
\sphinxAtStartPar
11111111
&
\sphinxAtStartPar
00000000
&
\sphinxAtStartPar
00000000
&
\sphinxAtStartPar
255.255.0.0
\\
\hline
\sphinxAtStartPar
Patrón de clase C
&
\sphinxAtStartPar
110RRRRR
&
\sphinxAtStartPar
RRRRRRRR
&
\sphinxAtStartPar
RRRRRRRR
&
\sphinxAtStartPar
HHHHHHHH
&\\
\hline
\sphinxAtStartPar
Primera de clase C
&
\sphinxAtStartPar
11000000
&
\sphinxAtStartPar
00000000
&
\sphinxAtStartPar
00000000
&
\sphinxAtStartPar
00000000
&
\sphinxAtStartPar
192.0.0.0
\\
\hline
\sphinxAtStartPar
Última de clase C
&
\sphinxAtStartPar
11011111
&
\sphinxAtStartPar
11111111
&
\sphinxAtStartPar
11111111
&
\sphinxAtStartPar
11111111
&
\sphinxAtStartPar
223.255.255.255
\\
\hline
\end{tabulary}
\par
\sphinxattableend\end{savenotes}


\section{Desperdicio y solución}
\label{\detokenize{t2_integracion_elementos/apuntes_t2:desperdicio-y-solucion}}
\sphinxAtStartPar
Debido a que el desperdicio era inasumible el IETF decidió lo siguiente:
\begin{enumerate}
\sphinxsetlistlabels{\arabic}{enumi}{enumii}{}{.}%
\item {} 
\sphinxAtStartPar
Una solución para el corto plazo, ya que el ritmo de conexión a Internet crecía exponencialmente. Esta solución fue el NAT.

\item {} 
\sphinxAtStartPar
Por otro lado para resolver el problema a largo plazo el IETF diseñó IPv6.

\end{enumerate}

\sphinxAtStartPar
Hablaremos de IPv6 en los apartados siguientes. En cuanto al NAT cabe destacar que actúa de la forma siguiente:

\sphinxAtStartPar
1.\sphinxhyphen{} Hay unos rangos de direcciones que \sphinxstylestrong{sí se pueden repetir en Internet}. Esto ocurre porque en realidad NUNCA SE USAN COMO DIRECCIONES DE DESTINO EN LA INTERNET PÚBLICA.
2. Todo router que use NAT en realidad cambia las IPs de origen de los paquetes que salen y pone en su lugar \sphinxstylestrong{su propia IP} (a la que llamamos pública). Cuando un paquete sale lo hará usando como IP de origen la IP del router.
3. El router anota en una tabla estas manipulaciones y cuando vengas las respuestas, el router volverá a manipular la IP para volver a poner la IP del equipo original que envió algo.

\sphinxAtStartPar
El NAT se caracteriza por:
\begin{enumerate}
\sphinxsetlistlabels{\arabic}{enumi}{enumii}{}{.}%
\item {} 
\sphinxAtStartPar
Funcionó muy bien. Eso es una ventaja.

\item {} 
\sphinxAtStartPar
Perjudica al rendimiento. Los router están contínuamente quitando y poniendo direcciones de los paquetes que entran y salen.

\item {} 
\sphinxAtStartPar
Determinadas operaciones, como abrir servicios, pueden ser difíciles de conseguir para el usuario medio, que se ve obligado a «abrir puertos.»

\end{enumerate}

\sphinxAtStartPar
Los rangos de direcciones «privadas» o repetibles en Internet son los siguientes:
\begin{itemize}
\item {} 
\sphinxAtStartPar
La red 10.0.0.0/8

\item {} 
\sphinxAtStartPar
La red 172.16.0.0/12

\item {} 
\sphinxAtStartPar
La red 192.168.0.0/16

\end{itemize}

\sphinxAtStartPar
En estos rangos aparece el concepto de «máscara abreviada». En lugar de escribir la máscara en binario o en decimal (como 255.0.0.0 o 255.255.255.0) se ha escrito en estos rangos es \sphinxstylestrong{la cantidad de unos que hay dentro de la parte de red de la máscara}

\sphinxAtStartPar
Los rangos de direcciones aceptables en esos casos son entonces:
\begin{itemize}
\item {} 
\sphinxAtStartPar
Desde 10.0.0.0 hasta 10.255.255.255

\item {} 
\sphinxAtStartPar
Desde 172.16.0.0 hasta 172.31.255.255

\item {} 
\sphinxAtStartPar
Desde 192.168.0.0 hasta 192.168.255.255

\end{itemize}

\sphinxAtStartPar
¿Qué se podría decir entonces sobre la IP 161.43.118.31?

\sphinxAtStartPar
Convirtamos 161 a binario: 10100001
\begin{itemize}
\item {} 
\sphinxAtStartPar
La IP empieza por 10, es decir, parece de clase B.

\item {} 
\sphinxAtStartPar
Dicha IP no está en el rango 172.16.0.0 a 172.31.255.255, así que es una \sphinxstylestrong{IP pública}

\item {} 
\sphinxAtStartPar
Al ser de clase B es probable que su máscara sea /16, es decir, 255.255.0.0.

\end{itemize}

\sphinxAtStartPar
¿Qué se podría decir sobre la 172.25.0.13?
* Es privada. Está entre 172.16.0.0 y 172.31.255.255. En concreto es del segundo bloque, o sea que su máscara es /12 o más bits.
* Una máscara /12 sería 11111111.11110000.00000000.00000000, o en decimal 255.240.0.0
* Si se convierte a binario, 172 queda como 10101100, empieza por la pareja de bits 10, es decir, es de clase B.


\subsection{Direcciones de red y de difusión}
\label{\detokenize{t2_integracion_elementos/apuntes_t2:direcciones-de-red-y-de-difusion}}
\sphinxAtStartPar
En las direcciones IP sabemos que tenemos dos partes
\begin{itemize}
\item {} 
\sphinxAtStartPar
Parte de red, también llamado \sphinxstylestrong{prefijo de red}.

\item {} 
\sphinxAtStartPar
Parte de host, que es una secuencia binaria que cambia en cada equipo.

\end{itemize}

\sphinxAtStartPar
Supongamos que tenemos un conjunto de servidores y que queremos ubicarlos en una red. Supongamos que usamos un rango privado como 192.168.1.0 con máscara /24

\sphinxAtStartPar
Esto significa que nuestros últimos 8 bits pueden ser estos
+============+
{\color{red}\bfseries{}|Últimos bits|}
+============+
|  00000000  |
+————+
|  00000001  |
+————+
|  00000010  |
+————+
|  00000011  |
+————+
|  00000100  |
+————+
|   ……   |
+————+
|   11111111 |
+————+

\sphinxAtStartPar
En realidad, la combinación \sphinxstylestrong{todos los bits a 0} y la \sphinxstylestrong{todos los bits a uno} NO SON DIRECCIONES IP VÁLIDAS ASIGNABLES A NODOS. Esto significa que no podemos poner a un equipo ni la 192.168.1.0 ni la 192.168.1.255. \sphinxstylestrong{Estas direcciones tienen un significado especial}
\begin{itemize}
\item {} 
\sphinxAtStartPar
La \sphinxstylestrong{todo a ceros} se usa para «nombrar la red». Así, nuestra red digamos que «se llama 192.168.1.0/24»

\item {} 
\sphinxAtStartPar
La \sphinxstylestrong{todo a unos} es la DIRECCIÓN DE DIFUSIÓN. Cuando un ordenador de nuestra red quiera enviar algo a todos usará como dirección de destino la 192.168.1.255

\end{itemize}

\sphinxAtStartPar
Por tanto, en realidad dentro de nuestra sala de servidores solo podemos poner direcciones entre 192.168.1.1 y 192.168.1.254


\section{Enrutamiento}
\label{\detokenize{t2_integracion_elementos/apuntes_t2:enrutamiento}}
\sphinxAtStartPar
El enrutamiento es el proceso de configurar nodos y routers para conseguir llevar los paquetes hasta su destino:
\begin{itemize}
\item {} 
\sphinxAtStartPar
En los nodos esto implica configurar el «gateway», «puerta de enlace», «router por defecto». Es decir indicar la IP de algún dispositivo de enrutamiento en la red.

\item {} 
\sphinxAtStartPar
En los nodos implica rellenar las tablas de rutas. Es decir, indicar al router las direcciones de red (con las máscaras) e indicarles la IP del siguiente nodo para llegar a esa red.

\end{itemize}

\sphinxAtStartPar
Para el ejemplo visto en clase, repetir el proceso para estas redes:
\begin{enumerate}
\sphinxsetlistlabels{\arabic}{enumi}{enumii}{}{.}%
\item {} 
\sphinxAtStartPar
Red 172.25.3.0/24 para la red izquierda, 16.2.0.0/16 para la red intermedia (la de los router) y red 61.24.3.0/24 para la red derecha.

\item {} \begin{itemize}
\item {} 
\sphinxAtStartPar
Red 41.0.0.0/24 para la red izquierda, 192.168.24.0/24 para la red intermedia (la de los router) y red 184.2.91.0/24 para la red derecha.

\end{itemize}

\end{enumerate}


\section{Protocolos de resolución de direcciones ARP, RARP.}
\label{\detokenize{t2_integracion_elementos/apuntes_t2:protocolos-de-resolucion-de-direcciones-arp-rarp}}

\section{Direcciones IPv6}
\label{\detokenize{t2_integracion_elementos/apuntes_t2:direcciones-ipv6}}

\subsection{Representación de direcciones}
\label{\detokenize{t2_integracion_elementos/apuntes_t2:representacion-de-direcciones}}
\sphinxAtStartPar
Dadas las limitaciones en las direcciones IPv4 se diseñó un nuevo formato de direcciones en el que hubiera muchas más posibilidades: \sphinxstylestrong{IPv6} En IPv6 hay \sphinxstyleemphasis{128 bits para direcciones} lo que supone un espacio de direcciones de 2 elevado a 128, un número realmente grande. Las direcciones IPv6 se escriben como secuencias de 8 grupos de 4 hexadecimales separadas por dos puntos, a continuación vemos algunos ejemplos:

\begin{sphinxVerbatim}[commandchars=\\\{\}]
\PYG{n}{fe80}\PYG{p}{:}\PYG{n}{a13d}\PYG{p}{:}\PYG{n}{d3d6}\PYG{p}{:}\PYG{n}{a190}\PYG{p}{:}\PYG{l+m+mi}{31}\PYG{n}{d2}\PYG{p}{:}\PYG{n}{a216}\PYG{p}{:}\PYG{l+m+mi}{3261}\PYG{p}{:}\PYG{l+m+mi}{1800}
\PYG{l+m+mi}{3410}\PYG{p}{:}\PYG{l+m+mi}{0000}\PYG{p}{:}\PYG{l+m+mi}{0000}\PYG{p}{:}\PYG{l+m+mi}{0000}\PYG{p}{:}\PYG{l+m+mi}{0000}\PYG{p}{:}\PYG{l+m+mi}{0000}\PYG{p}{:}\PYG{l+m+mi}{0000}\PYG{p}{:}\PYG{l+m+mi}{2900}
\end{sphinxVerbatim}

\sphinxAtStartPar
El segundo ejemplo muestra algo interesante y además muy habitual: \sphinxstylestrong{la mayor parte de las veces una dirección IPv6 tendrá muchos ceros consecutivos}. En ese caso, se puede abreviar esa dirección eliminando las secuencias de ceros \sphinxstylestrong{pero dejando un «doble dos puntos»} para indicar que hemos recortado una IPv6, así tendríamos que la última dirección la podemos escribir así:

\begin{sphinxVerbatim}[commandchars=\\\{\}]
\PYG{l+m+mi}{3410}\PYG{p}{:}\PYG{l+m+mi}{0000}\PYG{p}{:}\PYG{l+m+mi}{0000}\PYG{p}{:}\PYG{l+m+mi}{0000}\PYG{p}{:}\PYG{l+m+mi}{0000}\PYG{p}{:}\PYG{l+m+mi}{0000}\PYG{p}{:}\PYG{l+m+mi}{0000}\PYG{p}{:}\PYG{l+m+mi}{2900} \PYG{p}{(}\PYG{n}{sin} \PYG{n}{abreviar}\PYG{p}{)}
\PYG{l+m+mi}{3410}\PYG{p}{:}\PYG{p}{:}\PYG{l+m+mi}{2900} \PYG{p}{(}\PYG{n}{abreviada}\PYG{p}{)}
\end{sphinxVerbatim}

\sphinxAtStartPar
Pero ¡cuidado! esta abreviatura debe hacerse con cuidado. Supongamos una IPv6 como esta:

\begin{sphinxVerbatim}[commandchars=\\\{\}]
\PYG{l+m+mi}{5199}\PYG{p}{:}\PYG{l+m+mi}{0000}\PYG{p}{:}\PYG{l+m+mi}{0000}\PYG{p}{:}\PYG{l+m+mi}{1767}\PYG{p}{:}\PYG{l+m+mi}{0000}\PYG{p}{:}\PYG{l+m+mi}{0000}\PYG{p}{:}\PYG{l+m+mi}{0000}\PYG{p}{:}\PYG{l+m+mi}{00}\PYG{n}{a5}
\end{sphinxVerbatim}

\sphinxAtStartPar
Obsérvese que tenemos dos secuencias de ceros. Una de 8 ceros y otra de 12 ceros. La pregunta típica es ¿puedo abreviar ambos bloques? La respuesta es \sphinxstylestrong{NO}. Si escribiéramos la IPv6 así:

\begin{sphinxVerbatim}[commandchars=\\\{\}]
\PYG{l+m+mi}{5199}\PYG{p}{:}\PYG{p}{:}\PYG{l+m+mi}{1767}\PYG{p}{:}\PYG{p}{:}\PYG{l+m+mi}{00}\PYG{n}{a5}
\end{sphinxVerbatim}

\sphinxAtStartPar
entonces ocurriría que \sphinxstylestrong{la máquina no podría nunca saber cuantos ceros hay en cada bloque abreviado}. Por ello haremos lo siguiente:
\begin{enumerate}
\sphinxsetlistlabels{\arabic}{enumi}{enumii}{}{.}%
\item {} 
\sphinxAtStartPar
El bloque más grande de ceros, lo eliminaremos y pondremos el «doble dos puntos».

\item {} 
\sphinxAtStartPar
El bloque de ceros más pequeño se «recorta» dejándolo con un solo cero por bloque.

\item {} 
\sphinxAtStartPar
Si algun bloque tiene ceros por la izquierda se pueden eliminar (igual que en la vida real da igual escribir 15 que 0015)

\end{enumerate}

\sphinxAtStartPar
Así la dirección IPv6 5199:0000:0000:1767:0000:0000:0000:00a5
\begin{enumerate}
\sphinxsetlistlabels{\arabic}{enumi}{enumii}{}{.}%
\item {} 
\sphinxAtStartPar
Se recorta primero por el bloque de ceros de la derecha y queda 5199:0000:0000:1767::00a5

\item {} 
\sphinxAtStartPar
Y el 5199:0000:0000:1767::00a5 se recorta de nuevo en los ceros de la izquierda para quedar como 5199:0:0:1767::00a5

\item {} 
\sphinxAtStartPar
Por último observamos que en el bloque final hay un 00a5 que se puede escribir como a5, así que nuestra dirección queda finalmente como \sphinxstyleemphasis{5199:0:0:1767::a5}

\end{enumerate}


\subsection{Tipos de direcciones}
\label{\detokenize{t2_integracion_elementos/apuntes_t2:tipos-de-direcciones}}
\sphinxAtStartPar
Hay tres tipos básicos de direcciones IPv6: unicast, anycast y multicast.
\begin{itemize}
\item {} 
\sphinxAtStartPar
Las direcciones unicast son direcciones que indican una única conexión en todo Internet. Son las direcciones más comunes

\item {} 
\sphinxAtStartPar
Las direcciones anycast se usan por lo administradores para «formar grupos». En anycast habrá muchas máquinas con la mismo IPv6 anycast pero cuando se envíe algo a esa dirección anycast \sphinxstylestrong{solo se enviará a uno de ellos}. Los router se encargarán de entregarlo a la máquina más cerca que tenga esa dirección anycast.

\item {} 
\sphinxAtStartPar
Las direcciones multicast se usan en casos en los que varios nodos van a tener una misma IPv6 y cuando se envíe algo a esa IPv6 \sphinxstylestrong{todos la recibirán}.

\end{itemize}

\sphinxAtStartPar
Las direcciones reservadas por el IETF son las siguientes:


\begin{savenotes}\sphinxattablestart
\centering
\begin{tabulary}{\linewidth}[t]{|T|T|T|T|T|}
\hline
\sphinxstyletheadfamily 
\sphinxAtStartPar
Uso
&\sphinxstyletheadfamily 
\sphinxAtStartPar
Prefijo
&\sphinxstyletheadfamily 
\sphinxAtStartPar
Primera IPv6
&\sphinxstyletheadfamily 
\sphinxAtStartPar
Última IPv6
&\sphinxstyletheadfamily 
\sphinxAtStartPar
Fracción que ocupa
\\
\hline
\sphinxAtStartPar
Unicast global
&
\sphinxAtStartPar
2000::/3
&
\sphinxAtStartPar
2000::/3
&
\sphinxAtStartPar
3fff::/3
&
\sphinxAtStartPar
1/8
\\
\hline
\sphinxAtStartPar
Unicast local
único
&
\sphinxAtStartPar
fc00::/7
&
\sphinxAtStartPar
fc00::/7
&
\sphinxAtStartPar
fdff::/7
&
\sphinxAtStartPar
1/128
\\
\hline
\sphinxAtStartPar
Unicast local
en enlace
&
\sphinxAtStartPar
fe80::/10
&
\sphinxAtStartPar
fe80::/10
&
\sphinxAtStartPar
febf::/10
&
\sphinxAtStartPar
1/1024
\\
\hline
\sphinxAtStartPar
Multicast
&
\sphinxAtStartPar
ff00::/8
&
\sphinxAtStartPar
ff00:/8
&
\sphinxAtStartPar
ffff:/8
&
\sphinxAtStartPar
1/256
\\
\hline
\end{tabulary}
\par
\sphinxattableend\end{savenotes}


\section{Conjuntos de protocolos IPv6}
\label{\detokenize{t2_integracion_elementos/apuntes_t2:conjuntos-de-protocolos-ipv6}}

\section{Túneles IPv6}
\label{\detokenize{t2_integracion_elementos/apuntes_t2:tuneles-ipv6}}

\section{Direccionamiento dinámico (DHCP).}
\label{\detokenize{t2_integracion_elementos/apuntes_t2:direccionamiento-dinamico-dhcp}}

\section{Adaptadores.}
\label{\detokenize{t2_integracion_elementos/apuntes_t2:adaptadores}}

\section{Adaptadores alámbricos: instalación y configuración.Adaptadores inalámbricos: instalación y configuración.}
\label{\detokenize{t2_integracion_elementos/apuntes_t2:adaptadores-alambricos-instalacion-y-configuracion-adaptadores-inalambricos-instalacion-y-configuracion}}

\section{Monitorización de la red mediante aplicaciones que usan el protocolo SNMP.}
\label{\detokenize{t2_integracion_elementos/apuntes_t2:monitorizacion-de-la-red-mediante-aplicaciones-que-usan-el-protocolo-snmp}}

\chapter{Anexo: Ejercicios sobre compresión de direcciones IPv6}
\label{\detokenize{t2_integracion_elementos/apuntes_t2:anexo-ejercicios-sobre-compresion-de-direcciones-ipv6}}
\sphinxAtStartPar
Comprimir las direcciones IPv6 siguientes según las reglas de compresión del protocolo (las soluciones aparecen al final):


\begin{savenotes}\sphinxatlongtablestart\begin{longtable}[c]{|l|l|}
\sphinxthelongtablecaptionisattop
\caption{Ejercicios propuestos IPv6\strut}\label{\detokenize{t2_integracion_elementos/apuntes_t2:id1}}\\*[\sphinxlongtablecapskipadjust]
\hline
\sphinxstyletheadfamily 
\sphinxAtStartPar
Num ejercicio
&\sphinxstyletheadfamily 
\sphinxAtStartPar
IPv6
\\
\hline
\endfirsthead

\multicolumn{2}{c}%
{\makebox[0pt]{\sphinxtablecontinued{\tablename\ \thetable{} \textendash{} proviene de la página anterior}}}\\
\hline
\sphinxstyletheadfamily 
\sphinxAtStartPar
Num ejercicio
&\sphinxstyletheadfamily 
\sphinxAtStartPar
IPv6
\\
\hline
\endhead

\hline
\multicolumn{2}{r}{\makebox[0pt][r]{\sphinxtablecontinued{continué en la próxima página}}}\\
\endfoot

\endlastfoot

\sphinxAtStartPar
1
&
\sphinxAtStartPar
\sphinxcode{\sphinxupquote{e9f9:ba67:0000:f4e8:0000:b344:0000:77ce}}
\\
\hline
\sphinxAtStartPar
2
&
\sphinxAtStartPar
\sphinxcode{\sphinxupquote{1105:9002:08f6:d492:0000:810e:6fe2:26e9}}
\\
\hline
\sphinxAtStartPar
3
&
\sphinxAtStartPar
\sphinxcode{\sphinxupquote{0000:7cec:7cf3:0000:8874:0000:4df7:0000}}
\\
\hline
\sphinxAtStartPar
4
&
\sphinxAtStartPar
\sphinxcode{\sphinxupquote{3539:0000:0000:0000:0000:1001:0000:0000}}
\\
\hline
\sphinxAtStartPar
5
&
\sphinxAtStartPar
\sphinxcode{\sphinxupquote{0000:0000:942c:238f:0000:0000:5457:911e}}
\\
\hline
\sphinxAtStartPar
6
&
\sphinxAtStartPar
\sphinxcode{\sphinxupquote{0000:ec1b:1252:bc77:a392:364b:5d89:938b}}
\\
\hline
\sphinxAtStartPar
7
&
\sphinxAtStartPar
\sphinxcode{\sphinxupquote{84c1:79a9:2635:0000:0000:0000:0000:0000}}
\\
\hline
\sphinxAtStartPar
8
&
\sphinxAtStartPar
\sphinxcode{\sphinxupquote{0000:0000:0000:1f29:0348:0000:af6c:9306}}
\\
\hline
\sphinxAtStartPar
9
&
\sphinxAtStartPar
\sphinxcode{\sphinxupquote{3261:0000:77be:4c86:b322:0000:0000:5c8b}}
\\
\hline
\sphinxAtStartPar
10
&
\sphinxAtStartPar
\sphinxcode{\sphinxupquote{2749:0000:0000:0000:0000:03bb:df01:0000}}
\\
\hline
\sphinxAtStartPar
11
&
\sphinxAtStartPar
\sphinxcode{\sphinxupquote{0000:0000:b753:0000:0000:0000:0000:ec7f}}
\\
\hline
\sphinxAtStartPar
12
&
\sphinxAtStartPar
\sphinxcode{\sphinxupquote{0000:dd97:0000:2c00:0000:8ac8:0000:b783}}
\\
\hline
\sphinxAtStartPar
13
&
\sphinxAtStartPar
\sphinxcode{\sphinxupquote{0000:0000:0000:0000:b4aa:12c0:47a0:0000}}
\\
\hline
\sphinxAtStartPar
14
&
\sphinxAtStartPar
\sphinxcode{\sphinxupquote{f310:0000:0000:0000:0000:b63a:0000:0000}}
\\
\hline
\sphinxAtStartPar
15
&
\sphinxAtStartPar
\sphinxcode{\sphinxupquote{0000:0000:412a:0000:0000:2403:0000:3a00}}
\\
\hline
\sphinxAtStartPar
16
&
\sphinxAtStartPar
\sphinxcode{\sphinxupquote{0000:67fa:bd62:c27c:0000:0000:0000:f1af}}
\\
\hline
\sphinxAtStartPar
17
&
\sphinxAtStartPar
\sphinxcode{\sphinxupquote{0000:0000:5211:9028:0000:b9d0:b78b:0000}}
\\
\hline
\sphinxAtStartPar
18
&
\sphinxAtStartPar
\sphinxcode{\sphinxupquote{0000:3d58:0000:aa0a:7371:0000:0000:c0a6}}
\\
\hline
\sphinxAtStartPar
19
&
\sphinxAtStartPar
\sphinxcode{\sphinxupquote{45a2:e709:0000:0000:7373:746b:0000:dc24}}
\\
\hline
\sphinxAtStartPar
20
&
\sphinxAtStartPar
\sphinxcode{\sphinxupquote{9c47:0000:0000:0000:6413:3ed8:0000:0000}}
\\
\hline
\sphinxAtStartPar
21
&
\sphinxAtStartPar
\sphinxcode{\sphinxupquote{d43e:0000:0000:4de7:0000:754c:d79b:0000}}
\\
\hline
\sphinxAtStartPar
22
&
\sphinxAtStartPar
\sphinxcode{\sphinxupquote{3e9f:0000:0000:0000:0000:db5f:0000:0000}}
\\
\hline
\sphinxAtStartPar
23
&
\sphinxAtStartPar
\sphinxcode{\sphinxupquote{dab7:0000:b129:4837:0000:e8bb:cd1d:235c}}
\\
\hline
\sphinxAtStartPar
24
&
\sphinxAtStartPar
\sphinxcode{\sphinxupquote{ec0c:48b6:0000:0000:0000:0000:0000:0000}}
\\
\hline
\sphinxAtStartPar
25
&
\sphinxAtStartPar
\sphinxcode{\sphinxupquote{3633:8915:39f5:0000:0000:0d82:0000:0000}}
\\
\hline
\sphinxAtStartPar
26
&
\sphinxAtStartPar
\sphinxcode{\sphinxupquote{0000:af51:13a0:0000:fc84:f114:9af0:b988}}
\\
\hline
\sphinxAtStartPar
27
&
\sphinxAtStartPar
\sphinxcode{\sphinxupquote{73b6:55f0:0000:0000:0000:0000:0000:b887}}
\\
\hline
\sphinxAtStartPar
28
&
\sphinxAtStartPar
\sphinxcode{\sphinxupquote{0663:0000:4704:3132:0000:2f36:0000:d0ca}}
\\
\hline
\sphinxAtStartPar
29
&
\sphinxAtStartPar
\sphinxcode{\sphinxupquote{0000:0000:f66d:0000:b973:0000:0f5c:0000}}
\\
\hline
\sphinxAtStartPar
30
&
\sphinxAtStartPar
\sphinxcode{\sphinxupquote{cda2:0000:7f62:07fa:c569:0000:ee8f:740c}}
\\
\hline
\sphinxAtStartPar
31
&
\sphinxAtStartPar
\sphinxcode{\sphinxupquote{e75a:0000:f7cc:a6ca:5b28:0000:8d59:0000}}
\\
\hline
\sphinxAtStartPar
32
&
\sphinxAtStartPar
\sphinxcode{\sphinxupquote{c449:ea16:8e11:7d22:0000:0000:0000:5fa8}}
\\
\hline
\sphinxAtStartPar
33
&
\sphinxAtStartPar
\sphinxcode{\sphinxupquote{0000:0000:9a7a:d7a3:1b61:0000:0000:cd22}}
\\
\hline
\sphinxAtStartPar
34
&
\sphinxAtStartPar
\sphinxcode{\sphinxupquote{0000:0000:8b6c:293f:0000:0000:0000:d90c}}
\\
\hline
\sphinxAtStartPar
35
&
\sphinxAtStartPar
\sphinxcode{\sphinxupquote{f8d4:0000:0000:4fd3:0000:0000:1837:0000}}
\\
\hline
\sphinxAtStartPar
36
&
\sphinxAtStartPar
\sphinxcode{\sphinxupquote{0000:fb07:0000:0000:0000:d783:a576:f695}}
\\
\hline
\sphinxAtStartPar
37
&
\sphinxAtStartPar
\sphinxcode{\sphinxupquote{0000:0000:4cc9:fb0c:0000:0000:0000:3bd9}}
\\
\hline
\sphinxAtStartPar
38
&
\sphinxAtStartPar
\sphinxcode{\sphinxupquote{60fe:0000:0000:7c56:0000:0000:c619:0000}}
\\
\hline
\sphinxAtStartPar
39
&
\sphinxAtStartPar
\sphinxcode{\sphinxupquote{ccb3:0000:c821:0000:0000:0000:0000:c74c}}
\\
\hline
\sphinxAtStartPar
40
&
\sphinxAtStartPar
\sphinxcode{\sphinxupquote{0000:0000:0b95:21ea:0000:0000:0000:0000}}
\\
\hline
\sphinxAtStartPar
41
&
\sphinxAtStartPar
\sphinxcode{\sphinxupquote{0000:0000:c009:0000:4f26:0000:affb:53f3}}
\\
\hline
\sphinxAtStartPar
42
&
\sphinxAtStartPar
\sphinxcode{\sphinxupquote{0000:dcd2:71be:0000:734d:2e61:0000:9881}}
\\
\hline
\sphinxAtStartPar
43
&
\sphinxAtStartPar
\sphinxcode{\sphinxupquote{bae4:eb02:0000:f41a:145d:bb47:0000:0000}}
\\
\hline
\sphinxAtStartPar
44
&
\sphinxAtStartPar
\sphinxcode{\sphinxupquote{0000:0000:0000:0000:0000:740c:0000:1741}}
\\
\hline
\sphinxAtStartPar
45
&
\sphinxAtStartPar
\sphinxcode{\sphinxupquote{0000:0000:0000:0000:7cd8:1e15:c90f:ae1b}}
\\
\hline
\sphinxAtStartPar
46
&
\sphinxAtStartPar
\sphinxcode{\sphinxupquote{241c:8bb2:a902:dc92:1333:4bfb:0000:56c3}}
\\
\hline
\sphinxAtStartPar
47
&
\sphinxAtStartPar
\sphinxcode{\sphinxupquote{ec1f:2794:38cd:0000:e5f8:0000:c2cc:898c}}
\\
\hline
\sphinxAtStartPar
48
&
\sphinxAtStartPar
\sphinxcode{\sphinxupquote{86a3:f3ec:0000:0000:0000:c93a:b47c:0000}}
\\
\hline
\sphinxAtStartPar
49
&
\sphinxAtStartPar
\sphinxcode{\sphinxupquote{0000:c570:e19b:681a:0615:0000:0000:0000}}
\\
\hline
\sphinxAtStartPar
50
&
\sphinxAtStartPar
\sphinxcode{\sphinxupquote{0000:dc4e:0000:0000:0000:0000:0000:8ffd}}
\\
\hline
\sphinxAtStartPar
51
&
\sphinxAtStartPar
\sphinxcode{\sphinxupquote{4ecb:68e9:e08d:a371:0000:0000:0000:0000}}
\\
\hline
\sphinxAtStartPar
52
&
\sphinxAtStartPar
\sphinxcode{\sphinxupquote{0000:0000:0000:64b2:0000:7e75:8bb1:ec30}}
\\
\hline
\sphinxAtStartPar
53
&
\sphinxAtStartPar
\sphinxcode{\sphinxupquote{a7e1:6747:0bb4:0000:0000:0000:5bd7:0000}}
\\
\hline
\sphinxAtStartPar
54
&
\sphinxAtStartPar
\sphinxcode{\sphinxupquote{1803:0000:0000:0000:33e5:4828:0000:e00c}}
\\
\hline
\sphinxAtStartPar
55
&
\sphinxAtStartPar
\sphinxcode{\sphinxupquote{0000:e083:0000:df9f:4d92:0000:0000:477d}}
\\
\hline
\sphinxAtStartPar
56
&
\sphinxAtStartPar
\sphinxcode{\sphinxupquote{2001:0000:0000:f007:781f:0000:c2d5:a767}}
\\
\hline
\sphinxAtStartPar
57
&
\sphinxAtStartPar
\sphinxcode{\sphinxupquote{0000:94e8:0000:5a26:a616:b790:0000:d238}}
\\
\hline
\sphinxAtStartPar
58
&
\sphinxAtStartPar
\sphinxcode{\sphinxupquote{0000:1a4b:fd0e:0000:0000:e929:0000:0000}}
\\
\hline
\sphinxAtStartPar
59
&
\sphinxAtStartPar
\sphinxcode{\sphinxupquote{5c77:da9a:3305:39eb:0000:ade2:0750:7450}}
\\
\hline
\sphinxAtStartPar
60
&
\sphinxAtStartPar
\sphinxcode{\sphinxupquote{7d09:0000:9d50:0000:33dc:0000:0000:445b}}
\\
\hline
\sphinxAtStartPar
61
&
\sphinxAtStartPar
\sphinxcode{\sphinxupquote{4b1a:0000:fd0b:f0c5:86be:0000:551c:0000}}
\\
\hline
\sphinxAtStartPar
62
&
\sphinxAtStartPar
\sphinxcode{\sphinxupquote{589e:0000:53c7:93e3:0000:0000:12e9:093c}}
\\
\hline
\sphinxAtStartPar
63
&
\sphinxAtStartPar
\sphinxcode{\sphinxupquote{0000:3616:0000:8509:368f:6ffa:0000:0000}}
\\
\hline
\sphinxAtStartPar
64
&
\sphinxAtStartPar
\sphinxcode{\sphinxupquote{78cc:20e9:00e5:f0a9:eac3:0000:0000:0000}}
\\
\hline
\sphinxAtStartPar
65
&
\sphinxAtStartPar
\sphinxcode{\sphinxupquote{0000:0000:4d06:0000:0000:0000:b1dd:0000}}
\\
\hline
\sphinxAtStartPar
66
&
\sphinxAtStartPar
\sphinxcode{\sphinxupquote{0000:0000:0000:e4bf:0d24:d134:0b2a:100a}}
\\
\hline
\sphinxAtStartPar
67
&
\sphinxAtStartPar
\sphinxcode{\sphinxupquote{0000:0000:0000:0000:0000:0000:9159:8768}}
\\
\hline
\sphinxAtStartPar
68
&
\sphinxAtStartPar
\sphinxcode{\sphinxupquote{0000:0000:0000:d4b0:bea8:0000:abdb:f2f7}}
\\
\hline
\sphinxAtStartPar
69
&
\sphinxAtStartPar
\sphinxcode{\sphinxupquote{dc62:0000:5b76:0000:01e3:77fb:0000:0000}}
\\
\hline
\sphinxAtStartPar
70
&
\sphinxAtStartPar
\sphinxcode{\sphinxupquote{0000:c781:0000:f950:0000:2451:7b8a:0000}}
\\
\hline
\sphinxAtStartPar
71
&
\sphinxAtStartPar
\sphinxcode{\sphinxupquote{4deb:e14c:c9b0:e65c:2265:0000:0000:29b3}}
\\
\hline
\sphinxAtStartPar
72
&
\sphinxAtStartPar
\sphinxcode{\sphinxupquote{0000:3ebe:5d28:0000:7697:0000:0000:1708}}
\\
\hline
\sphinxAtStartPar
73
&
\sphinxAtStartPar
\sphinxcode{\sphinxupquote{0000:0000:0000:f141:0000:215a:0000:0000}}
\\
\hline
\sphinxAtStartPar
74
&
\sphinxAtStartPar
\sphinxcode{\sphinxupquote{e64a:ff9f:eb7c:923b:3a5f:0000:0000:0000}}
\\
\hline
\sphinxAtStartPar
75
&
\sphinxAtStartPar
\sphinxcode{\sphinxupquote{36a8:58be:9b67:be76:66c3:0000:90a5:0000}}
\\
\hline
\sphinxAtStartPar
76
&
\sphinxAtStartPar
\sphinxcode{\sphinxupquote{b096:0000:0000:0000:7291:0000:eefd:0000}}
\\
\hline
\sphinxAtStartPar
77
&
\sphinxAtStartPar
\sphinxcode{\sphinxupquote{0000:0000:036c:0000:e583:d41a:956c:394f}}
\\
\hline
\sphinxAtStartPar
78
&
\sphinxAtStartPar
\sphinxcode{\sphinxupquote{79e4:8350:0000:6d5b:cac2:0000:dd6b:e62a}}
\\
\hline
\sphinxAtStartPar
79
&
\sphinxAtStartPar
\sphinxcode{\sphinxupquote{34f5:0000:2ecb:0000:0000:27c6:0000:0000}}
\\
\hline
\sphinxAtStartPar
80
&
\sphinxAtStartPar
\sphinxcode{\sphinxupquote{b2d2:b338:0000:0000:3160:20bc:f4e6:5878}}
\\
\hline
\sphinxAtStartPar
81
&
\sphinxAtStartPar
\sphinxcode{\sphinxupquote{0000:0000:0000:4ddc:0000:1646:d72c:0000}}
\\
\hline
\sphinxAtStartPar
82
&
\sphinxAtStartPar
\sphinxcode{\sphinxupquote{0000:e0bb:111c:0000:0000:6fd4:0000:8891}}
\\
\hline
\sphinxAtStartPar
83
&
\sphinxAtStartPar
\sphinxcode{\sphinxupquote{c04b:0000:0000:bac7:0000:e028:0000:c8e5}}
\\
\hline
\sphinxAtStartPar
84
&
\sphinxAtStartPar
\sphinxcode{\sphinxupquote{f680:0000:0000:0000:0000:0000:dfa9:0000}}
\\
\hline
\sphinxAtStartPar
85
&
\sphinxAtStartPar
\sphinxcode{\sphinxupquote{e19b:0000:101a:3fcc:ae97:0000:7970:f214}}
\\
\hline
\sphinxAtStartPar
86
&
\sphinxAtStartPar
\sphinxcode{\sphinxupquote{50c1:9e9c:0000:0000:ed17:0000:8e99:0000}}
\\
\hline
\sphinxAtStartPar
87
&
\sphinxAtStartPar
\sphinxcode{\sphinxupquote{0000:ab50:4066:2809:f314:0000:92da:0000}}
\\
\hline
\sphinxAtStartPar
88
&
\sphinxAtStartPar
\sphinxcode{\sphinxupquote{ff48:0000:0000:22e7:9656:0000:0000:0000}}
\\
\hline
\sphinxAtStartPar
89
&
\sphinxAtStartPar
\sphinxcode{\sphinxupquote{0000:0000:0000:0000:0000:0000:14a5:0000}}
\\
\hline
\sphinxAtStartPar
90
&
\sphinxAtStartPar
\sphinxcode{\sphinxupquote{8d13:0000:237a:c4d7:0000:0000:4df0:c8d0}}
\\
\hline
\sphinxAtStartPar
91
&
\sphinxAtStartPar
\sphinxcode{\sphinxupquote{4a7e:caaa:0000:0000:ec08:ce1f:0000:0000}}
\\
\hline
\sphinxAtStartPar
92
&
\sphinxAtStartPar
\sphinxcode{\sphinxupquote{1c21:0000:0000:0000:0000:0000:e5c0:fc84}}
\\
\hline
\sphinxAtStartPar
93
&
\sphinxAtStartPar
\sphinxcode{\sphinxupquote{0000:0000:32da:419f:0000:5b69:dad0:bc58}}
\\
\hline
\sphinxAtStartPar
94
&
\sphinxAtStartPar
\sphinxcode{\sphinxupquote{e73c:b036:3efd:0000:0000:0d87:0000:6197}}
\\
\hline
\sphinxAtStartPar
95
&
\sphinxAtStartPar
\sphinxcode{\sphinxupquote{0000:0000:0000:5bb0:bf99:0000:a21e:0000}}
\\
\hline
\sphinxAtStartPar
96
&
\sphinxAtStartPar
\sphinxcode{\sphinxupquote{0000:9a47:5197:a901:0000:0000:3ac3:39c8}}
\\
\hline
\sphinxAtStartPar
97
&
\sphinxAtStartPar
\sphinxcode{\sphinxupquote{0000:14e3:0000:0000:06d0:e328:20a4:ea05}}
\\
\hline
\sphinxAtStartPar
98
&
\sphinxAtStartPar
\sphinxcode{\sphinxupquote{d9b4:e5de:7478:a8ac:2a19:3ef6:a970:0000}}
\\
\hline
\sphinxAtStartPar
99
&
\sphinxAtStartPar
\sphinxcode{\sphinxupquote{dcd1:0000:a0df:0000:0000:f58a:0000:f323}}
\\
\hline
\sphinxAtStartPar
100
&
\sphinxAtStartPar
\sphinxcode{\sphinxupquote{0000:021d:64f1:df12:e8ac:0000:489f:75a0}}
\\
\hline
\end{longtable}\sphinxatlongtableend\end{savenotes}


\section{Soluciones a compresión de direcciones IPv6}
\label{\detokenize{t2_integracion_elementos/apuntes_t2:soluciones-a-compresion-de-direcciones-ipv6}}
\sphinxAtStartPar
A continuación se muestran las soluciones a los ejercicios propuestos:


\begin{savenotes}\sphinxatlongtablestart\begin{longtable}[c]{|l|l|l|}
\sphinxthelongtablecaptionisattop
\caption{Ejercicios resueltos IPv6\strut}\label{\detokenize{t2_integracion_elementos/apuntes_t2:id2}}\\*[\sphinxlongtablecapskipadjust]
\hline
\sphinxstyletheadfamily 
\sphinxAtStartPar
Num ejercicio
&\sphinxstyletheadfamily 
\sphinxAtStartPar
IPv6
&\sphinxstyletheadfamily 
\sphinxAtStartPar
Comprimida
\\
\hline
\endfirsthead

\multicolumn{3}{c}%
{\makebox[0pt]{\sphinxtablecontinued{\tablename\ \thetable{} \textendash{} proviene de la página anterior}}}\\
\hline
\sphinxstyletheadfamily 
\sphinxAtStartPar
Num ejercicio
&\sphinxstyletheadfamily 
\sphinxAtStartPar
IPv6
&\sphinxstyletheadfamily 
\sphinxAtStartPar
Comprimida
\\
\hline
\endhead

\hline
\multicolumn{3}{r}{\makebox[0pt][r]{\sphinxtablecontinued{continué en la próxima página}}}\\
\endfoot

\endlastfoot

\sphinxAtStartPar
1
&
\sphinxAtStartPar
\sphinxcode{\sphinxupquote{e9f9:ba67:0000:f4e8:0000:b344:0000:77ce}}
&
\sphinxAtStartPar
\sphinxcode{\sphinxupquote{e9f9:ba67:0:f4e8:0:b344:0:77ce}}
\\
\hline
\sphinxAtStartPar
2
&
\sphinxAtStartPar
\sphinxcode{\sphinxupquote{1105:9002:08f6:d492:0000:810e:6fe2:26e9}}
&
\sphinxAtStartPar
\sphinxcode{\sphinxupquote{1105:9002:8f6:d492:0:810e:6fe2:26e9}}
\\
\hline
\sphinxAtStartPar
3
&
\sphinxAtStartPar
\sphinxcode{\sphinxupquote{0000:7cec:7cf3:0000:8874:0000:4df7:0000}}
&
\sphinxAtStartPar
\sphinxcode{\sphinxupquote{0:7cec:7cf3:0:8874:0:4df7:0}}
\\
\hline
\sphinxAtStartPar
4
&
\sphinxAtStartPar
\sphinxcode{\sphinxupquote{3539:0000:0000:0000:0000:1001:0000:0000}}
&
\sphinxAtStartPar
\sphinxcode{\sphinxupquote{3539::1001:0:0}}
\\
\hline
\sphinxAtStartPar
5
&
\sphinxAtStartPar
\sphinxcode{\sphinxupquote{0000:0000:942c:238f:0000:0000:5457:911e}}
&
\sphinxAtStartPar
\sphinxcode{\sphinxupquote{::942c:238f:0:0:5457:911e}}
\\
\hline
\sphinxAtStartPar
6
&
\sphinxAtStartPar
\sphinxcode{\sphinxupquote{0000:ec1b:1252:bc77:a392:364b:5d89:938b}}
&
\sphinxAtStartPar
\sphinxcode{\sphinxupquote{0:ec1b:1252:bc77:a392:364b:5d89:938b}}
\\
\hline
\sphinxAtStartPar
7
&
\sphinxAtStartPar
\sphinxcode{\sphinxupquote{84c1:79a9:2635:0000:0000:0000:0000:0000}}
&
\sphinxAtStartPar
\sphinxcode{\sphinxupquote{84c1:79a9:2635::}}
\\
\hline
\sphinxAtStartPar
8
&
\sphinxAtStartPar
\sphinxcode{\sphinxupquote{0000:0000:0000:1f29:0348:0000:af6c:9306}}
&
\sphinxAtStartPar
\sphinxcode{\sphinxupquote{::1f29:348:0:af6c:9306}}
\\
\hline
\sphinxAtStartPar
9
&
\sphinxAtStartPar
\sphinxcode{\sphinxupquote{3261:0000:77be:4c86:b322:0000:0000:5c8b}}
&
\sphinxAtStartPar
\sphinxcode{\sphinxupquote{3261:0:77be:4c86:b322::5c8b}}
\\
\hline
\sphinxAtStartPar
10
&
\sphinxAtStartPar
\sphinxcode{\sphinxupquote{2749:0000:0000:0000:0000:03bb:df01:0000}}
&
\sphinxAtStartPar
\sphinxcode{\sphinxupquote{2749::3bb:df01:0}}
\\
\hline
\sphinxAtStartPar
11
&
\sphinxAtStartPar
\sphinxcode{\sphinxupquote{0000:0000:b753:0000:0000:0000:0000:ec7f}}
&
\sphinxAtStartPar
\sphinxcode{\sphinxupquote{0:0:b753::ec7f}}
\\
\hline
\sphinxAtStartPar
12
&
\sphinxAtStartPar
\sphinxcode{\sphinxupquote{0000:dd97:0000:2c00:0000:8ac8:0000:b783}}
&
\sphinxAtStartPar
\sphinxcode{\sphinxupquote{0:dd97:0:2c00:0:8ac8:0:b783}}
\\
\hline
\sphinxAtStartPar
13
&
\sphinxAtStartPar
\sphinxcode{\sphinxupquote{0000:0000:0000:0000:b4aa:12c0:47a0:0000}}
&
\sphinxAtStartPar
\sphinxcode{\sphinxupquote{::b4aa:12c0:47a0:0}}
\\
\hline
\sphinxAtStartPar
14
&
\sphinxAtStartPar
\sphinxcode{\sphinxupquote{f310:0000:0000:0000:0000:b63a:0000:0000}}
&
\sphinxAtStartPar
\sphinxcode{\sphinxupquote{f310::b63a:0:0}}
\\
\hline
\sphinxAtStartPar
15
&
\sphinxAtStartPar
\sphinxcode{\sphinxupquote{0000:0000:412a:0000:0000:2403:0000:3a00}}
&
\sphinxAtStartPar
\sphinxcode{\sphinxupquote{::412a:0:0:2403:0:3a00}}
\\
\hline
\sphinxAtStartPar
16
&
\sphinxAtStartPar
\sphinxcode{\sphinxupquote{0000:67fa:bd62:c27c:0000:0000:0000:f1af}}
&
\sphinxAtStartPar
\sphinxcode{\sphinxupquote{0:67fa:bd62:c27c::f1af}}
\\
\hline
\sphinxAtStartPar
17
&
\sphinxAtStartPar
\sphinxcode{\sphinxupquote{0000:0000:5211:9028:0000:b9d0:b78b:0000}}
&
\sphinxAtStartPar
\sphinxcode{\sphinxupquote{::5211:9028:0:b9d0:b78b:0}}
\\
\hline
\sphinxAtStartPar
18
&
\sphinxAtStartPar
\sphinxcode{\sphinxupquote{0000:3d58:0000:aa0a:7371:0000:0000:c0a6}}
&
\sphinxAtStartPar
\sphinxcode{\sphinxupquote{0:3d58:0:aa0a:7371::c0a6}}
\\
\hline
\sphinxAtStartPar
19
&
\sphinxAtStartPar
\sphinxcode{\sphinxupquote{45a2:e709:0000:0000:7373:746b:0000:dc24}}
&
\sphinxAtStartPar
\sphinxcode{\sphinxupquote{45a2:e709::7373:746b:0:dc24}}
\\
\hline
\sphinxAtStartPar
20
&
\sphinxAtStartPar
\sphinxcode{\sphinxupquote{9c47:0000:0000:0000:6413:3ed8:0000:0000}}
&
\sphinxAtStartPar
\sphinxcode{\sphinxupquote{9c47::6413:3ed8:0:0}}
\\
\hline
\sphinxAtStartPar
21
&
\sphinxAtStartPar
\sphinxcode{\sphinxupquote{d43e:0000:0000:4de7:0000:754c:d79b:0000}}
&
\sphinxAtStartPar
\sphinxcode{\sphinxupquote{d43e::4de7:0:754c:d79b:0}}
\\
\hline
\sphinxAtStartPar
22
&
\sphinxAtStartPar
\sphinxcode{\sphinxupquote{3e9f:0000:0000:0000:0000:db5f:0000:0000}}
&
\sphinxAtStartPar
\sphinxcode{\sphinxupquote{3e9f::db5f:0:0}}
\\
\hline
\sphinxAtStartPar
23
&
\sphinxAtStartPar
\sphinxcode{\sphinxupquote{dab7:0000:b129:4837:0000:e8bb:cd1d:235c}}
&
\sphinxAtStartPar
\sphinxcode{\sphinxupquote{dab7:0:b129:4837:0:e8bb:cd1d:235c}}
\\
\hline
\sphinxAtStartPar
24
&
\sphinxAtStartPar
\sphinxcode{\sphinxupquote{ec0c:48b6:0000:0000:0000:0000:0000:0000}}
&
\sphinxAtStartPar
\sphinxcode{\sphinxupquote{ec0c:48b6::}}
\\
\hline
\sphinxAtStartPar
25
&
\sphinxAtStartPar
\sphinxcode{\sphinxupquote{3633:8915:39f5:0000:0000:0d82:0000:0000}}
&
\sphinxAtStartPar
\sphinxcode{\sphinxupquote{3633:8915:39f5::d82:0:0}}
\\
\hline
\sphinxAtStartPar
26
&
\sphinxAtStartPar
\sphinxcode{\sphinxupquote{0000:af51:13a0:0000:fc84:f114:9af0:b988}}
&
\sphinxAtStartPar
\sphinxcode{\sphinxupquote{0:af51:13a0:0:fc84:f114:9af0:b988}}
\\
\hline
\sphinxAtStartPar
27
&
\sphinxAtStartPar
\sphinxcode{\sphinxupquote{73b6:55f0:0000:0000:0000:0000:0000:b887}}
&
\sphinxAtStartPar
\sphinxcode{\sphinxupquote{73b6:55f0::b887}}
\\
\hline
\sphinxAtStartPar
28
&
\sphinxAtStartPar
\sphinxcode{\sphinxupquote{0663:0000:4704:3132:0000:2f36:0000:d0ca}}
&
\sphinxAtStartPar
\sphinxcode{\sphinxupquote{663:0:4704:3132:0:2f36:0:d0ca}}
\\
\hline
\sphinxAtStartPar
29
&
\sphinxAtStartPar
\sphinxcode{\sphinxupquote{0000:0000:f66d:0000:b973:0000:0f5c:0000}}
&
\sphinxAtStartPar
\sphinxcode{\sphinxupquote{::f66d:0:b973:0:f5c:0}}
\\
\hline
\sphinxAtStartPar
30
&
\sphinxAtStartPar
\sphinxcode{\sphinxupquote{cda2:0000:7f62:07fa:c569:0000:ee8f:740c}}
&
\sphinxAtStartPar
\sphinxcode{\sphinxupquote{cda2:0:7f62:7fa:c569:0:ee8f:740c}}
\\
\hline
\sphinxAtStartPar
31
&
\sphinxAtStartPar
\sphinxcode{\sphinxupquote{e75a:0000:f7cc:a6ca:5b28:0000:8d59:0000}}
&
\sphinxAtStartPar
\sphinxcode{\sphinxupquote{e75a:0:f7cc:a6ca:5b28:0:8d59:0}}
\\
\hline
\sphinxAtStartPar
32
&
\sphinxAtStartPar
\sphinxcode{\sphinxupquote{c449:ea16:8e11:7d22:0000:0000:0000:5fa8}}
&
\sphinxAtStartPar
\sphinxcode{\sphinxupquote{c449:ea16:8e11:7d22::5fa8}}
\\
\hline
\sphinxAtStartPar
33
&
\sphinxAtStartPar
\sphinxcode{\sphinxupquote{0000:0000:9a7a:d7a3:1b61:0000:0000:cd22}}
&
\sphinxAtStartPar
\sphinxcode{\sphinxupquote{::9a7a:d7a3:1b61:0:0:cd22}}
\\
\hline
\sphinxAtStartPar
34
&
\sphinxAtStartPar
\sphinxcode{\sphinxupquote{0000:0000:8b6c:293f:0000:0000:0000:d90c}}
&
\sphinxAtStartPar
\sphinxcode{\sphinxupquote{0:0:8b6c:293f::d90c}}
\\
\hline
\sphinxAtStartPar
35
&
\sphinxAtStartPar
\sphinxcode{\sphinxupquote{f8d4:0000:0000:4fd3:0000:0000:1837:0000}}
&
\sphinxAtStartPar
\sphinxcode{\sphinxupquote{f8d4::4fd3:0:0:1837:0}}
\\
\hline
\sphinxAtStartPar
36
&
\sphinxAtStartPar
\sphinxcode{\sphinxupquote{0000:fb07:0000:0000:0000:d783:a576:f695}}
&
\sphinxAtStartPar
\sphinxcode{\sphinxupquote{0:fb07::d783:a576:f695}}
\\
\hline
\sphinxAtStartPar
37
&
\sphinxAtStartPar
\sphinxcode{\sphinxupquote{0000:0000:4cc9:fb0c:0000:0000:0000:3bd9}}
&
\sphinxAtStartPar
\sphinxcode{\sphinxupquote{0:0:4cc9:fb0c::3bd9}}
\\
\hline
\sphinxAtStartPar
38
&
\sphinxAtStartPar
\sphinxcode{\sphinxupquote{60fe:0000:0000:7c56:0000:0000:c619:0000}}
&
\sphinxAtStartPar
\sphinxcode{\sphinxupquote{60fe::7c56:0:0:c619:0}}
\\
\hline
\sphinxAtStartPar
39
&
\sphinxAtStartPar
\sphinxcode{\sphinxupquote{ccb3:0000:c821:0000:0000:0000:0000:c74c}}
&
\sphinxAtStartPar
\sphinxcode{\sphinxupquote{ccb3:0:c821::c74c}}
\\
\hline
\sphinxAtStartPar
40
&
\sphinxAtStartPar
\sphinxcode{\sphinxupquote{0000:0000:0b95:21ea:0000:0000:0000:0000}}
&
\sphinxAtStartPar
\sphinxcode{\sphinxupquote{0:0:b95:21ea::}}
\\
\hline
\sphinxAtStartPar
41
&
\sphinxAtStartPar
\sphinxcode{\sphinxupquote{0000:0000:c009:0000:4f26:0000:affb:53f3}}
&
\sphinxAtStartPar
\sphinxcode{\sphinxupquote{::c009:0:4f26:0:affb:53f3}}
\\
\hline
\sphinxAtStartPar
42
&
\sphinxAtStartPar
\sphinxcode{\sphinxupquote{0000:dcd2:71be:0000:734d:2e61:0000:9881}}
&
\sphinxAtStartPar
\sphinxcode{\sphinxupquote{0:dcd2:71be:0:734d:2e61:0:9881}}
\\
\hline
\sphinxAtStartPar
43
&
\sphinxAtStartPar
\sphinxcode{\sphinxupquote{bae4:eb02:0000:f41a:145d:bb47:0000:0000}}
&
\sphinxAtStartPar
\sphinxcode{\sphinxupquote{bae4:eb02:0:f41a:145d:bb47::}}
\\
\hline
\sphinxAtStartPar
44
&
\sphinxAtStartPar
\sphinxcode{\sphinxupquote{0000:0000:0000:0000:0000:740c:0000:1741}}
&
\sphinxAtStartPar
\sphinxcode{\sphinxupquote{::740c:0:1741}}
\\
\hline
\sphinxAtStartPar
45
&
\sphinxAtStartPar
\sphinxcode{\sphinxupquote{0000:0000:0000:0000:7cd8:1e15:c90f:ae1b}}
&
\sphinxAtStartPar
\sphinxcode{\sphinxupquote{::7cd8:1e15:c90f:ae1b}}
\\
\hline
\sphinxAtStartPar
46
&
\sphinxAtStartPar
\sphinxcode{\sphinxupquote{241c:8bb2:a902:dc92:1333:4bfb:0000:56c3}}
&
\sphinxAtStartPar
\sphinxcode{\sphinxupquote{241c:8bb2:a902:dc92:1333:4bfb:0:56c3}}
\\
\hline
\sphinxAtStartPar
47
&
\sphinxAtStartPar
\sphinxcode{\sphinxupquote{ec1f:2794:38cd:0000:e5f8:0000:c2cc:898c}}
&
\sphinxAtStartPar
\sphinxcode{\sphinxupquote{ec1f:2794:38cd:0:e5f8:0:c2cc:898c}}
\\
\hline
\sphinxAtStartPar
48
&
\sphinxAtStartPar
\sphinxcode{\sphinxupquote{86a3:f3ec:0000:0000:0000:c93a:b47c:0000}}
&
\sphinxAtStartPar
\sphinxcode{\sphinxupquote{86a3:f3ec::c93a:b47c:0}}
\\
\hline
\sphinxAtStartPar
49
&
\sphinxAtStartPar
\sphinxcode{\sphinxupquote{0000:c570:e19b:681a:0615:0000:0000:0000}}
&
\sphinxAtStartPar
\sphinxcode{\sphinxupquote{0:c570:e19b:681a:615::}}
\\
\hline
\sphinxAtStartPar
50
&
\sphinxAtStartPar
\sphinxcode{\sphinxupquote{0000:dc4e:0000:0000:0000:0000:0000:8ffd}}
&
\sphinxAtStartPar
\sphinxcode{\sphinxupquote{0:dc4e::8ffd}}
\\
\hline
\sphinxAtStartPar
51
&
\sphinxAtStartPar
\sphinxcode{\sphinxupquote{4ecb:68e9:e08d:a371:0000:0000:0000:0000}}
&
\sphinxAtStartPar
\sphinxcode{\sphinxupquote{4ecb:68e9:e08d:a371::}}
\\
\hline
\sphinxAtStartPar
52
&
\sphinxAtStartPar
\sphinxcode{\sphinxupquote{0000:0000:0000:64b2:0000:7e75:8bb1:ec30}}
&
\sphinxAtStartPar
\sphinxcode{\sphinxupquote{::64b2:0:7e75:8bb1:ec30}}
\\
\hline
\sphinxAtStartPar
53
&
\sphinxAtStartPar
\sphinxcode{\sphinxupquote{a7e1:6747:0bb4:0000:0000:0000:5bd7:0000}}
&
\sphinxAtStartPar
\sphinxcode{\sphinxupquote{a7e1:6747:bb4::5bd7:0}}
\\
\hline
\sphinxAtStartPar
54
&
\sphinxAtStartPar
\sphinxcode{\sphinxupquote{1803:0000:0000:0000:33e5:4828:0000:e00c}}
&
\sphinxAtStartPar
\sphinxcode{\sphinxupquote{1803::33e5:4828:0:e00c}}
\\
\hline
\sphinxAtStartPar
55
&
\sphinxAtStartPar
\sphinxcode{\sphinxupquote{0000:e083:0000:df9f:4d92:0000:0000:477d}}
&
\sphinxAtStartPar
\sphinxcode{\sphinxupquote{0:e083:0:df9f:4d92::477d}}
\\
\hline
\sphinxAtStartPar
56
&
\sphinxAtStartPar
\sphinxcode{\sphinxupquote{2001:0000:0000:f007:781f:0000:c2d5:a767}}
&
\sphinxAtStartPar
\sphinxcode{\sphinxupquote{2001::f007:781f:0:c2d5:a767}}
\\
\hline
\sphinxAtStartPar
57
&
\sphinxAtStartPar
\sphinxcode{\sphinxupquote{0000:94e8:0000:5a26:a616:b790:0000:d238}}
&
\sphinxAtStartPar
\sphinxcode{\sphinxupquote{0:94e8:0:5a26:a616:b790:0:d238}}
\\
\hline
\sphinxAtStartPar
58
&
\sphinxAtStartPar
\sphinxcode{\sphinxupquote{0000:1a4b:fd0e:0000:0000:e929:0000:0000}}
&
\sphinxAtStartPar
\sphinxcode{\sphinxupquote{0:1a4b:fd0e::e929:0:0}}
\\
\hline
\sphinxAtStartPar
59
&
\sphinxAtStartPar
\sphinxcode{\sphinxupquote{5c77:da9a:3305:39eb:0000:ade2:0750:7450}}
&
\sphinxAtStartPar
\sphinxcode{\sphinxupquote{5c77:da9a:3305:39eb:0:ade2:750:7450}}
\\
\hline
\sphinxAtStartPar
60
&
\sphinxAtStartPar
\sphinxcode{\sphinxupquote{7d09:0000:9d50:0000:33dc:0000:0000:445b}}
&
\sphinxAtStartPar
\sphinxcode{\sphinxupquote{7d09:0:9d50:0:33dc::445b}}
\\
\hline
\sphinxAtStartPar
61
&
\sphinxAtStartPar
\sphinxcode{\sphinxupquote{4b1a:0000:fd0b:f0c5:86be:0000:551c:0000}}
&
\sphinxAtStartPar
\sphinxcode{\sphinxupquote{4b1a:0:fd0b:f0c5:86be:0:551c:0}}
\\
\hline
\sphinxAtStartPar
62
&
\sphinxAtStartPar
\sphinxcode{\sphinxupquote{589e:0000:53c7:93e3:0000:0000:12e9:093c}}
&
\sphinxAtStartPar
\sphinxcode{\sphinxupquote{589e:0:53c7:93e3::12e9:93c}}
\\
\hline
\sphinxAtStartPar
63
&
\sphinxAtStartPar
\sphinxcode{\sphinxupquote{0000:3616:0000:8509:368f:6ffa:0000:0000}}
&
\sphinxAtStartPar
\sphinxcode{\sphinxupquote{0:3616:0:8509:368f:6ffa::}}
\\
\hline
\sphinxAtStartPar
64
&
\sphinxAtStartPar
\sphinxcode{\sphinxupquote{78cc:20e9:00e5:f0a9:eac3:0000:0000:0000}}
&
\sphinxAtStartPar
\sphinxcode{\sphinxupquote{78cc:20e9:e5:f0a9:eac3::}}
\\
\hline
\sphinxAtStartPar
65
&
\sphinxAtStartPar
\sphinxcode{\sphinxupquote{0000:0000:4d06:0000:0000:0000:b1dd:0000}}
&
\sphinxAtStartPar
\sphinxcode{\sphinxupquote{0:0:4d06::b1dd:0}}
\\
\hline
\sphinxAtStartPar
66
&
\sphinxAtStartPar
\sphinxcode{\sphinxupquote{0000:0000:0000:e4bf:0d24:d134:0b2a:100a}}
&
\sphinxAtStartPar
\sphinxcode{\sphinxupquote{::e4bf:d24:d134:b2a:100a}}
\\
\hline
\sphinxAtStartPar
67
&
\sphinxAtStartPar
\sphinxcode{\sphinxupquote{0000:0000:0000:0000:0000:0000:9159:8768}}
&
\sphinxAtStartPar
\sphinxcode{\sphinxupquote{::9159:8768}}
\\
\hline
\sphinxAtStartPar
68
&
\sphinxAtStartPar
\sphinxcode{\sphinxupquote{0000:0000:0000:d4b0:bea8:0000:abdb:f2f7}}
&
\sphinxAtStartPar
\sphinxcode{\sphinxupquote{::d4b0:bea8:0:abdb:f2f7}}
\\
\hline
\sphinxAtStartPar
69
&
\sphinxAtStartPar
\sphinxcode{\sphinxupquote{dc62:0000:5b76:0000:01e3:77fb:0000:0000}}
&
\sphinxAtStartPar
\sphinxcode{\sphinxupquote{dc62:0:5b76:0:1e3:77fb::}}
\\
\hline
\sphinxAtStartPar
70
&
\sphinxAtStartPar
\sphinxcode{\sphinxupquote{0000:c781:0000:f950:0000:2451:7b8a:0000}}
&
\sphinxAtStartPar
\sphinxcode{\sphinxupquote{0:c781:0:f950:0:2451:7b8a:0}}
\\
\hline
\sphinxAtStartPar
71
&
\sphinxAtStartPar
\sphinxcode{\sphinxupquote{4deb:e14c:c9b0:e65c:2265:0000:0000:29b3}}
&
\sphinxAtStartPar
\sphinxcode{\sphinxupquote{4deb:e14c:c9b0:e65c:2265::29b3}}
\\
\hline
\sphinxAtStartPar
72
&
\sphinxAtStartPar
\sphinxcode{\sphinxupquote{0000:3ebe:5d28:0000:7697:0000:0000:1708}}
&
\sphinxAtStartPar
\sphinxcode{\sphinxupquote{0:3ebe:5d28:0:7697::1708}}
\\
\hline
\sphinxAtStartPar
73
&
\sphinxAtStartPar
\sphinxcode{\sphinxupquote{0000:0000:0000:f141:0000:215a:0000:0000}}
&
\sphinxAtStartPar
\sphinxcode{\sphinxupquote{::f141:0:215a:0:0}}
\\
\hline
\sphinxAtStartPar
74
&
\sphinxAtStartPar
\sphinxcode{\sphinxupquote{e64a:ff9f:eb7c:923b:3a5f:0000:0000:0000}}
&
\sphinxAtStartPar
\sphinxcode{\sphinxupquote{e64a:ff9f:eb7c:923b:3a5f::}}
\\
\hline
\sphinxAtStartPar
75
&
\sphinxAtStartPar
\sphinxcode{\sphinxupquote{36a8:58be:9b67:be76:66c3:0000:90a5:0000}}
&
\sphinxAtStartPar
\sphinxcode{\sphinxupquote{36a8:58be:9b67:be76:66c3:0:90a5:0}}
\\
\hline
\sphinxAtStartPar
76
&
\sphinxAtStartPar
\sphinxcode{\sphinxupquote{b096:0000:0000:0000:7291:0000:eefd:0000}}
&
\sphinxAtStartPar
\sphinxcode{\sphinxupquote{b096::7291:0:eefd:0}}
\\
\hline
\sphinxAtStartPar
77
&
\sphinxAtStartPar
\sphinxcode{\sphinxupquote{0000:0000:036c:0000:e583:d41a:956c:394f}}
&
\sphinxAtStartPar
\sphinxcode{\sphinxupquote{::36c:0:e583:d41a:956c:394f}}
\\
\hline
\sphinxAtStartPar
78
&
\sphinxAtStartPar
\sphinxcode{\sphinxupquote{79e4:8350:0000:6d5b:cac2:0000:dd6b:e62a}}
&
\sphinxAtStartPar
\sphinxcode{\sphinxupquote{79e4:8350:0:6d5b:cac2:0:dd6b:e62a}}
\\
\hline
\sphinxAtStartPar
79
&
\sphinxAtStartPar
\sphinxcode{\sphinxupquote{34f5:0000:2ecb:0000:0000:27c6:0000:0000}}
&
\sphinxAtStartPar
\sphinxcode{\sphinxupquote{34f5:0:2ecb::27c6:0:0}}
\\
\hline
\sphinxAtStartPar
80
&
\sphinxAtStartPar
\sphinxcode{\sphinxupquote{b2d2:b338:0000:0000:3160:20bc:f4e6:5878}}
&
\sphinxAtStartPar
\sphinxcode{\sphinxupquote{b2d2:b338::3160:20bc:f4e6:5878}}
\\
\hline
\sphinxAtStartPar
81
&
\sphinxAtStartPar
\sphinxcode{\sphinxupquote{0000:0000:0000:4ddc:0000:1646:d72c:0000}}
&
\sphinxAtStartPar
\sphinxcode{\sphinxupquote{::4ddc:0:1646:d72c:0}}
\\
\hline
\sphinxAtStartPar
82
&
\sphinxAtStartPar
\sphinxcode{\sphinxupquote{0000:e0bb:111c:0000:0000:6fd4:0000:8891}}
&
\sphinxAtStartPar
\sphinxcode{\sphinxupquote{0:e0bb:111c::6fd4:0:8891}}
\\
\hline
\sphinxAtStartPar
83
&
\sphinxAtStartPar
\sphinxcode{\sphinxupquote{c04b:0000:0000:bac7:0000:e028:0000:c8e5}}
&
\sphinxAtStartPar
\sphinxcode{\sphinxupquote{c04b::bac7:0:e028:0:c8e5}}
\\
\hline
\sphinxAtStartPar
84
&
\sphinxAtStartPar
\sphinxcode{\sphinxupquote{f680:0000:0000:0000:0000:0000:dfa9:0000}}
&
\sphinxAtStartPar
\sphinxcode{\sphinxupquote{f680::dfa9:0}}
\\
\hline
\sphinxAtStartPar
85
&
\sphinxAtStartPar
\sphinxcode{\sphinxupquote{e19b:0000:101a:3fcc:ae97:0000:7970:f214}}
&
\sphinxAtStartPar
\sphinxcode{\sphinxupquote{e19b:0:101a:3fcc:ae97:0:7970:f214}}
\\
\hline
\sphinxAtStartPar
86
&
\sphinxAtStartPar
\sphinxcode{\sphinxupquote{50c1:9e9c:0000:0000:ed17:0000:8e99:0000}}
&
\sphinxAtStartPar
\sphinxcode{\sphinxupquote{50c1:9e9c::ed17:0:8e99:0}}
\\
\hline
\sphinxAtStartPar
87
&
\sphinxAtStartPar
\sphinxcode{\sphinxupquote{0000:ab50:4066:2809:f314:0000:92da:0000}}
&
\sphinxAtStartPar
\sphinxcode{\sphinxupquote{0:ab50:4066:2809:f314:0:92da:0}}
\\
\hline
\sphinxAtStartPar
88
&
\sphinxAtStartPar
\sphinxcode{\sphinxupquote{ff48:0000:0000:22e7:9656:0000:0000:0000}}
&
\sphinxAtStartPar
\sphinxcode{\sphinxupquote{ff48:0:0:22e7:9656::}}
\\
\hline
\sphinxAtStartPar
89
&
\sphinxAtStartPar
\sphinxcode{\sphinxupquote{0000:0000:0000:0000:0000:0000:14a5:0000}}
&
\sphinxAtStartPar
\sphinxcode{\sphinxupquote{::14a5:0}}
\\
\hline
\sphinxAtStartPar
90
&
\sphinxAtStartPar
\sphinxcode{\sphinxupquote{8d13:0000:237a:c4d7:0000:0000:4df0:c8d0}}
&
\sphinxAtStartPar
\sphinxcode{\sphinxupquote{8d13:0:237a:c4d7::4df0:c8d0}}
\\
\hline
\sphinxAtStartPar
91
&
\sphinxAtStartPar
\sphinxcode{\sphinxupquote{4a7e:caaa:0000:0000:ec08:ce1f:0000:0000}}
&
\sphinxAtStartPar
\sphinxcode{\sphinxupquote{4a7e:caaa::ec08:ce1f:0:0}}
\\
\hline
\sphinxAtStartPar
92
&
\sphinxAtStartPar
\sphinxcode{\sphinxupquote{1c21:0000:0000:0000:0000:0000:e5c0:fc84}}
&
\sphinxAtStartPar
\sphinxcode{\sphinxupquote{1c21::e5c0:fc84}}
\\
\hline
\sphinxAtStartPar
93
&
\sphinxAtStartPar
\sphinxcode{\sphinxupquote{0000:0000:32da:419f:0000:5b69:dad0:bc58}}
&
\sphinxAtStartPar
\sphinxcode{\sphinxupquote{::32da:419f:0:5b69:dad0:bc58}}
\\
\hline
\sphinxAtStartPar
94
&
\sphinxAtStartPar
\sphinxcode{\sphinxupquote{e73c:b036:3efd:0000:0000:0d87:0000:6197}}
&
\sphinxAtStartPar
\sphinxcode{\sphinxupquote{e73c:b036:3efd::d87:0:6197}}
\\
\hline
\sphinxAtStartPar
95
&
\sphinxAtStartPar
\sphinxcode{\sphinxupquote{0000:0000:0000:5bb0:bf99:0000:a21e:0000}}
&
\sphinxAtStartPar
\sphinxcode{\sphinxupquote{::5bb0:bf99:0:a21e:0}}
\\
\hline
\sphinxAtStartPar
96
&
\sphinxAtStartPar
\sphinxcode{\sphinxupquote{0000:9a47:5197:a901:0000:0000:3ac3:39c8}}
&
\sphinxAtStartPar
\sphinxcode{\sphinxupquote{0:9a47:5197:a901::3ac3:39c8}}
\\
\hline
\sphinxAtStartPar
97
&
\sphinxAtStartPar
\sphinxcode{\sphinxupquote{0000:14e3:0000:0000:06d0:e328:20a4:ea05}}
&
\sphinxAtStartPar
\sphinxcode{\sphinxupquote{0:14e3::6d0:e328:20a4:ea05}}
\\
\hline
\sphinxAtStartPar
98
&
\sphinxAtStartPar
\sphinxcode{\sphinxupquote{d9b4:e5de:7478:a8ac:2a19:3ef6:a970:0000}}
&
\sphinxAtStartPar
\sphinxcode{\sphinxupquote{d9b4:e5de:7478:a8ac:2a19:3ef6:a970:0}}
\\
\hline
\sphinxAtStartPar
99
&
\sphinxAtStartPar
\sphinxcode{\sphinxupquote{dcd1:0000:a0df:0000:0000:f58a:0000:f323}}
&
\sphinxAtStartPar
\sphinxcode{\sphinxupquote{dcd1:0:a0df::f58a:0:f323}}
\\
\hline
\sphinxAtStartPar
100
&
\sphinxAtStartPar
\sphinxcode{\sphinxupquote{0000:021d:64f1:df12:e8ac:0000:489f:75a0}}
&
\sphinxAtStartPar
\sphinxcode{\sphinxupquote{0:21d:64f1:df12:e8ac:0:489f:75a0}}
\\
\hline
\end{longtable}\sphinxatlongtableend\end{savenotes}


\chapter{Anexo: ejercicios sobre clasificación de direcciones IPv6}
\label{\detokenize{t2_integracion_elementos/apuntes_t2:anexo-ejercicios-sobre-clasificacion-de-direcciones-ipv6}}
\sphinxAtStartPar
Dadas las siguientes direcciones IPv6 indica de qué tipo son:
\begin{enumerate}
\sphinxsetlistlabels{\arabic}{enumi}{enumii}{}{.}%
\item {} 
\sphinxAtStartPar
\sphinxcode{\sphinxupquote{feab:e7b8:6626:0:a16e:0:6efe:995f}}

\item {} 
\sphinxAtStartPar
\sphinxcode{\sphinxupquote{fdf6:22e5:e01:0:9af6::}}

\item {} 
\sphinxAtStartPar
\sphinxcode{\sphinxupquote{fd53:a56e:0:950d:4bc::3c92}}

\item {} 
\sphinxAtStartPar
\sphinxcode{\sphinxupquote{ff22:a6f1::933c:ff7b:f150}}

\item {} 
\sphinxAtStartPar
\sphinxcode{\sphinxupquote{fea3:0:d97a::eac:bbb1:c90b}}

\item {} 
\sphinxAtStartPar
\sphinxcode{\sphinxupquote{fd67:0:3456:e82b:4617:138d:936:3834}}

\item {} 
\sphinxAtStartPar
\sphinxcode{\sphinxupquote{fdfb:1935::25cf:4987:0:0}}

\item {} 
\sphinxAtStartPar
\sphinxcode{\sphinxupquote{2277:2700:2b57:0:4d38::}}

\item {} 
\sphinxAtStartPar
\sphinxcode{\sphinxupquote{ffad:bdd2:3e17:0:98f7:7f6b::}}

\item {} 
\sphinxAtStartPar
\sphinxcode{\sphinxupquote{3415::}}

\item {} 
\sphinxAtStartPar
\sphinxcode{\sphinxupquote{ff6d:0:6a25:df12:88d:5fa6::}}

\item {} 
\sphinxAtStartPar
\sphinxcode{\sphinxupquote{ff22::71ac:0:aca}}

\item {} 
\sphinxAtStartPar
\sphinxcode{\sphinxupquote{fc0a:0:cbf3:70ae:e72c:0:c4db:0}}

\item {} 
\sphinxAtStartPar
\sphinxcode{\sphinxupquote{26bc:0:49e::5f84:d8c5:0}}

\item {} 
\sphinxAtStartPar
\sphinxcode{\sphinxupquote{fe90:af2c:c2d6:195e:bc86:0:4f08:f7ca}}

\item {} 
\sphinxAtStartPar
\sphinxcode{\sphinxupquote{2537:5b85:8e74:b4d0:b2fc::}}

\item {} 
\sphinxAtStartPar
\sphinxcode{\sphinxupquote{fe96::b88b:32f8:0:d025:0}}

\item {} 
\sphinxAtStartPar
\sphinxcode{\sphinxupquote{3170::98dc:d2c0:c886:0:0}}

\item {} 
\sphinxAtStartPar
\sphinxcode{\sphinxupquote{29ae:0:243:0:d814:1b69:f171:0}}

\item {} 
\sphinxAtStartPar
\sphinxcode{\sphinxupquote{274d:d1e0:19f2:0:fa26:fb94:529e:1378}}

\item {} 
\sphinxAtStartPar
\sphinxcode{\sphinxupquote{3c9e:6c30::71e3:0:0:8f68}}

\item {} 
\sphinxAtStartPar
\sphinxcode{\sphinxupquote{fd9b::a80f}}

\item {} 
\sphinxAtStartPar
\sphinxcode{\sphinxupquote{feb6:0:6c6d:c6dd:454c:4ea:d71e:cee}}

\item {} 
\sphinxAtStartPar
\sphinxcode{\sphinxupquote{fea1:e171:60e7:d8e7:4d43::df9}}

\item {} 
\sphinxAtStartPar
\sphinxcode{\sphinxupquote{fc98::67d0:0:68a5:0:0}}

\item {} 
\sphinxAtStartPar
\sphinxcode{\sphinxupquote{fea6:791a:c569:86a1:bf71:adcb:11:fe15}}

\item {} 
\sphinxAtStartPar
\sphinxcode{\sphinxupquote{fdfd:9440:8179:7eef:0:a606:0:1fe7}}

\item {} 
\sphinxAtStartPar
\sphinxcode{\sphinxupquote{3b2f:90fa:0:bb9e:13b:ab79::}}

\item {} 
\sphinxAtStartPar
\sphinxcode{\sphinxupquote{fea9:1415:0:e7bc:0:fbcd::}}

\item {} 
\sphinxAtStartPar
\sphinxcode{\sphinxupquote{ff5e:0:336f:0:7f53::}}

\item {} 
\sphinxAtStartPar
\sphinxcode{\sphinxupquote{fe83:dd6c:0:9584:3367:b654:75b3:0}}

\item {} 
\sphinxAtStartPar
\sphinxcode{\sphinxupquote{fc58:0:bc51:10a3::}}

\item {} 
\sphinxAtStartPar
\sphinxcode{\sphinxupquote{ff3a:e8a3::3c07:5a1e:0}}

\item {} 
\sphinxAtStartPar
\sphinxcode{\sphinxupquote{feae::2230:0:e8b}}

\item {} 
\sphinxAtStartPar
\sphinxcode{\sphinxupquote{ffcd:0:6ecc:9718:0:c20e::}}

\item {} 
\sphinxAtStartPar
\sphinxcode{\sphinxupquote{ff9e:0:246f:0:ed59::7fb4}}

\item {} 
\sphinxAtStartPar
\sphinxcode{\sphinxupquote{ff68:0:e756:95e1::67f1:0}}

\item {} 
\sphinxAtStartPar
\sphinxcode{\sphinxupquote{fc5e:d122:7da3:896e:626c::}}

\item {} 
\sphinxAtStartPar
\sphinxcode{\sphinxupquote{3bc1:0:ab12:85bf:2274::}}

\item {} 
\sphinxAtStartPar
\sphinxcode{\sphinxupquote{ff09:7ffd:d25c:2e26:aeac:3a45:0:f1e}}

\item {} 
\sphinxAtStartPar
\sphinxcode{\sphinxupquote{fd03:0:5db4:d114:2972:0:c484:0}}

\item {} 
\sphinxAtStartPar
\sphinxcode{\sphinxupquote{feac:cfe9::e3c9:0}}

\item {} 
\sphinxAtStartPar
\sphinxcode{\sphinxupquote{ff05::9c60:0:0:4b4a}}

\item {} 
\sphinxAtStartPar
\sphinxcode{\sphinxupquote{2499:0:aeac:0:12e:cbdf:7b95:0}}

\item {} 
\sphinxAtStartPar
\sphinxcode{\sphinxupquote{ff9f::5cc0}}

\item {} 
\sphinxAtStartPar
\sphinxcode{\sphinxupquote{fdc5:289e:0:c613::7add:6ea5}}

\item {} 
\sphinxAtStartPar
\sphinxcode{\sphinxupquote{ff79:8c::ece7:0:0}}

\item {} 
\sphinxAtStartPar
\sphinxcode{\sphinxupquote{fd16:8393:2506::9bb4:75be:0}}

\item {} 
\sphinxAtStartPar
\sphinxcode{\sphinxupquote{ff9c:875b:3f71:0:299e::}}

\item {} 
\sphinxAtStartPar
\sphinxcode{\sphinxupquote{fddd::1d58:0:bfa5:f060:0}}

\end{enumerate}


\section{Soluciones a la clasificaciones de direcciones IPv6}
\label{\detokenize{t2_integracion_elementos/apuntes_t2:soluciones-a-la-clasificaciones-de-direcciones-ipv6}}\begin{enumerate}
\sphinxsetlistlabels{\arabic}{enumi}{enumii}{}{.}%
\item {} 
\sphinxAtStartPar
\sphinxcode{\sphinxupquote{feab:e7b8:6626:0:a16e:0:6efe:995f}} es de tipo unicast local en enlace

\item {} 
\sphinxAtStartPar
\sphinxcode{\sphinxupquote{fdf6:22e5:e01:0:9af6::}} es de tipo unicast local único

\item {} 
\sphinxAtStartPar
\sphinxcode{\sphinxupquote{fd53:a56e:0:950d:4bc::3c92}} es de tipo unicast local único

\item {} 
\sphinxAtStartPar
\sphinxcode{\sphinxupquote{ff22:a6f1::933c:ff7b:f150}} es de tipo multicast

\item {} 
\sphinxAtStartPar
\sphinxcode{\sphinxupquote{fea3:0:d97a::eac:bbb1:c90b}} es de tipo unicast local en enlace

\item {} 
\sphinxAtStartPar
\sphinxcode{\sphinxupquote{fd67:0:3456:e82b:4617:138d:936:3834}} es de tipo unicast local único

\item {} 
\sphinxAtStartPar
\sphinxcode{\sphinxupquote{fdfb:1935::25cf:4987:0:0}} es de tipo unicast local único

\item {} 
\sphinxAtStartPar
\sphinxcode{\sphinxupquote{2277:2700:2b57:0:4d38::}} es de tipo unicast global

\item {} 
\sphinxAtStartPar
\sphinxcode{\sphinxupquote{ffad:bdd2:3e17:0:98f7:7f6b::}} es de tipo multicast

\item {} 
\sphinxAtStartPar
\sphinxcode{\sphinxupquote{3415::}} es de tipo unicast global

\item {} 
\sphinxAtStartPar
\sphinxcode{\sphinxupquote{ff6d:0:6a25:df12:88d:5fa6::}} es de tipo multicast

\item {} 
\sphinxAtStartPar
\sphinxcode{\sphinxupquote{ff22::71ac:0:aca}} es de tipo multicast

\item {} 
\sphinxAtStartPar
\sphinxcode{\sphinxupquote{fc0a:0:cbf3:70ae:e72c:0:c4db:0}} es de tipo unicast local único

\item {} 
\sphinxAtStartPar
\sphinxcode{\sphinxupquote{26bc:0:49e::5f84:d8c5:0}} es de tipo unicast global

\item {} 
\sphinxAtStartPar
\sphinxcode{\sphinxupquote{fe90:af2c:c2d6:195e:bc86:0:4f08:f7ca}} es de tipo unicast local en enlace

\item {} 
\sphinxAtStartPar
\sphinxcode{\sphinxupquote{2537:5b85:8e74:b4d0:b2fc::}} es de tipo unicast global

\item {} 
\sphinxAtStartPar
\sphinxcode{\sphinxupquote{fe96::b88b:32f8:0:d025:0}} es de tipo unicast local en enlace

\item {} 
\sphinxAtStartPar
\sphinxcode{\sphinxupquote{3170::98dc:d2c0:c886:0:0}} es de tipo unicast global

\item {} 
\sphinxAtStartPar
\sphinxcode{\sphinxupquote{29ae:0:243:0:d814:1b69:f171:0}} es de tipo unicast global

\item {} 
\sphinxAtStartPar
\sphinxcode{\sphinxupquote{274d:d1e0:19f2:0:fa26:fb94:529e:1378}} es de tipo unicast global

\item {} 
\sphinxAtStartPar
\sphinxcode{\sphinxupquote{3c9e:6c30::71e3:0:0:8f68}} es de tipo unicast global

\item {} 
\sphinxAtStartPar
\sphinxcode{\sphinxupquote{fd9b::a80f}} es de tipo unicast local único

\item {} 
\sphinxAtStartPar
\sphinxcode{\sphinxupquote{feb6:0:6c6d:c6dd:454c:4ea:d71e:cee}} es de tipo unicast local en enlace

\item {} 
\sphinxAtStartPar
\sphinxcode{\sphinxupquote{fea1:e171:60e7:d8e7:4d43::df9}} es de tipo unicast local en enlace

\item {} 
\sphinxAtStartPar
\sphinxcode{\sphinxupquote{fc98::67d0:0:68a5:0:0}} es de tipo unicast local único

\item {} 
\sphinxAtStartPar
\sphinxcode{\sphinxupquote{fea6:791a:c569:86a1:bf71:adcb:11:fe15}} es de tipo unicast local en enlace

\item {} 
\sphinxAtStartPar
\sphinxcode{\sphinxupquote{fdfd:9440:8179:7eef:0:a606:0:1fe7}} es de tipo unicast local único

\item {} 
\sphinxAtStartPar
\sphinxcode{\sphinxupquote{3b2f:90fa:0:bb9e:13b:ab79::}} es de tipo unicast global

\item {} 
\sphinxAtStartPar
\sphinxcode{\sphinxupquote{fea9:1415:0:e7bc:0:fbcd::}} es de tipo unicast local en enlace

\item {} 
\sphinxAtStartPar
\sphinxcode{\sphinxupquote{ff5e:0:336f:0:7f53::}} es de tipo multicast

\item {} 
\sphinxAtStartPar
\sphinxcode{\sphinxupquote{fe83:dd6c:0:9584:3367:b654:75b3:0}} es de tipo unicast local en enlace

\item {} 
\sphinxAtStartPar
\sphinxcode{\sphinxupquote{fc58:0:bc51:10a3::}} es de tipo unicast local único

\item {} 
\sphinxAtStartPar
\sphinxcode{\sphinxupquote{ff3a:e8a3::3c07:5a1e:0}} es de tipo multicast

\item {} 
\sphinxAtStartPar
\sphinxcode{\sphinxupquote{feae::2230:0:e8b}} es de tipo unicast local en enlace

\item {} 
\sphinxAtStartPar
\sphinxcode{\sphinxupquote{ffcd:0:6ecc:9718:0:c20e::}} es de tipo multicast

\item {} 
\sphinxAtStartPar
\sphinxcode{\sphinxupquote{ff9e:0:246f:0:ed59::7fb4}} es de tipo multicast

\item {} 
\sphinxAtStartPar
\sphinxcode{\sphinxupquote{ff68:0:e756:95e1::67f1:0}} es de tipo multicast

\item {} 
\sphinxAtStartPar
\sphinxcode{\sphinxupquote{fc5e:d122:7da3:896e:626c::}} es de tipo unicast local único

\item {} 
\sphinxAtStartPar
\sphinxcode{\sphinxupquote{3bc1:0:ab12:85bf:2274::}} es de tipo unicast global

\item {} 
\sphinxAtStartPar
\sphinxcode{\sphinxupquote{ff09:7ffd:d25c:2e26:aeac:3a45:0:f1e}} es de tipo multicast

\item {} 
\sphinxAtStartPar
\sphinxcode{\sphinxupquote{fd03:0:5db4:d114:2972:0:c484:0}} es de tipo unicast local único

\item {} 
\sphinxAtStartPar
\sphinxcode{\sphinxupquote{feac:cfe9::e3c9:0}} es de tipo unicast local en enlace

\item {} 
\sphinxAtStartPar
\sphinxcode{\sphinxupquote{ff05::9c60:0:0:4b4a}} es de tipo multicast

\item {} 
\sphinxAtStartPar
\sphinxcode{\sphinxupquote{2499:0:aeac:0:12e:cbdf:7b95:0}} es de tipo unicast global

\item {} 
\sphinxAtStartPar
\sphinxcode{\sphinxupquote{ff9f::5cc0}} es de tipo multicast

\item {} 
\sphinxAtStartPar
\sphinxcode{\sphinxupquote{fdc5:289e:0:c613::7add:6ea5}} es de tipo unicast local único

\item {} 
\sphinxAtStartPar
\sphinxcode{\sphinxupquote{ff79:8c::ece7:0:0}} es de tipo multicast

\item {} 
\sphinxAtStartPar
\sphinxcode{\sphinxupquote{fd16:8393:2506::9bb4:75be:0}} es de tipo unicast local único

\item {} 
\sphinxAtStartPar
\sphinxcode{\sphinxupquote{ff9c:875b:3f71:0:299e::}} es de tipo multicast

\item {} 
\sphinxAtStartPar
\sphinxcode{\sphinxupquote{fddd::1d58:0:bfa5:f060:0}} es de tipo unicast local único

\end{enumerate}


\chapter{Configuración y administración de conmutadores}
\label{\detokenize{t3_conmutadores/apuntes_t3:configuracion-y-administracion-de-conmutadores}}\label{\detokenize{t3_conmutadores/apuntes_t3::doc}}
\sphinxAtStartPar
Segmentación de la red.
Ventajas que presenta.
Conmutadores y dominios de colisión y  «broadcast».
Segmentación de redes.
Formas de conexión al conmutador para su configuración.
Configuración del conmutador.
Configuración estática y dinámica de la tabla de direcciones MAC.
Diagnóstico de incidencias del conmutador.
Las tormentas de «broadcast».
El protocolo Spanning\sphinxhyphen{}Tree.


\chapter{Configuración y administración básica de routers}
\label{\detokenize{t4_routers/apuntes_t4:configuracion-y-administracion-basica-de-routers}}\label{\detokenize{t4_routers/apuntes_t4::doc}}
\sphinxAtStartPar
Los «routers» en las LAN y en las WAN.
Componentes del router.
Formas de conexión al router para su configuración inicial.
Comandos para configuración del router.
Comandos para administración del router.
Configuración del enrutamiento estático.
Definición y ubicación de listas de control de acceso (ACLs).


\chapter{Configuración de redes virtuales VLAN}
\label{\detokenize{t5_vlans/apuntes_t5:configuracion-de-redes-virtuales-vlan}}\label{\detokenize{t5_vlans/apuntes_t5::doc}}
\sphinxAtStartPar
El diseño de redes locales a tres capas (núcleo, distribución y acceso).
Implantación y configuración de redes virtuales.
Diagnóstico de incidencias en redes virtuales.
Definición de enlaces troncales en los conmutadores y routers.
El protocolo IEEE802.1Q
Protocolos para la administración centralizada de redes virtuales; el protocolo VTP


\chapter{Configuración y administración de protocolos dinámicos}
\label{\detokenize{t6_enrutamiento_dinamico/apuntes_t6:configuracion-y-administracion-de-protocolos-dinamicos}}\label{\detokenize{t6_enrutamiento_dinamico/apuntes_t6::doc}}
\sphinxAtStartPar
Protocolos enrutables y protocolos de enrutamiento.
Protocolos de enrutamiento interior y exterior.
El enrutamiento sin clase.
La subdivisión de redes y el uso de máscaras de longitud variable (VLSM).
El protocolo RIPv2; comparación con RIPv1.
Configuración y administración de RIPv1.
Configuración y administración de RIPv2.
Diagnóstico de incidencias en Ripv2.
Los protocolos de enrutamiento estado\sphinxhyphen{}enlace
Configuración y administración en OSPF.
Diagnóstico de incidencias en OSPF.
Configuración y administración de protocolos de enrutamiento propietarios.


\chapter{Configuración del acceso a Internet desde una LAN}
\label{\detokenize{t7_acceso_internet/apuntes_t7:configuracion-del-acceso-a-internet-desde-una-lan}}\label{\detokenize{t7_acceso_internet/apuntes_t7::doc}}
\sphinxAtStartPar
Direccionamiento interno y direccionamiento externo.
NAT origen y NAT destino.NAT estático, dinámico, de sobrecarga (PAT) e inverso.
Configuración de NAT.
Diagnostico de incidencias de NAT.
Configuración de PAT.
Diagnóstico de fallos de PAT.
Introducción a las tecnologías WAN: Frame Relay, RDSI, ADSL.
Las tecnologías Wifi y Wimax.
Las tecnologías UMTS y HSDPA.
Tecnologías emergentes basadas en cable e inalámbricas.


\chapter{Anexo: Ejercicios propuestos para el tema 1}
\label{\detokenize{anexos/t1_ejercicios:anexo-ejercicios-propuestos-para-el-tema-1}}\label{\detokenize{anexos/t1_ejercicios::doc}}

\section{Conversión de números}
\label{\detokenize{anexos/t1_ejercicios:conversion-de-numeros}}
\sphinxAtStartPar
Convertir a binarios los siguientes números:
\begin{itemize}
\item {} 
\sphinxAtStartPar
43

\item {} 
\sphinxAtStartPar
67

\item {} 
\sphinxAtStartPar
95

\item {} 
\sphinxAtStartPar
121

\item {} 
\sphinxAtStartPar
193

\item {} 
\sphinxAtStartPar
217

\item {} 
\sphinxAtStartPar
675

\end{itemize}


\section{Soluciones a ejercicios seleccionados}
\label{\detokenize{anexos/t1_ejercicios:soluciones-a-ejercicios-seleccionados}}
\sphinxAtStartPar
Convertir a binarios los siguientes números:
\begin{itemize}
\item {} 
\sphinxAtStartPar
43: debe salir 0010.1011

\item {} 
\sphinxAtStartPar
67: debe salir 0100.0011

\item {} 
\sphinxAtStartPar
95: debe salir 0101.1111

\item {} 
\sphinxAtStartPar
121: debe salir 0111.1001

\item {} 
\sphinxAtStartPar
193: debe salir 1100.0001

\item {} 
\sphinxAtStartPar
217: debe salir 1101.1001

\item {} 
\sphinxAtStartPar
675: debe salir 0010.1010.0011

\end{itemize}


\chapter{Anexo: ejercicios sobre rangos de direcciones}
\label{\detokenize{t2_integracion_elementos/ejercicios_subredes_ipv4/rangos_direcciones:anexo-ejercicios-sobre-rangos-de-direcciones}}\label{\detokenize{t2_integracion_elementos/ejercicios_subredes_ipv4/rangos_direcciones::doc}}

\section{Ejercicio 1}
\label{\detokenize{t2_integracion_elementos/ejercicios_subredes_ipv4/rangos_direcciones:ejercicio-1}}
\sphinxAtStartPar
Obtener el rango de direcciones posible para 46.233.192.0/20


\section{Ejercicio 2}
\label{\detokenize{t2_integracion_elementos/ejercicios_subredes_ipv4/rangos_direcciones:ejercicio-2}}
\sphinxAtStartPar
Obtener el rango de direcciones posible para 199.134.141.0/25


\section{Ejercicio 3}
\label{\detokenize{t2_integracion_elementos/ejercicios_subredes_ipv4/rangos_direcciones:ejercicio-3}}
\sphinxAtStartPar
Obtener el rango de direcciones posible para 22.252.0.0/14


\section{Ejercicio 4}
\label{\detokenize{t2_integracion_elementos/ejercicios_subredes_ipv4/rangos_direcciones:ejercicio-4}}
\sphinxAtStartPar
Obtener el rango de direcciones posible para 79.168.0.0/14


\section{Ejercicio 5}
\label{\detokenize{t2_integracion_elementos/ejercicios_subredes_ipv4/rangos_direcciones:ejercicio-5}}
\sphinxAtStartPar
Obtener el rango de direcciones posible para 131.215.184.64/26


\section{Ejercicio 6}
\label{\detokenize{t2_integracion_elementos/ejercicios_subredes_ipv4/rangos_direcciones:ejercicio-6}}
\sphinxAtStartPar
Obtener el rango de direcciones posible para 214.122.164.192/26


\section{Ejercicio 7}
\label{\detokenize{t2_integracion_elementos/ejercicios_subredes_ipv4/rangos_direcciones:ejercicio-7}}
\sphinxAtStartPar
Obtener el rango de direcciones posible para 209.173.168.84/30


\section{Ejercicio 8}
\label{\detokenize{t2_integracion_elementos/ejercicios_subredes_ipv4/rangos_direcciones:ejercicio-8}}
\sphinxAtStartPar
Obtener el rango de direcciones posible para 153.111.128.0/17


\section{Ejercicio 9}
\label{\detokenize{t2_integracion_elementos/ejercicios_subredes_ipv4/rangos_direcciones:ejercicio-9}}
\sphinxAtStartPar
Obtener el rango de direcciones posible para 134.87.135.192/26


\section{Ejercicio 10}
\label{\detokenize{t2_integracion_elementos/ejercicios_subredes_ipv4/rangos_direcciones:ejercicio-10}}
\sphinxAtStartPar
Obtener el rango de direcciones posible para 161.198.239.168/29


\section{Ejercicio 11}
\label{\detokenize{t2_integracion_elementos/ejercicios_subredes_ipv4/rangos_direcciones:ejercicio-11}}
\sphinxAtStartPar
Obtener el rango de direcciones posible para 72.61.160.0/19


\section{Ejercicio 12}
\label{\detokenize{t2_integracion_elementos/ejercicios_subredes_ipv4/rangos_direcciones:ejercicio-12}}
\sphinxAtStartPar
Obtener el rango de direcciones posible para 137.170.96.0/20


\section{Ejercicio 13}
\label{\detokenize{t2_integracion_elementos/ejercicios_subredes_ipv4/rangos_direcciones:ejercicio-13}}
\sphinxAtStartPar
Obtener el rango de direcciones posible para 32.96.0.0/13


\section{Ejercicio 14}
\label{\detokenize{t2_integracion_elementos/ejercicios_subredes_ipv4/rangos_direcciones:ejercicio-14}}
\sphinxAtStartPar
Obtener el rango de direcciones posible para 44.128.0.0/10


\section{Ejercicio 15}
\label{\detokenize{t2_integracion_elementos/ejercicios_subredes_ipv4/rangos_direcciones:ejercicio-15}}
\sphinxAtStartPar
Obtener el rango de direcciones posible para 50.142.0.0/17


\section{Ejercicio 16}
\label{\detokenize{t2_integracion_elementos/ejercicios_subredes_ipv4/rangos_direcciones:ejercicio-16}}
\sphinxAtStartPar
Obtener el rango de direcciones posible para 57.128.0.0/10


\section{Ejercicio 17}
\label{\detokenize{t2_integracion_elementos/ejercicios_subredes_ipv4/rangos_direcciones:ejercicio-17}}
\sphinxAtStartPar
Obtener el rango de direcciones posible para 66.16.0.0/12


\section{Ejercicio 18}
\label{\detokenize{t2_integracion_elementos/ejercicios_subredes_ipv4/rangos_direcciones:ejercicio-18}}
\sphinxAtStartPar
Obtener el rango de direcciones posible para 163.46.54.128/25


\section{Ejercicio 19}
\label{\detokenize{t2_integracion_elementos/ejercicios_subredes_ipv4/rangos_direcciones:ejercicio-19}}
\sphinxAtStartPar
Obtener el rango de direcciones posible para 68.112.0.0/13


\section{Ejercicio 20}
\label{\detokenize{t2_integracion_elementos/ejercicios_subredes_ipv4/rangos_direcciones:ejercicio-20}}
\sphinxAtStartPar
Obtener el rango de direcciones posible para 171.51.209.0/25


\section{Ejercicio 21}
\label{\detokenize{t2_integracion_elementos/ejercicios_subredes_ipv4/rangos_direcciones:ejercicio-21}}
\sphinxAtStartPar
Obtener el rango de direcciones posible para 40.161.64.0/19


\section{Ejercicio 22}
\label{\detokenize{t2_integracion_elementos/ejercicios_subredes_ipv4/rangos_direcciones:ejercicio-22}}
\sphinxAtStartPar
Obtener el rango de direcciones posible para 35.150.240.0/20


\section{Ejercicio 23}
\label{\detokenize{t2_integracion_elementos/ejercicios_subredes_ipv4/rangos_direcciones:ejercicio-23}}
\sphinxAtStartPar
Obtener el rango de direcciones posible para 35.70.0.0/15


\section{Ejercicio 24}
\label{\detokenize{t2_integracion_elementos/ejercicios_subredes_ipv4/rangos_direcciones:ejercicio-24}}
\sphinxAtStartPar
Obtener el rango de direcciones posible para 4.8.0.0/15


\section{Ejercicio 25}
\label{\detokenize{t2_integracion_elementos/ejercicios_subredes_ipv4/rangos_direcciones:ejercicio-25}}
\sphinxAtStartPar
Obtener el rango de direcciones posible para 118.195.102.0/24


\section{Ejercicio 26}
\label{\detokenize{t2_integracion_elementos/ejercicios_subredes_ipv4/rangos_direcciones:ejercicio-26}}
\sphinxAtStartPar
Obtener el rango de direcciones posible para 56.127.224.0/19


\section{Ejercicio 27}
\label{\detokenize{t2_integracion_elementos/ejercicios_subredes_ipv4/rangos_direcciones:ejercicio-27}}
\sphinxAtStartPar
Obtener el rango de direcciones posible para 73.146.96.0/19


\section{Ejercicio 28}
\label{\detokenize{t2_integracion_elementos/ejercicios_subredes_ipv4/rangos_direcciones:ejercicio-28}}
\sphinxAtStartPar
Obtener el rango de direcciones posible para 130.122.104.0/22


\section{Ejercicio 29}
\label{\detokenize{t2_integracion_elementos/ejercicios_subredes_ipv4/rangos_direcciones:ejercicio-29}}
\sphinxAtStartPar
Obtener el rango de direcciones posible para 99.74.239.64/28


\section{Ejercicio 30}
\label{\detokenize{t2_integracion_elementos/ejercicios_subredes_ipv4/rangos_direcciones:ejercicio-30}}
\sphinxAtStartPar
Obtener el rango de direcciones posible para 92.32.0.0/12


\section{Ejercicio 31}
\label{\detokenize{t2_integracion_elementos/ejercicios_subredes_ipv4/rangos_direcciones:ejercicio-31}}
\sphinxAtStartPar
Obtener el rango de direcciones posible para 44.96.0.0/11


\section{Ejercicio 32}
\label{\detokenize{t2_integracion_elementos/ejercicios_subredes_ipv4/rangos_direcciones:ejercicio-32}}
\sphinxAtStartPar
Obtener el rango de direcciones posible para 152.129.48.0/21


\section{Ejercicio 33}
\label{\detokenize{t2_integracion_elementos/ejercicios_subredes_ipv4/rangos_direcciones:ejercicio-33}}
\sphinxAtStartPar
Obtener el rango de direcciones posible para 121.1.0.0/18


\section{Ejercicio 34}
\label{\detokenize{t2_integracion_elementos/ejercicios_subredes_ipv4/rangos_direcciones:ejercicio-34}}
\sphinxAtStartPar
Obtener el rango de direcciones posible para 179.127.48.0/20


\section{Ejercicio 35}
\label{\detokenize{t2_integracion_elementos/ejercicios_subredes_ipv4/rangos_direcciones:ejercicio-35}}
\sphinxAtStartPar
Obtener el rango de direcciones posible para 47.64.0.0/10


\section{Ejercicio 36}
\label{\detokenize{t2_integracion_elementos/ejercicios_subredes_ipv4/rangos_direcciones:ejercicio-36}}
\sphinxAtStartPar
Obtener el rango de direcciones posible para 70.100.128.0/17


\section{Ejercicio 37}
\label{\detokenize{t2_integracion_elementos/ejercicios_subredes_ipv4/rangos_direcciones:ejercicio-37}}
\sphinxAtStartPar
Obtener el rango de direcciones posible para 18.64.0.0/10


\section{Ejercicio 38}
\label{\detokenize{t2_integracion_elementos/ejercicios_subredes_ipv4/rangos_direcciones:ejercicio-38}}
\sphinxAtStartPar
Obtener el rango de direcciones posible para 110.0.0.0/12


\section{Ejercicio 39}
\label{\detokenize{t2_integracion_elementos/ejercicios_subredes_ipv4/rangos_direcciones:ejercicio-39}}
\sphinxAtStartPar
Obtener el rango de direcciones posible para 119.0.0.0/13


\section{Ejercicio 40}
\label{\detokenize{t2_integracion_elementos/ejercicios_subredes_ipv4/rangos_direcciones:ejercicio-40}}
\sphinxAtStartPar
Obtener el rango de direcciones posible para 77.244.0.0/14


\section{Ejercicio 41}
\label{\detokenize{t2_integracion_elementos/ejercicios_subredes_ipv4/rangos_direcciones:ejercicio-41}}
\sphinxAtStartPar
Obtener el rango de direcciones posible para 201.241.110.0/25


\section{Ejercicio 42}
\label{\detokenize{t2_integracion_elementos/ejercicios_subredes_ipv4/rangos_direcciones:ejercicio-42}}
\sphinxAtStartPar
Obtener el rango de direcciones posible para 39.192.0.0/10


\section{Ejercicio 43}
\label{\detokenize{t2_integracion_elementos/ejercicios_subredes_ipv4/rangos_direcciones:ejercicio-43}}
\sphinxAtStartPar
Obtener el rango de direcciones posible para 138.212.0.0/16


\section{Ejercicio 44}
\label{\detokenize{t2_integracion_elementos/ejercicios_subredes_ipv4/rangos_direcciones:ejercicio-44}}
\sphinxAtStartPar
Obtener el rango de direcciones posible para 137.240.128.0/20


\section{Ejercicio 45}
\label{\detokenize{t2_integracion_elementos/ejercicios_subredes_ipv4/rangos_direcciones:ejercicio-45}}
\sphinxAtStartPar
Obtener el rango de direcciones posible para 222.103.44.192/26


\section{Ejercicio 46}
\label{\detokenize{t2_integracion_elementos/ejercicios_subredes_ipv4/rangos_direcciones:ejercicio-46}}
\sphinxAtStartPar
Obtener el rango de direcciones posible para 140.218.36.252/30


\section{Ejercicio 47}
\label{\detokenize{t2_integracion_elementos/ejercicios_subredes_ipv4/rangos_direcciones:ejercicio-47}}
\sphinxAtStartPar
Obtener el rango de direcciones posible para 20.151.128.0/17


\section{Ejercicio 48}
\label{\detokenize{t2_integracion_elementos/ejercicios_subredes_ipv4/rangos_direcciones:ejercicio-48}}
\sphinxAtStartPar
Obtener el rango de direcciones posible para 146.9.224.0/19


\section{Ejercicio 49}
\label{\detokenize{t2_integracion_elementos/ejercicios_subredes_ipv4/rangos_direcciones:ejercicio-49}}
\sphinxAtStartPar
Obtener el rango de direcciones posible para 207.172.113.0/30


\section{Ejercicio 50}
\label{\detokenize{t2_integracion_elementos/ejercicios_subredes_ipv4/rangos_direcciones:ejercicio-50}}
\sphinxAtStartPar
Obtener el rango de direcciones posible para 71.240.0.0/13


\section{Ejercicio 1}
\label{\detokenize{t2_integracion_elementos/ejercicios_subredes_ipv4/rangos_direcciones:id1}}
\sphinxAtStartPar
Para el enunciado \sphinxstyleemphasis{«Obtener el rango de direcciones posible para 46.233.192.0/20»}, la solución sería:
\begin{enumerate}
\sphinxsetlistlabels{\arabic}{enumi}{enumii}{}{.}%
\item {} 
\sphinxAtStartPar
La primera IP, que sería la IP de la red, sería 46.233.192.0

\item {} 
\sphinxAtStartPar
La primera IP asignable sería la 46.233.192.1

\item {} 
\sphinxAtStartPar
La última IP asignable sería la 46.233.207.254

\item {} 
\sphinxAtStartPar
La última IP, que sería la de difusión, sería 46.233.207.255

\end{enumerate}


\section{Ejercicio 2}
\label{\detokenize{t2_integracion_elementos/ejercicios_subredes_ipv4/rangos_direcciones:id2}}
\sphinxAtStartPar
Para el enunciado \sphinxstyleemphasis{«Obtener el rango de direcciones posible para 199.134.141.0/25»}, la solución sería:
\begin{enumerate}
\sphinxsetlistlabels{\arabic}{enumi}{enumii}{}{.}%
\item {} 
\sphinxAtStartPar
La primera IP, que sería la IP de la red, sería 199.134.141.0

\item {} 
\sphinxAtStartPar
La primera IP asignable sería la 199.134.141.1

\item {} 
\sphinxAtStartPar
La última IP asignable sería la 199.134.141.126

\item {} 
\sphinxAtStartPar
La última IP, que sería la de difusión, sería 199.134.141.127

\end{enumerate}


\section{Ejercicio 3}
\label{\detokenize{t2_integracion_elementos/ejercicios_subredes_ipv4/rangos_direcciones:id3}}
\sphinxAtStartPar
Para el enunciado \sphinxstyleemphasis{«Obtener el rango de direcciones posible para 22.252.0.0/14»}, la solución sería:
\begin{enumerate}
\sphinxsetlistlabels{\arabic}{enumi}{enumii}{}{.}%
\item {} 
\sphinxAtStartPar
La primera IP, que sería la IP de la red, sería 22.252.0.0

\item {} 
\sphinxAtStartPar
La primera IP asignable sería la 22.252.0.1

\item {} 
\sphinxAtStartPar
La última IP asignable sería la 22.255.255.254

\item {} 
\sphinxAtStartPar
La última IP, que sería la de difusión, sería 22.255.255.255

\end{enumerate}


\section{Ejercicio 4}
\label{\detokenize{t2_integracion_elementos/ejercicios_subredes_ipv4/rangos_direcciones:id4}}
\sphinxAtStartPar
Para el enunciado \sphinxstyleemphasis{«Obtener el rango de direcciones posible para 79.168.0.0/14»}, la solución sería:
\begin{enumerate}
\sphinxsetlistlabels{\arabic}{enumi}{enumii}{}{.}%
\item {} 
\sphinxAtStartPar
La primera IP, que sería la IP de la red, sería 79.168.0.0

\item {} 
\sphinxAtStartPar
La primera IP asignable sería la 79.168.0.1

\item {} 
\sphinxAtStartPar
La última IP asignable sería la 79.171.255.254

\item {} 
\sphinxAtStartPar
La última IP, que sería la de difusión, sería 79.171.255.255

\end{enumerate}


\section{Ejercicio 5}
\label{\detokenize{t2_integracion_elementos/ejercicios_subredes_ipv4/rangos_direcciones:id5}}
\sphinxAtStartPar
Para el enunciado \sphinxstyleemphasis{«Obtener el rango de direcciones posible para 131.215.184.64/26»}, la solución sería:
\begin{enumerate}
\sphinxsetlistlabels{\arabic}{enumi}{enumii}{}{.}%
\item {} 
\sphinxAtStartPar
La primera IP, que sería la IP de la red, sería 131.215.184.64

\item {} 
\sphinxAtStartPar
La primera IP asignable sería la 131.215.184.65

\item {} 
\sphinxAtStartPar
La última IP asignable sería la 131.215.184.126

\item {} 
\sphinxAtStartPar
La última IP, que sería la de difusión, sería 131.215.184.127

\end{enumerate}


\section{Ejercicio 6}
\label{\detokenize{t2_integracion_elementos/ejercicios_subredes_ipv4/rangos_direcciones:id6}}
\sphinxAtStartPar
Para el enunciado \sphinxstyleemphasis{«Obtener el rango de direcciones posible para 214.122.164.192/26»}, la solución sería:
\begin{enumerate}
\sphinxsetlistlabels{\arabic}{enumi}{enumii}{}{.}%
\item {} 
\sphinxAtStartPar
La primera IP, que sería la IP de la red, sería 214.122.164.192

\item {} 
\sphinxAtStartPar
La primera IP asignable sería la 214.122.164.193

\item {} 
\sphinxAtStartPar
La última IP asignable sería la 214.122.164.254

\item {} 
\sphinxAtStartPar
La última IP, que sería la de difusión, sería 214.122.164.255

\end{enumerate}


\section{Ejercicio 7}
\label{\detokenize{t2_integracion_elementos/ejercicios_subredes_ipv4/rangos_direcciones:id7}}
\sphinxAtStartPar
Para el enunciado \sphinxstyleemphasis{«Obtener el rango de direcciones posible para 209.173.168.84/30»}, la solución sería:
\begin{enumerate}
\sphinxsetlistlabels{\arabic}{enumi}{enumii}{}{.}%
\item {} 
\sphinxAtStartPar
La primera IP, que sería la IP de la red, sería 209.173.168.84

\item {} 
\sphinxAtStartPar
La primera IP asignable sería la 209.173.168.85

\item {} 
\sphinxAtStartPar
La última IP asignable sería la 209.173.168.86

\item {} 
\sphinxAtStartPar
La última IP, que sería la de difusión, sería 209.173.168.87

\end{enumerate}


\section{Ejercicio 8}
\label{\detokenize{t2_integracion_elementos/ejercicios_subredes_ipv4/rangos_direcciones:id8}}
\sphinxAtStartPar
Para el enunciado \sphinxstyleemphasis{«Obtener el rango de direcciones posible para 153.111.128.0/17»}, la solución sería:
\begin{enumerate}
\sphinxsetlistlabels{\arabic}{enumi}{enumii}{}{.}%
\item {} 
\sphinxAtStartPar
La primera IP, que sería la IP de la red, sería 153.111.128.0

\item {} 
\sphinxAtStartPar
La primera IP asignable sería la 153.111.128.1

\item {} 
\sphinxAtStartPar
La última IP asignable sería la 153.111.255.254

\item {} 
\sphinxAtStartPar
La última IP, que sería la de difusión, sería 153.111.255.255

\end{enumerate}


\section{Ejercicio 9}
\label{\detokenize{t2_integracion_elementos/ejercicios_subredes_ipv4/rangos_direcciones:id9}}
\sphinxAtStartPar
Para el enunciado \sphinxstyleemphasis{«Obtener el rango de direcciones posible para 134.87.135.192/26»}, la solución sería:
\begin{enumerate}
\sphinxsetlistlabels{\arabic}{enumi}{enumii}{}{.}%
\item {} 
\sphinxAtStartPar
La primera IP, que sería la IP de la red, sería 134.87.135.192

\item {} 
\sphinxAtStartPar
La primera IP asignable sería la 134.87.135.193

\item {} 
\sphinxAtStartPar
La última IP asignable sería la 134.87.135.254

\item {} 
\sphinxAtStartPar
La última IP, que sería la de difusión, sería 134.87.135.255

\end{enumerate}


\section{Ejercicio 10}
\label{\detokenize{t2_integracion_elementos/ejercicios_subredes_ipv4/rangos_direcciones:id10}}
\sphinxAtStartPar
Para el enunciado \sphinxstyleemphasis{«Obtener el rango de direcciones posible para 161.198.239.168/29»}, la solución sería:
\begin{enumerate}
\sphinxsetlistlabels{\arabic}{enumi}{enumii}{}{.}%
\item {} 
\sphinxAtStartPar
La primera IP, que sería la IP de la red, sería 161.198.239.168

\item {} 
\sphinxAtStartPar
La primera IP asignable sería la 161.198.239.169

\item {} 
\sphinxAtStartPar
La última IP asignable sería la 161.198.239.174

\item {} 
\sphinxAtStartPar
La última IP, que sería la de difusión, sería 161.198.239.175

\end{enumerate}


\section{Ejercicio 11}
\label{\detokenize{t2_integracion_elementos/ejercicios_subredes_ipv4/rangos_direcciones:id11}}
\sphinxAtStartPar
Para el enunciado \sphinxstyleemphasis{«Obtener el rango de direcciones posible para 72.61.160.0/19»}, la solución sería:
\begin{enumerate}
\sphinxsetlistlabels{\arabic}{enumi}{enumii}{}{.}%
\item {} 
\sphinxAtStartPar
La primera IP, que sería la IP de la red, sería 72.61.160.0

\item {} 
\sphinxAtStartPar
La primera IP asignable sería la 72.61.160.1

\item {} 
\sphinxAtStartPar
La última IP asignable sería la 72.61.191.254

\item {} 
\sphinxAtStartPar
La última IP, que sería la de difusión, sería 72.61.191.255

\end{enumerate}


\section{Ejercicio 12}
\label{\detokenize{t2_integracion_elementos/ejercicios_subredes_ipv4/rangos_direcciones:id12}}
\sphinxAtStartPar
Para el enunciado \sphinxstyleemphasis{«Obtener el rango de direcciones posible para 137.170.96.0/20»}, la solución sería:
\begin{enumerate}
\sphinxsetlistlabels{\arabic}{enumi}{enumii}{}{.}%
\item {} 
\sphinxAtStartPar
La primera IP, que sería la IP de la red, sería 137.170.96.0

\item {} 
\sphinxAtStartPar
La primera IP asignable sería la 137.170.96.1

\item {} 
\sphinxAtStartPar
La última IP asignable sería la 137.170.111.254

\item {} 
\sphinxAtStartPar
La última IP, que sería la de difusión, sería 137.170.111.255

\end{enumerate}


\section{Ejercicio 13}
\label{\detokenize{t2_integracion_elementos/ejercicios_subredes_ipv4/rangos_direcciones:id13}}
\sphinxAtStartPar
Para el enunciado \sphinxstyleemphasis{«Obtener el rango de direcciones posible para 32.96.0.0/13»}, la solución sería:
\begin{enumerate}
\sphinxsetlistlabels{\arabic}{enumi}{enumii}{}{.}%
\item {} 
\sphinxAtStartPar
La primera IP, que sería la IP de la red, sería 32.96.0.0

\item {} 
\sphinxAtStartPar
La primera IP asignable sería la 32.96.0.1

\item {} 
\sphinxAtStartPar
La última IP asignable sería la 32.103.255.254

\item {} 
\sphinxAtStartPar
La última IP, que sería la de difusión, sería 32.103.255.255

\end{enumerate}


\section{Ejercicio 14}
\label{\detokenize{t2_integracion_elementos/ejercicios_subredes_ipv4/rangos_direcciones:id14}}
\sphinxAtStartPar
Para el enunciado \sphinxstyleemphasis{«Obtener el rango de direcciones posible para 44.128.0.0/10»}, la solución sería:
\begin{enumerate}
\sphinxsetlistlabels{\arabic}{enumi}{enumii}{}{.}%
\item {} 
\sphinxAtStartPar
La primera IP, que sería la IP de la red, sería 44.128.0.0

\item {} 
\sphinxAtStartPar
La primera IP asignable sería la 44.128.0.1

\item {} 
\sphinxAtStartPar
La última IP asignable sería la 44.191.255.254

\item {} 
\sphinxAtStartPar
La última IP, que sería la de difusión, sería 44.191.255.255

\end{enumerate}


\section{Ejercicio 15}
\label{\detokenize{t2_integracion_elementos/ejercicios_subredes_ipv4/rangos_direcciones:id15}}
\sphinxAtStartPar
Para el enunciado \sphinxstyleemphasis{«Obtener el rango de direcciones posible para 50.142.0.0/17»}, la solución sería:
\begin{enumerate}
\sphinxsetlistlabels{\arabic}{enumi}{enumii}{}{.}%
\item {} 
\sphinxAtStartPar
La primera IP, que sería la IP de la red, sería 50.142.0.0

\item {} 
\sphinxAtStartPar
La primera IP asignable sería la 50.142.0.1

\item {} 
\sphinxAtStartPar
La última IP asignable sería la 50.142.127.254

\item {} 
\sphinxAtStartPar
La última IP, que sería la de difusión, sería 50.142.127.255

\end{enumerate}


\section{Ejercicio 16}
\label{\detokenize{t2_integracion_elementos/ejercicios_subredes_ipv4/rangos_direcciones:id16}}
\sphinxAtStartPar
Para el enunciado \sphinxstyleemphasis{«Obtener el rango de direcciones posible para 57.128.0.0/10»}, la solución sería:
\begin{enumerate}
\sphinxsetlistlabels{\arabic}{enumi}{enumii}{}{.}%
\item {} 
\sphinxAtStartPar
La primera IP, que sería la IP de la red, sería 57.128.0.0

\item {} 
\sphinxAtStartPar
La primera IP asignable sería la 57.128.0.1

\item {} 
\sphinxAtStartPar
La última IP asignable sería la 57.191.255.254

\item {} 
\sphinxAtStartPar
La última IP, que sería la de difusión, sería 57.191.255.255

\end{enumerate}


\section{Ejercicio 17}
\label{\detokenize{t2_integracion_elementos/ejercicios_subredes_ipv4/rangos_direcciones:id17}}
\sphinxAtStartPar
Para el enunciado \sphinxstyleemphasis{«Obtener el rango de direcciones posible para 66.16.0.0/12»}, la solución sería:
\begin{enumerate}
\sphinxsetlistlabels{\arabic}{enumi}{enumii}{}{.}%
\item {} 
\sphinxAtStartPar
La primera IP, que sería la IP de la red, sería 66.16.0.0

\item {} 
\sphinxAtStartPar
La primera IP asignable sería la 66.16.0.1

\item {} 
\sphinxAtStartPar
La última IP asignable sería la 66.31.255.254

\item {} 
\sphinxAtStartPar
La última IP, que sería la de difusión, sería 66.31.255.255

\end{enumerate}


\section{Ejercicio 18}
\label{\detokenize{t2_integracion_elementos/ejercicios_subredes_ipv4/rangos_direcciones:id18}}
\sphinxAtStartPar
Para el enunciado \sphinxstyleemphasis{«Obtener el rango de direcciones posible para 163.46.54.128/25»}, la solución sería:
\begin{enumerate}
\sphinxsetlistlabels{\arabic}{enumi}{enumii}{}{.}%
\item {} 
\sphinxAtStartPar
La primera IP, que sería la IP de la red, sería 163.46.54.128

\item {} 
\sphinxAtStartPar
La primera IP asignable sería la 163.46.54.129

\item {} 
\sphinxAtStartPar
La última IP asignable sería la 163.46.54.254

\item {} 
\sphinxAtStartPar
La última IP, que sería la de difusión, sería 163.46.54.255

\end{enumerate}


\section{Ejercicio 19}
\label{\detokenize{t2_integracion_elementos/ejercicios_subredes_ipv4/rangos_direcciones:id19}}
\sphinxAtStartPar
Para el enunciado \sphinxstyleemphasis{«Obtener el rango de direcciones posible para 68.112.0.0/13»}, la solución sería:
\begin{enumerate}
\sphinxsetlistlabels{\arabic}{enumi}{enumii}{}{.}%
\item {} 
\sphinxAtStartPar
La primera IP, que sería la IP de la red, sería 68.112.0.0

\item {} 
\sphinxAtStartPar
La primera IP asignable sería la 68.112.0.1

\item {} 
\sphinxAtStartPar
La última IP asignable sería la 68.119.255.254

\item {} 
\sphinxAtStartPar
La última IP, que sería la de difusión, sería 68.119.255.255

\end{enumerate}


\section{Ejercicio 20}
\label{\detokenize{t2_integracion_elementos/ejercicios_subredes_ipv4/rangos_direcciones:id20}}
\sphinxAtStartPar
Para el enunciado \sphinxstyleemphasis{«Obtener el rango de direcciones posible para 171.51.209.0/25»}, la solución sería:
\begin{enumerate}
\sphinxsetlistlabels{\arabic}{enumi}{enumii}{}{.}%
\item {} 
\sphinxAtStartPar
La primera IP, que sería la IP de la red, sería 171.51.209.0

\item {} 
\sphinxAtStartPar
La primera IP asignable sería la 171.51.209.1

\item {} 
\sphinxAtStartPar
La última IP asignable sería la 171.51.209.126

\item {} 
\sphinxAtStartPar
La última IP, que sería la de difusión, sería 171.51.209.127

\end{enumerate}


\section{Ejercicio 21}
\label{\detokenize{t2_integracion_elementos/ejercicios_subredes_ipv4/rangos_direcciones:id21}}
\sphinxAtStartPar
Para el enunciado \sphinxstyleemphasis{«Obtener el rango de direcciones posible para 40.161.64.0/19»}, la solución sería:
\begin{enumerate}
\sphinxsetlistlabels{\arabic}{enumi}{enumii}{}{.}%
\item {} 
\sphinxAtStartPar
La primera IP, que sería la IP de la red, sería 40.161.64.0

\item {} 
\sphinxAtStartPar
La primera IP asignable sería la 40.161.64.1

\item {} 
\sphinxAtStartPar
La última IP asignable sería la 40.161.95.254

\item {} 
\sphinxAtStartPar
La última IP, que sería la de difusión, sería 40.161.95.255

\end{enumerate}


\section{Ejercicio 22}
\label{\detokenize{t2_integracion_elementos/ejercicios_subredes_ipv4/rangos_direcciones:id22}}
\sphinxAtStartPar
Para el enunciado \sphinxstyleemphasis{«Obtener el rango de direcciones posible para 35.150.240.0/20»}, la solución sería:
\begin{enumerate}
\sphinxsetlistlabels{\arabic}{enumi}{enumii}{}{.}%
\item {} 
\sphinxAtStartPar
La primera IP, que sería la IP de la red, sería 35.150.240.0

\item {} 
\sphinxAtStartPar
La primera IP asignable sería la 35.150.240.1

\item {} 
\sphinxAtStartPar
La última IP asignable sería la 35.150.255.254

\item {} 
\sphinxAtStartPar
La última IP, que sería la de difusión, sería 35.150.255.255

\end{enumerate}


\section{Ejercicio 23}
\label{\detokenize{t2_integracion_elementos/ejercicios_subredes_ipv4/rangos_direcciones:id23}}
\sphinxAtStartPar
Para el enunciado \sphinxstyleemphasis{«Obtener el rango de direcciones posible para 35.70.0.0/15»}, la solución sería:
\begin{enumerate}
\sphinxsetlistlabels{\arabic}{enumi}{enumii}{}{.}%
\item {} 
\sphinxAtStartPar
La primera IP, que sería la IP de la red, sería 35.70.0.0

\item {} 
\sphinxAtStartPar
La primera IP asignable sería la 35.70.0.1

\item {} 
\sphinxAtStartPar
La última IP asignable sería la 35.71.255.254

\item {} 
\sphinxAtStartPar
La última IP, que sería la de difusión, sería 35.71.255.255

\end{enumerate}


\section{Ejercicio 24}
\label{\detokenize{t2_integracion_elementos/ejercicios_subredes_ipv4/rangos_direcciones:id24}}
\sphinxAtStartPar
Para el enunciado \sphinxstyleemphasis{«Obtener el rango de direcciones posible para 4.8.0.0/15»}, la solución sería:
\begin{enumerate}
\sphinxsetlistlabels{\arabic}{enumi}{enumii}{}{.}%
\item {} 
\sphinxAtStartPar
La primera IP, que sería la IP de la red, sería 4.8.0.0

\item {} 
\sphinxAtStartPar
La primera IP asignable sería la 4.8.0.1

\item {} 
\sphinxAtStartPar
La última IP asignable sería la 4.9.255.254

\item {} 
\sphinxAtStartPar
La última IP, que sería la de difusión, sería 4.9.255.255

\end{enumerate}


\section{Ejercicio 25}
\label{\detokenize{t2_integracion_elementos/ejercicios_subredes_ipv4/rangos_direcciones:id25}}
\sphinxAtStartPar
Para el enunciado \sphinxstyleemphasis{«Obtener el rango de direcciones posible para 118.195.102.0/24»}, la solución sería:
\begin{enumerate}
\sphinxsetlistlabels{\arabic}{enumi}{enumii}{}{.}%
\item {} 
\sphinxAtStartPar
La primera IP, que sería la IP de la red, sería 118.195.102.0

\item {} 
\sphinxAtStartPar
La primera IP asignable sería la 118.195.102.1

\item {} 
\sphinxAtStartPar
La última IP asignable sería la 118.195.102.254

\item {} 
\sphinxAtStartPar
La última IP, que sería la de difusión, sería 118.195.102.255

\end{enumerate}


\section{Ejercicio 26}
\label{\detokenize{t2_integracion_elementos/ejercicios_subredes_ipv4/rangos_direcciones:id26}}
\sphinxAtStartPar
Para el enunciado \sphinxstyleemphasis{«Obtener el rango de direcciones posible para 56.127.224.0/19»}, la solución sería:
\begin{enumerate}
\sphinxsetlistlabels{\arabic}{enumi}{enumii}{}{.}%
\item {} 
\sphinxAtStartPar
La primera IP, que sería la IP de la red, sería 56.127.224.0

\item {} 
\sphinxAtStartPar
La primera IP asignable sería la 56.127.224.1

\item {} 
\sphinxAtStartPar
La última IP asignable sería la 56.127.255.254

\item {} 
\sphinxAtStartPar
La última IP, que sería la de difusión, sería 56.127.255.255

\end{enumerate}


\section{Ejercicio 27}
\label{\detokenize{t2_integracion_elementos/ejercicios_subredes_ipv4/rangos_direcciones:id27}}
\sphinxAtStartPar
Para el enunciado \sphinxstyleemphasis{«Obtener el rango de direcciones posible para 73.146.96.0/19»}, la solución sería:
\begin{enumerate}
\sphinxsetlistlabels{\arabic}{enumi}{enumii}{}{.}%
\item {} 
\sphinxAtStartPar
La primera IP, que sería la IP de la red, sería 73.146.96.0

\item {} 
\sphinxAtStartPar
La primera IP asignable sería la 73.146.96.1

\item {} 
\sphinxAtStartPar
La última IP asignable sería la 73.146.127.254

\item {} 
\sphinxAtStartPar
La última IP, que sería la de difusión, sería 73.146.127.255

\end{enumerate}


\section{Ejercicio 28}
\label{\detokenize{t2_integracion_elementos/ejercicios_subredes_ipv4/rangos_direcciones:id28}}
\sphinxAtStartPar
Para el enunciado \sphinxstyleemphasis{«Obtener el rango de direcciones posible para 130.122.104.0/22»}, la solución sería:
\begin{enumerate}
\sphinxsetlistlabels{\arabic}{enumi}{enumii}{}{.}%
\item {} 
\sphinxAtStartPar
La primera IP, que sería la IP de la red, sería 130.122.104.0

\item {} 
\sphinxAtStartPar
La primera IP asignable sería la 130.122.104.1

\item {} 
\sphinxAtStartPar
La última IP asignable sería la 130.122.107.254

\item {} 
\sphinxAtStartPar
La última IP, que sería la de difusión, sería 130.122.107.255

\end{enumerate}


\section{Ejercicio 29}
\label{\detokenize{t2_integracion_elementos/ejercicios_subredes_ipv4/rangos_direcciones:id29}}
\sphinxAtStartPar
Para el enunciado \sphinxstyleemphasis{«Obtener el rango de direcciones posible para 99.74.239.64/28»}, la solución sería:
\begin{enumerate}
\sphinxsetlistlabels{\arabic}{enumi}{enumii}{}{.}%
\item {} 
\sphinxAtStartPar
La primera IP, que sería la IP de la red, sería 99.74.239.64

\item {} 
\sphinxAtStartPar
La primera IP asignable sería la 99.74.239.65

\item {} 
\sphinxAtStartPar
La última IP asignable sería la 99.74.239.78

\item {} 
\sphinxAtStartPar
La última IP, que sería la de difusión, sería 99.74.239.79

\end{enumerate}


\section{Ejercicio 30}
\label{\detokenize{t2_integracion_elementos/ejercicios_subredes_ipv4/rangos_direcciones:id30}}
\sphinxAtStartPar
Para el enunciado \sphinxstyleemphasis{«Obtener el rango de direcciones posible para 92.32.0.0/12»}, la solución sería:
\begin{enumerate}
\sphinxsetlistlabels{\arabic}{enumi}{enumii}{}{.}%
\item {} 
\sphinxAtStartPar
La primera IP, que sería la IP de la red, sería 92.32.0.0

\item {} 
\sphinxAtStartPar
La primera IP asignable sería la 92.32.0.1

\item {} 
\sphinxAtStartPar
La última IP asignable sería la 92.47.255.254

\item {} 
\sphinxAtStartPar
La última IP, que sería la de difusión, sería 92.47.255.255

\end{enumerate}


\section{Ejercicio 31}
\label{\detokenize{t2_integracion_elementos/ejercicios_subredes_ipv4/rangos_direcciones:id31}}
\sphinxAtStartPar
Para el enunciado \sphinxstyleemphasis{«Obtener el rango de direcciones posible para 44.96.0.0/11»}, la solución sería:
\begin{enumerate}
\sphinxsetlistlabels{\arabic}{enumi}{enumii}{}{.}%
\item {} 
\sphinxAtStartPar
La primera IP, que sería la IP de la red, sería 44.96.0.0

\item {} 
\sphinxAtStartPar
La primera IP asignable sería la 44.96.0.1

\item {} 
\sphinxAtStartPar
La última IP asignable sería la 44.127.255.254

\item {} 
\sphinxAtStartPar
La última IP, que sería la de difusión, sería 44.127.255.255

\end{enumerate}


\section{Ejercicio 32}
\label{\detokenize{t2_integracion_elementos/ejercicios_subredes_ipv4/rangos_direcciones:id32}}
\sphinxAtStartPar
Para el enunciado \sphinxstyleemphasis{«Obtener el rango de direcciones posible para 152.129.48.0/21»}, la solución sería:
\begin{enumerate}
\sphinxsetlistlabels{\arabic}{enumi}{enumii}{}{.}%
\item {} 
\sphinxAtStartPar
La primera IP, que sería la IP de la red, sería 152.129.48.0

\item {} 
\sphinxAtStartPar
La primera IP asignable sería la 152.129.48.1

\item {} 
\sphinxAtStartPar
La última IP asignable sería la 152.129.55.254

\item {} 
\sphinxAtStartPar
La última IP, que sería la de difusión, sería 152.129.55.255

\end{enumerate}


\section{Ejercicio 33}
\label{\detokenize{t2_integracion_elementos/ejercicios_subredes_ipv4/rangos_direcciones:id33}}
\sphinxAtStartPar
Para el enunciado \sphinxstyleemphasis{«Obtener el rango de direcciones posible para 121.1.0.0/18»}, la solución sería:
\begin{enumerate}
\sphinxsetlistlabels{\arabic}{enumi}{enumii}{}{.}%
\item {} 
\sphinxAtStartPar
La primera IP, que sería la IP de la red, sería 121.1.0.0

\item {} 
\sphinxAtStartPar
La primera IP asignable sería la 121.1.0.1

\item {} 
\sphinxAtStartPar
La última IP asignable sería la 121.1.63.254

\item {} 
\sphinxAtStartPar
La última IP, que sería la de difusión, sería 121.1.63.255

\end{enumerate}


\section{Ejercicio 34}
\label{\detokenize{t2_integracion_elementos/ejercicios_subredes_ipv4/rangos_direcciones:id34}}
\sphinxAtStartPar
Para el enunciado \sphinxstyleemphasis{«Obtener el rango de direcciones posible para 179.127.48.0/20»}, la solución sería:
\begin{enumerate}
\sphinxsetlistlabels{\arabic}{enumi}{enumii}{}{.}%
\item {} 
\sphinxAtStartPar
La primera IP, que sería la IP de la red, sería 179.127.48.0

\item {} 
\sphinxAtStartPar
La primera IP asignable sería la 179.127.48.1

\item {} 
\sphinxAtStartPar
La última IP asignable sería la 179.127.63.254

\item {} 
\sphinxAtStartPar
La última IP, que sería la de difusión, sería 179.127.63.255

\end{enumerate}


\section{Ejercicio 35}
\label{\detokenize{t2_integracion_elementos/ejercicios_subredes_ipv4/rangos_direcciones:id35}}
\sphinxAtStartPar
Para el enunciado \sphinxstyleemphasis{«Obtener el rango de direcciones posible para 47.64.0.0/10»}, la solución sería:
\begin{enumerate}
\sphinxsetlistlabels{\arabic}{enumi}{enumii}{}{.}%
\item {} 
\sphinxAtStartPar
La primera IP, que sería la IP de la red, sería 47.64.0.0

\item {} 
\sphinxAtStartPar
La primera IP asignable sería la 47.64.0.1

\item {} 
\sphinxAtStartPar
La última IP asignable sería la 47.127.255.254

\item {} 
\sphinxAtStartPar
La última IP, que sería la de difusión, sería 47.127.255.255

\end{enumerate}


\section{Ejercicio 36}
\label{\detokenize{t2_integracion_elementos/ejercicios_subredes_ipv4/rangos_direcciones:id36}}
\sphinxAtStartPar
Para el enunciado \sphinxstyleemphasis{«Obtener el rango de direcciones posible para 70.100.128.0/17»}, la solución sería:
\begin{enumerate}
\sphinxsetlistlabels{\arabic}{enumi}{enumii}{}{.}%
\item {} 
\sphinxAtStartPar
La primera IP, que sería la IP de la red, sería 70.100.128.0

\item {} 
\sphinxAtStartPar
La primera IP asignable sería la 70.100.128.1

\item {} 
\sphinxAtStartPar
La última IP asignable sería la 70.100.255.254

\item {} 
\sphinxAtStartPar
La última IP, que sería la de difusión, sería 70.100.255.255

\end{enumerate}


\section{Ejercicio 37}
\label{\detokenize{t2_integracion_elementos/ejercicios_subredes_ipv4/rangos_direcciones:id37}}
\sphinxAtStartPar
Para el enunciado \sphinxstyleemphasis{«Obtener el rango de direcciones posible para 18.64.0.0/10»}, la solución sería:
\begin{enumerate}
\sphinxsetlistlabels{\arabic}{enumi}{enumii}{}{.}%
\item {} 
\sphinxAtStartPar
La primera IP, que sería la IP de la red, sería 18.64.0.0

\item {} 
\sphinxAtStartPar
La primera IP asignable sería la 18.64.0.1

\item {} 
\sphinxAtStartPar
La última IP asignable sería la 18.127.255.254

\item {} 
\sphinxAtStartPar
La última IP, que sería la de difusión, sería 18.127.255.255

\end{enumerate}


\section{Ejercicio 38}
\label{\detokenize{t2_integracion_elementos/ejercicios_subredes_ipv4/rangos_direcciones:id38}}
\sphinxAtStartPar
Para el enunciado \sphinxstyleemphasis{«Obtener el rango de direcciones posible para 110.0.0.0/12»}, la solución sería:
\begin{enumerate}
\sphinxsetlistlabels{\arabic}{enumi}{enumii}{}{.}%
\item {} 
\sphinxAtStartPar
La primera IP, que sería la IP de la red, sería 110.0.0.0

\item {} 
\sphinxAtStartPar
La primera IP asignable sería la 110.0.0.1

\item {} 
\sphinxAtStartPar
La última IP asignable sería la 110.15.255.254

\item {} 
\sphinxAtStartPar
La última IP, que sería la de difusión, sería 110.15.255.255

\end{enumerate}


\section{Ejercicio 39}
\label{\detokenize{t2_integracion_elementos/ejercicios_subredes_ipv4/rangos_direcciones:id39}}
\sphinxAtStartPar
Para el enunciado \sphinxstyleemphasis{«Obtener el rango de direcciones posible para 119.0.0.0/13»}, la solución sería:
\begin{enumerate}
\sphinxsetlistlabels{\arabic}{enumi}{enumii}{}{.}%
\item {} 
\sphinxAtStartPar
La primera IP, que sería la IP de la red, sería 119.0.0.0

\item {} 
\sphinxAtStartPar
La primera IP asignable sería la 119.0.0.1

\item {} 
\sphinxAtStartPar
La última IP asignable sería la 119.7.255.254

\item {} 
\sphinxAtStartPar
La última IP, que sería la de difusión, sería 119.7.255.255

\end{enumerate}


\section{Ejercicio 40}
\label{\detokenize{t2_integracion_elementos/ejercicios_subredes_ipv4/rangos_direcciones:id40}}
\sphinxAtStartPar
Para el enunciado \sphinxstyleemphasis{«Obtener el rango de direcciones posible para 77.244.0.0/14»}, la solución sería:
\begin{enumerate}
\sphinxsetlistlabels{\arabic}{enumi}{enumii}{}{.}%
\item {} 
\sphinxAtStartPar
La primera IP, que sería la IP de la red, sería 77.244.0.0

\item {} 
\sphinxAtStartPar
La primera IP asignable sería la 77.244.0.1

\item {} 
\sphinxAtStartPar
La última IP asignable sería la 77.247.255.254

\item {} 
\sphinxAtStartPar
La última IP, que sería la de difusión, sería 77.247.255.255

\end{enumerate}


\section{Ejercicio 41}
\label{\detokenize{t2_integracion_elementos/ejercicios_subredes_ipv4/rangos_direcciones:id41}}
\sphinxAtStartPar
Para el enunciado \sphinxstyleemphasis{«Obtener el rango de direcciones posible para 201.241.110.0/25»}, la solución sería:
\begin{enumerate}
\sphinxsetlistlabels{\arabic}{enumi}{enumii}{}{.}%
\item {} 
\sphinxAtStartPar
La primera IP, que sería la IP de la red, sería 201.241.110.0

\item {} 
\sphinxAtStartPar
La primera IP asignable sería la 201.241.110.1

\item {} 
\sphinxAtStartPar
La última IP asignable sería la 201.241.110.126

\item {} 
\sphinxAtStartPar
La última IP, que sería la de difusión, sería 201.241.110.127

\end{enumerate}


\section{Ejercicio 42}
\label{\detokenize{t2_integracion_elementos/ejercicios_subredes_ipv4/rangos_direcciones:id42}}
\sphinxAtStartPar
Para el enunciado \sphinxstyleemphasis{«Obtener el rango de direcciones posible para 39.192.0.0/10»}, la solución sería:
\begin{enumerate}
\sphinxsetlistlabels{\arabic}{enumi}{enumii}{}{.}%
\item {} 
\sphinxAtStartPar
La primera IP, que sería la IP de la red, sería 39.192.0.0

\item {} 
\sphinxAtStartPar
La primera IP asignable sería la 39.192.0.1

\item {} 
\sphinxAtStartPar
La última IP asignable sería la 39.255.255.254

\item {} 
\sphinxAtStartPar
La última IP, que sería la de difusión, sería 39.255.255.255

\end{enumerate}


\section{Ejercicio 43}
\label{\detokenize{t2_integracion_elementos/ejercicios_subredes_ipv4/rangos_direcciones:id43}}
\sphinxAtStartPar
Para el enunciado \sphinxstyleemphasis{«Obtener el rango de direcciones posible para 138.212.0.0/16»}, la solución sería:
\begin{enumerate}
\sphinxsetlistlabels{\arabic}{enumi}{enumii}{}{.}%
\item {} 
\sphinxAtStartPar
La primera IP, que sería la IP de la red, sería 138.212.0.0

\item {} 
\sphinxAtStartPar
La primera IP asignable sería la 138.212.0.1

\item {} 
\sphinxAtStartPar
La última IP asignable sería la 138.212.255.254

\item {} 
\sphinxAtStartPar
La última IP, que sería la de difusión, sería 138.212.255.255

\end{enumerate}


\section{Ejercicio 44}
\label{\detokenize{t2_integracion_elementos/ejercicios_subredes_ipv4/rangos_direcciones:id44}}
\sphinxAtStartPar
Para el enunciado \sphinxstyleemphasis{«Obtener el rango de direcciones posible para 137.240.128.0/20»}, la solución sería:
\begin{enumerate}
\sphinxsetlistlabels{\arabic}{enumi}{enumii}{}{.}%
\item {} 
\sphinxAtStartPar
La primera IP, que sería la IP de la red, sería 137.240.128.0

\item {} 
\sphinxAtStartPar
La primera IP asignable sería la 137.240.128.1

\item {} 
\sphinxAtStartPar
La última IP asignable sería la 137.240.143.254

\item {} 
\sphinxAtStartPar
La última IP, que sería la de difusión, sería 137.240.143.255

\end{enumerate}


\section{Ejercicio 45}
\label{\detokenize{t2_integracion_elementos/ejercicios_subredes_ipv4/rangos_direcciones:id45}}
\sphinxAtStartPar
Para el enunciado \sphinxstyleemphasis{«Obtener el rango de direcciones posible para 222.103.44.192/26»}, la solución sería:
\begin{enumerate}
\sphinxsetlistlabels{\arabic}{enumi}{enumii}{}{.}%
\item {} 
\sphinxAtStartPar
La primera IP, que sería la IP de la red, sería 222.103.44.192

\item {} 
\sphinxAtStartPar
La primera IP asignable sería la 222.103.44.193

\item {} 
\sphinxAtStartPar
La última IP asignable sería la 222.103.44.254

\item {} 
\sphinxAtStartPar
La última IP, que sería la de difusión, sería 222.103.44.255

\end{enumerate}


\section{Ejercicio 46}
\label{\detokenize{t2_integracion_elementos/ejercicios_subredes_ipv4/rangos_direcciones:id46}}
\sphinxAtStartPar
Para el enunciado \sphinxstyleemphasis{«Obtener el rango de direcciones posible para 140.218.36.252/30»}, la solución sería:
\begin{enumerate}
\sphinxsetlistlabels{\arabic}{enumi}{enumii}{}{.}%
\item {} 
\sphinxAtStartPar
La primera IP, que sería la IP de la red, sería 140.218.36.252

\item {} 
\sphinxAtStartPar
La primera IP asignable sería la 140.218.36.253

\item {} 
\sphinxAtStartPar
La última IP asignable sería la 140.218.36.254

\item {} 
\sphinxAtStartPar
La última IP, que sería la de difusión, sería 140.218.36.255

\end{enumerate}


\section{Ejercicio 47}
\label{\detokenize{t2_integracion_elementos/ejercicios_subredes_ipv4/rangos_direcciones:id47}}
\sphinxAtStartPar
Para el enunciado \sphinxstyleemphasis{«Obtener el rango de direcciones posible para 20.151.128.0/17»}, la solución sería:
\begin{enumerate}
\sphinxsetlistlabels{\arabic}{enumi}{enumii}{}{.}%
\item {} 
\sphinxAtStartPar
La primera IP, que sería la IP de la red, sería 20.151.128.0

\item {} 
\sphinxAtStartPar
La primera IP asignable sería la 20.151.128.1

\item {} 
\sphinxAtStartPar
La última IP asignable sería la 20.151.255.254

\item {} 
\sphinxAtStartPar
La última IP, que sería la de difusión, sería 20.151.255.255

\end{enumerate}


\section{Ejercicio 48}
\label{\detokenize{t2_integracion_elementos/ejercicios_subredes_ipv4/rangos_direcciones:id48}}
\sphinxAtStartPar
Para el enunciado \sphinxstyleemphasis{«Obtener el rango de direcciones posible para 146.9.224.0/19»}, la solución sería:
\begin{enumerate}
\sphinxsetlistlabels{\arabic}{enumi}{enumii}{}{.}%
\item {} 
\sphinxAtStartPar
La primera IP, que sería la IP de la red, sería 146.9.224.0

\item {} 
\sphinxAtStartPar
La primera IP asignable sería la 146.9.224.1

\item {} 
\sphinxAtStartPar
La última IP asignable sería la 146.9.255.254

\item {} 
\sphinxAtStartPar
La última IP, que sería la de difusión, sería 146.9.255.255

\end{enumerate}


\section{Ejercicio 49}
\label{\detokenize{t2_integracion_elementos/ejercicios_subredes_ipv4/rangos_direcciones:id49}}
\sphinxAtStartPar
Para el enunciado \sphinxstyleemphasis{«Obtener el rango de direcciones posible para 207.172.113.0/30»}, la solución sería:
\begin{enumerate}
\sphinxsetlistlabels{\arabic}{enumi}{enumii}{}{.}%
\item {} 
\sphinxAtStartPar
La primera IP, que sería la IP de la red, sería 207.172.113.0

\item {} 
\sphinxAtStartPar
La primera IP asignable sería la 207.172.113.1

\item {} 
\sphinxAtStartPar
La última IP asignable sería la 207.172.113.2

\item {} 
\sphinxAtStartPar
La última IP, que sería la de difusión, sería 207.172.113.3

\end{enumerate}


\section{Ejercicio 50}
\label{\detokenize{t2_integracion_elementos/ejercicios_subredes_ipv4/rangos_direcciones:id50}}
\sphinxAtStartPar
Para el enunciado \sphinxstyleemphasis{«Obtener el rango de direcciones posible para 71.240.0.0/13»}, la solución sería:
\begin{enumerate}
\sphinxsetlistlabels{\arabic}{enumi}{enumii}{}{.}%
\item {} 
\sphinxAtStartPar
La primera IP, que sería la IP de la red, sería 71.240.0.0

\item {} 
\sphinxAtStartPar
La primera IP asignable sería la 71.240.0.1

\item {} 
\sphinxAtStartPar
La última IP asignable sería la 71.247.255.254

\item {} 
\sphinxAtStartPar
La última IP, que sería la de difusión, sería 71.247.255.255

\end{enumerate}


\chapter{Anexo: ejercicios de enrutamiento}
\label{\detokenize{t2_integracion_elementos/ejercicios_subredes_ipv4/ejercicios_dos_router:anexo-ejercicios-de-enrutamiento}}\label{\detokenize{t2_integracion_elementos/ejercicios_subredes_ipv4/ejercicios_dos_router::doc}}

\section{Ejercicios con dos router}
\label{\detokenize{t2_integracion_elementos/ejercicios_subredes_ipv4/ejercicios_dos_router:ejercicios-con-dos-router}}
\sphinxAtStartPar
Para todos estos casos se puede asumir una arquitectura como la siguiente


\subsection{Ejercicio 1}
\label{\detokenize{t2_integracion_elementos/ejercicios_subredes_ipv4/ejercicios_dos_router:ejercicio-1}}
\sphinxAtStartPar
Dada la arquitectura de la red de la figura, asignar direcciones IP, máscaras, puertas de enlace y tablas de rutas de manera que haya conectividad entre todos
los nodos de la red. Se desean utilizar las siguientes redes:
\begin{itemize}
\item {} 
\sphinxAtStartPar
Red 63.192.0.0/11 en el área izquierda.

\item {} 
\sphinxAtStartPar
Red 3.244.0.0/16 en el área central.

\item {} 
\sphinxAtStartPar
Red 81.244.63.64/26 en el área derecha

\end{itemize}

\begin{figure}[htbp]
\centering

\noindent\sphinxincludegraphics{{RedDosRouters}.png}
\end{figure}

\sphinxAtStartPar
Aparte de eso, se desean respetar unos ciertos estándares:
\begin{itemize}
\item {} 
\sphinxAtStartPar
Los routers de acceso a red deben tener siempre la primera IP de la red.

\item {} 
\sphinxAtStartPar
Los routers de distribución (los centrales) deberán tener la primera IP en el punto izquierdo y la última en el derecho.

\end{itemize}


\subsection{Ejercicio 2}
\label{\detokenize{t2_integracion_elementos/ejercicios_subredes_ipv4/ejercicios_dos_router:ejercicio-2}}
\sphinxAtStartPar
Dada la arquitectura de la red de la figura, asignar direcciones IP, máscaras, puertas de enlace y tablas de rutas de manera que haya conectividad entre todos
los nodos de la red. Se desean utilizar las siguientes redes:
\begin{itemize}
\item {} 
\sphinxAtStartPar
Red 36.20.206.68/30 en el área izquierda.

\item {} 
\sphinxAtStartPar
Red 11.0.0.0/13 en el área central.

\item {} 
\sphinxAtStartPar
Red 49.32.0.0/11 en el área derecha

\end{itemize}

\begin{figure}[htbp]
\centering

\noindent\sphinxincludegraphics{{RedDosRouters}.png}
\end{figure}

\sphinxAtStartPar
Aparte de eso, se desean respetar unos ciertos estándares:
\begin{itemize}
\item {} 
\sphinxAtStartPar
Los routers de acceso a red deben tener siempre la primera IP de la red.

\item {} 
\sphinxAtStartPar
Los routers de distribución (los centrales) deberán tener la primera IP en el punto izquierdo y la última en el derecho.

\end{itemize}


\subsection{Ejercicio 3}
\label{\detokenize{t2_integracion_elementos/ejercicios_subredes_ipv4/ejercicios_dos_router:ejercicio-3}}
\sphinxAtStartPar
Dada la arquitectura de la red de la figura, asignar direcciones IP, máscaras, puertas de enlace y tablas de rutas de manera que haya conectividad entre todos
los nodos de la red. Se desean utilizar las siguientes redes:
\begin{itemize}
\item {} 
\sphinxAtStartPar
Red 26.30.192.0/19 en el área izquierda.

\item {} 
\sphinxAtStartPar
Red 156.92.185.128/25 en el área central.

\item {} 
\sphinxAtStartPar
Red 212.193.128.0/26 en el área derecha

\end{itemize}

\begin{figure}[htbp]
\centering

\noindent\sphinxincludegraphics{{RedDosRouters}.png}
\end{figure}

\sphinxAtStartPar
Aparte de eso, se desean respetar unos ciertos estándares:
\begin{itemize}
\item {} 
\sphinxAtStartPar
Los routers de acceso a red deben tener siempre la primera IP de la red.

\item {} 
\sphinxAtStartPar
Los routers de distribución (los centrales) deberán tener la primera IP en el punto izquierdo y la última en el derecho.

\end{itemize}


\subsection{Ejercicio 4}
\label{\detokenize{t2_integracion_elementos/ejercicios_subredes_ipv4/ejercicios_dos_router:ejercicio-4}}
\sphinxAtStartPar
Dada la arquitectura de la red de la figura, asignar direcciones IP, máscaras, puertas de enlace y tablas de rutas de manera que haya conectividad entre todos
los nodos de la red. Se desean utilizar las siguientes redes:
\begin{itemize}
\item {} 
\sphinxAtStartPar
Red 205.90.58.120/30 en el área izquierda.

\item {} 
\sphinxAtStartPar
Red 91.210.20.0/23 en el área central.

\item {} 
\sphinxAtStartPar
Red 199.39.201.32/28 en el área derecha

\end{itemize}

\begin{figure}[htbp]
\centering

\noindent\sphinxincludegraphics{{RedDosRouters}.png}
\end{figure}

\sphinxAtStartPar
Aparte de eso, se desean respetar unos ciertos estándares:
\begin{itemize}
\item {} 
\sphinxAtStartPar
Los routers de acceso a red deben tener siempre la primera IP de la red.

\item {} 
\sphinxAtStartPar
Los routers de distribución (los centrales) deberán tener la primera IP en el punto izquierdo y la última en el derecho.

\end{itemize}


\subsection{Ejercicio 5}
\label{\detokenize{t2_integracion_elementos/ejercicios_subredes_ipv4/ejercicios_dos_router:ejercicio-5}}
\sphinxAtStartPar
Dada la arquitectura de la red de la figura, asignar direcciones IP, máscaras, puertas de enlace y tablas de rutas de manera que haya conectividad entre todos
los nodos de la red. Se desean utilizar las siguientes redes:
\begin{itemize}
\item {} 
\sphinxAtStartPar
Red 37.82.32.0/22 en el área izquierda.

\item {} 
\sphinxAtStartPar
Red 191.91.0.0/18 en el área central.

\item {} 
\sphinxAtStartPar
Red 212.11.243.192/26 en el área derecha

\end{itemize}

\begin{figure}[htbp]
\centering

\noindent\sphinxincludegraphics{{RedDosRouters}.png}
\end{figure}

\sphinxAtStartPar
Aparte de eso, se desean respetar unos ciertos estándares:
\begin{itemize}
\item {} 
\sphinxAtStartPar
Los routers de acceso a red deben tener siempre la primera IP de la red.

\item {} 
\sphinxAtStartPar
Los routers de distribución (los centrales) deberán tener la primera IP en el punto izquierdo y la última en el derecho.

\end{itemize}


\subsection{Ejercicio 6}
\label{\detokenize{t2_integracion_elementos/ejercicios_subredes_ipv4/ejercicios_dos_router:ejercicio-6}}
\sphinxAtStartPar
Dada la arquitectura de la red de la figura, asignar direcciones IP, máscaras, puertas de enlace y tablas de rutas de manera que haya conectividad entre todos
los nodos de la red. Se desean utilizar las siguientes redes:
\begin{itemize}
\item {} 
\sphinxAtStartPar
Red 115.141.50.180/30 en el área izquierda.

\item {} 
\sphinxAtStartPar
Red 135.78.64.0/18 en el área central.

\item {} 
\sphinxAtStartPar
Red 130.24.143.0/24 en el área derecha

\end{itemize}

\begin{figure}[htbp]
\centering

\noindent\sphinxincludegraphics{{RedDosRouters}.png}
\end{figure}

\sphinxAtStartPar
Aparte de eso, se desean respetar unos ciertos estándares:
\begin{itemize}
\item {} 
\sphinxAtStartPar
Los routers de acceso a red deben tener siempre la primera IP de la red.

\item {} 
\sphinxAtStartPar
Los routers de distribución (los centrales) deberán tener la primera IP en el punto izquierdo y la última en el derecho.

\end{itemize}


\subsection{Ejercicio 7}
\label{\detokenize{t2_integracion_elementos/ejercicios_subredes_ipv4/ejercicios_dos_router:ejercicio-7}}
\sphinxAtStartPar
Dada la arquitectura de la red de la figura, asignar direcciones IP, máscaras, puertas de enlace y tablas de rutas de manera que haya conectividad entre todos
los nodos de la red. Se desean utilizar las siguientes redes:
\begin{itemize}
\item {} 
\sphinxAtStartPar
Red 131.129.160.0/19 en el área izquierda.

\item {} 
\sphinxAtStartPar
Red 213.63.0.96/27 en el área central.

\item {} 
\sphinxAtStartPar
Red 80.208.0.0/12 en el área derecha

\end{itemize}

\begin{figure}[htbp]
\centering

\noindent\sphinxincludegraphics{{RedDosRouters}.png}
\end{figure}

\sphinxAtStartPar
Aparte de eso, se desean respetar unos ciertos estándares:
\begin{itemize}
\item {} 
\sphinxAtStartPar
Los routers de acceso a red deben tener siempre la primera IP de la red.

\item {} 
\sphinxAtStartPar
Los routers de distribución (los centrales) deberán tener la primera IP en el punto izquierdo y la última en el derecho.

\end{itemize}


\subsection{Ejercicio 8}
\label{\detokenize{t2_integracion_elementos/ejercicios_subredes_ipv4/ejercicios_dos_router:ejercicio-8}}
\sphinxAtStartPar
Dada la arquitectura de la red de la figura, asignar direcciones IP, máscaras, puertas de enlace y tablas de rutas de manera que haya conectividad entre todos
los nodos de la red. Se desean utilizar las siguientes redes:
\begin{itemize}
\item {} 
\sphinxAtStartPar
Red 197.122.17.8/29 en el área izquierda.

\item {} 
\sphinxAtStartPar
Red 84.76.0.0/15 en el área central.

\item {} 
\sphinxAtStartPar
Red 58.144.0.0/13 en el área derecha

\end{itemize}

\begin{figure}[htbp]
\centering

\noindent\sphinxincludegraphics{{RedDosRouters}.png}
\end{figure}

\sphinxAtStartPar
Aparte de eso, se desean respetar unos ciertos estándares:
\begin{itemize}
\item {} 
\sphinxAtStartPar
Los routers de acceso a red deben tener siempre la primera IP de la red.

\item {} 
\sphinxAtStartPar
Los routers de distribución (los centrales) deberán tener la primera IP en el punto izquierdo y la última en el derecho.

\end{itemize}


\subsection{Ejercicio 9}
\label{\detokenize{t2_integracion_elementos/ejercicios_subredes_ipv4/ejercicios_dos_router:ejercicio-9}}
\sphinxAtStartPar
Dada la arquitectura de la red de la figura, asignar direcciones IP, máscaras, puertas de enlace y tablas de rutas de manera que haya conectividad entre todos
los nodos de la red. Se desean utilizar las siguientes redes:
\begin{itemize}
\item {} 
\sphinxAtStartPar
Red 161.126.229.152/29 en el área izquierda.

\item {} 
\sphinxAtStartPar
Red 182.100.101.232/30 en el área central.

\item {} 
\sphinxAtStartPar
Red 59.132.0.0/14 en el área derecha

\end{itemize}

\begin{figure}[htbp]
\centering

\noindent\sphinxincludegraphics{{RedDosRouters}.png}
\end{figure}

\sphinxAtStartPar
Aparte de eso, se desean respetar unos ciertos estándares:
\begin{itemize}
\item {} 
\sphinxAtStartPar
Los routers de acceso a red deben tener siempre la primera IP de la red.

\item {} 
\sphinxAtStartPar
Los routers de distribución (los centrales) deberán tener la primera IP en el punto izquierdo y la última en el derecho.

\end{itemize}


\subsection{Ejercicio 10}
\label{\detokenize{t2_integracion_elementos/ejercicios_subredes_ipv4/ejercicios_dos_router:ejercicio-10}}
\sphinxAtStartPar
Dada la arquitectura de la red de la figura, asignar direcciones IP, máscaras, puertas de enlace y tablas de rutas de manera que haya conectividad entre todos
los nodos de la red. Se desean utilizar las siguientes redes:
\begin{itemize}
\item {} 
\sphinxAtStartPar
Red 104.25.128.0/17 en el área izquierda.

\item {} 
\sphinxAtStartPar
Red 187.143.84.192/27 en el área central.

\item {} 
\sphinxAtStartPar
Red 142.92.128.0/17 en el área derecha

\end{itemize}

\begin{figure}[htbp]
\centering

\noindent\sphinxincludegraphics{{RedDosRouters}.png}
\end{figure}

\sphinxAtStartPar
Aparte de eso, se desean respetar unos ciertos estándares:
\begin{itemize}
\item {} 
\sphinxAtStartPar
Los routers de acceso a red deben tener siempre la primera IP de la red.

\item {} 
\sphinxAtStartPar
Los routers de distribución (los centrales) deberán tener la primera IP en el punto izquierdo y la última en el derecho.

\end{itemize}


\subsection{Ejercicio 11}
\label{\detokenize{t2_integracion_elementos/ejercicios_subredes_ipv4/ejercicios_dos_router:ejercicio-11}}
\sphinxAtStartPar
Dada la arquitectura de la red de la figura, asignar direcciones IP, máscaras, puertas de enlace y tablas de rutas de manera que haya conectividad entre todos
los nodos de la red. Se desean utilizar las siguientes redes:
\begin{itemize}
\item {} 
\sphinxAtStartPar
Red 36.64.0.0/10 en el área izquierda.

\item {} 
\sphinxAtStartPar
Red 177.55.8.0/21 en el área central.

\item {} 
\sphinxAtStartPar
Red 187.48.0.0/16 en el área derecha

\end{itemize}

\begin{figure}[htbp]
\centering

\noindent\sphinxincludegraphics{{RedDosRouters}.png}
\end{figure}

\sphinxAtStartPar
Aparte de eso, se desean respetar unos ciertos estándares:
\begin{itemize}
\item {} 
\sphinxAtStartPar
Los routers de acceso a red deben tener siempre la primera IP de la red.

\item {} 
\sphinxAtStartPar
Los routers de distribución (los centrales) deberán tener la primera IP en el punto izquierdo y la última en el derecho.

\end{itemize}


\subsection{Ejercicio 12}
\label{\detokenize{t2_integracion_elementos/ejercicios_subredes_ipv4/ejercicios_dos_router:ejercicio-12}}
\sphinxAtStartPar
Dada la arquitectura de la red de la figura, asignar direcciones IP, máscaras, puertas de enlace y tablas de rutas de manera que haya conectividad entre todos
los nodos de la red. Se desean utilizar las siguientes redes:
\begin{itemize}
\item {} 
\sphinxAtStartPar
Red 19.166.126.128/26 en el área izquierda.

\item {} 
\sphinxAtStartPar
Red 184.147.243.128/25 en el área central.

\item {} 
\sphinxAtStartPar
Red 109.101.192.0/18 en el área derecha

\end{itemize}

\begin{figure}[htbp]
\centering

\noindent\sphinxincludegraphics{{RedDosRouters}.png}
\end{figure}

\sphinxAtStartPar
Aparte de eso, se desean respetar unos ciertos estándares:
\begin{itemize}
\item {} 
\sphinxAtStartPar
Los routers de acceso a red deben tener siempre la primera IP de la red.

\item {} 
\sphinxAtStartPar
Los routers de distribución (los centrales) deberán tener la primera IP en el punto izquierdo y la última en el derecho.

\end{itemize}


\subsection{Ejercicio 13}
\label{\detokenize{t2_integracion_elementos/ejercicios_subredes_ipv4/ejercicios_dos_router:ejercicio-13}}
\sphinxAtStartPar
Dada la arquitectura de la red de la figura, asignar direcciones IP, máscaras, puertas de enlace y tablas de rutas de manera que haya conectividad entre todos
los nodos de la red. Se desean utilizar las siguientes redes:
\begin{itemize}
\item {} 
\sphinxAtStartPar
Red 19.64.0.0/10 en el área izquierda.

\item {} 
\sphinxAtStartPar
Red 152.188.64.0/18 en el área central.

\item {} 
\sphinxAtStartPar
Red 151.170.0.0/16 en el área derecha

\end{itemize}

\begin{figure}[htbp]
\centering

\noindent\sphinxincludegraphics{{RedDosRouters}.png}
\end{figure}

\sphinxAtStartPar
Aparte de eso, se desean respetar unos ciertos estándares:
\begin{itemize}
\item {} 
\sphinxAtStartPar
Los routers de acceso a red deben tener siempre la primera IP de la red.

\item {} 
\sphinxAtStartPar
Los routers de distribución (los centrales) deberán tener la primera IP en el punto izquierdo y la última en el derecho.

\end{itemize}


\subsection{Ejercicio 14}
\label{\detokenize{t2_integracion_elementos/ejercicios_subredes_ipv4/ejercicios_dos_router:ejercicio-14}}
\sphinxAtStartPar
Dada la arquitectura de la red de la figura, asignar direcciones IP, máscaras, puertas de enlace y tablas de rutas de manera que haya conectividad entre todos
los nodos de la red. Se desean utilizar las siguientes redes:
\begin{itemize}
\item {} 
\sphinxAtStartPar
Red 105.5.128.0/17 en el área izquierda.

\item {} 
\sphinxAtStartPar
Red 148.199.0.0/16 en el área central.

\item {} 
\sphinxAtStartPar
Red 76.163.23.88/29 en el área derecha

\end{itemize}

\begin{figure}[htbp]
\centering

\noindent\sphinxincludegraphics{{RedDosRouters}.png}
\end{figure}

\sphinxAtStartPar
Aparte de eso, se desean respetar unos ciertos estándares:
\begin{itemize}
\item {} 
\sphinxAtStartPar
Los routers de acceso a red deben tener siempre la primera IP de la red.

\item {} 
\sphinxAtStartPar
Los routers de distribución (los centrales) deberán tener la primera IP en el punto izquierdo y la última en el derecho.

\end{itemize}


\subsection{Ejercicio 15}
\label{\detokenize{t2_integracion_elementos/ejercicios_subredes_ipv4/ejercicios_dos_router:ejercicio-15}}
\sphinxAtStartPar
Dada la arquitectura de la red de la figura, asignar direcciones IP, máscaras, puertas de enlace y tablas de rutas de manera que haya conectividad entre todos
los nodos de la red. Se desean utilizar las siguientes redes:
\begin{itemize}
\item {} 
\sphinxAtStartPar
Red 89.128.0.0/11 en el área izquierda.

\item {} 
\sphinxAtStartPar
Red 173.190.20.0/22 en el área central.

\item {} 
\sphinxAtStartPar
Red 133.8.96.0/19 en el área derecha

\end{itemize}

\begin{figure}[htbp]
\centering

\noindent\sphinxincludegraphics{{RedDosRouters}.png}
\end{figure}

\sphinxAtStartPar
Aparte de eso, se desean respetar unos ciertos estándares:
\begin{itemize}
\item {} 
\sphinxAtStartPar
Los routers de acceso a red deben tener siempre la primera IP de la red.

\item {} 
\sphinxAtStartPar
Los routers de distribución (los centrales) deberán tener la primera IP en el punto izquierdo y la última en el derecho.

\end{itemize}


\subsection{Ejercicio 16}
\label{\detokenize{t2_integracion_elementos/ejercicios_subredes_ipv4/ejercicios_dos_router:ejercicio-16}}
\sphinxAtStartPar
Dada la arquitectura de la red de la figura, asignar direcciones IP, máscaras, puertas de enlace y tablas de rutas de manera que haya conectividad entre todos
los nodos de la red. Se desean utilizar las siguientes redes:
\begin{itemize}
\item {} 
\sphinxAtStartPar
Red 53.160.0.0/11 en el área izquierda.

\item {} 
\sphinxAtStartPar
Red 84.128.0.0/10 en el área central.

\item {} 
\sphinxAtStartPar
Red 179.37.224.0/20 en el área derecha

\end{itemize}

\begin{figure}[htbp]
\centering

\noindent\sphinxincludegraphics{{RedDosRouters}.png}
\end{figure}

\sphinxAtStartPar
Aparte de eso, se desean respetar unos ciertos estándares:
\begin{itemize}
\item {} 
\sphinxAtStartPar
Los routers de acceso a red deben tener siempre la primera IP de la red.

\item {} 
\sphinxAtStartPar
Los routers de distribución (los centrales) deberán tener la primera IP en el punto izquierdo y la última en el derecho.

\end{itemize}


\subsection{Ejercicio 17}
\label{\detokenize{t2_integracion_elementos/ejercicios_subredes_ipv4/ejercicios_dos_router:ejercicio-17}}
\sphinxAtStartPar
Dada la arquitectura de la red de la figura, asignar direcciones IP, máscaras, puertas de enlace y tablas de rutas de manera que haya conectividad entre todos
los nodos de la red. Se desean utilizar las siguientes redes:
\begin{itemize}
\item {} 
\sphinxAtStartPar
Red 29.221.128.0/18 en el área izquierda.

\item {} 
\sphinxAtStartPar
Red 66.181.248.0/21 en el área central.

\item {} 
\sphinxAtStartPar
Red 209.235.210.160/27 en el área derecha

\end{itemize}

\begin{figure}[htbp]
\centering

\noindent\sphinxincludegraphics{{RedDosRouters}.png}
\end{figure}

\sphinxAtStartPar
Aparte de eso, se desean respetar unos ciertos estándares:
\begin{itemize}
\item {} 
\sphinxAtStartPar
Los routers de acceso a red deben tener siempre la primera IP de la red.

\item {} 
\sphinxAtStartPar
Los routers de distribución (los centrales) deberán tener la primera IP en el punto izquierdo y la última en el derecho.

\end{itemize}


\subsection{Ejercicio 18}
\label{\detokenize{t2_integracion_elementos/ejercicios_subredes_ipv4/ejercicios_dos_router:ejercicio-18}}
\sphinxAtStartPar
Dada la arquitectura de la red de la figura, asignar direcciones IP, máscaras, puertas de enlace y tablas de rutas de manera que haya conectividad entre todos
los nodos de la red. Se desean utilizar las siguientes redes:
\begin{itemize}
\item {} 
\sphinxAtStartPar
Red 207.104.97.0/24 en el área izquierda.

\item {} 
\sphinxAtStartPar
Red 163.202.80.0/22 en el área central.

\item {} 
\sphinxAtStartPar
Red 13.160.0.0/11 en el área derecha

\end{itemize}

\begin{figure}[htbp]
\centering

\noindent\sphinxincludegraphics{{RedDosRouters}.png}
\end{figure}

\sphinxAtStartPar
Aparte de eso, se desean respetar unos ciertos estándares:
\begin{itemize}
\item {} 
\sphinxAtStartPar
Los routers de acceso a red deben tener siempre la primera IP de la red.

\item {} 
\sphinxAtStartPar
Los routers de distribución (los centrales) deberán tener la primera IP en el punto izquierdo y la última en el derecho.

\end{itemize}


\subsection{Ejercicio 19}
\label{\detokenize{t2_integracion_elementos/ejercicios_subredes_ipv4/ejercicios_dos_router:ejercicio-19}}
\sphinxAtStartPar
Dada la arquitectura de la red de la figura, asignar direcciones IP, máscaras, puertas de enlace y tablas de rutas de manera que haya conectividad entre todos
los nodos de la red. Se desean utilizar las siguientes redes:
\begin{itemize}
\item {} 
\sphinxAtStartPar
Red 190.232.0.0/16 en el área izquierda.

\item {} 
\sphinxAtStartPar
Red 17.19.204.192/26 en el área central.

\item {} 
\sphinxAtStartPar
Red 102.160.0.0/12 en el área derecha

\end{itemize}

\begin{figure}[htbp]
\centering

\noindent\sphinxincludegraphics{{RedDosRouters}.png}
\end{figure}

\sphinxAtStartPar
Aparte de eso, se desean respetar unos ciertos estándares:
\begin{itemize}
\item {} 
\sphinxAtStartPar
Los routers de acceso a red deben tener siempre la primera IP de la red.

\item {} 
\sphinxAtStartPar
Los routers de distribución (los centrales) deberán tener la primera IP en el punto izquierdo y la última en el derecho.

\end{itemize}


\subsection{Ejercicio 20}
\label{\detokenize{t2_integracion_elementos/ejercicios_subredes_ipv4/ejercicios_dos_router:ejercicio-20}}
\sphinxAtStartPar
Dada la arquitectura de la red de la figura, asignar direcciones IP, máscaras, puertas de enlace y tablas de rutas de manera que haya conectividad entre todos
los nodos de la red. Se desean utilizar las siguientes redes:
\begin{itemize}
\item {} 
\sphinxAtStartPar
Red 213.223.45.0/24 en el área izquierda.

\item {} 
\sphinxAtStartPar
Red 43.234.135.64/26 en el área central.

\item {} 
\sphinxAtStartPar
Red 223.200.34.224/27 en el área derecha

\end{itemize}

\begin{figure}[htbp]
\centering

\noindent\sphinxincludegraphics{{RedDosRouters}.png}
\end{figure}

\sphinxAtStartPar
Aparte de eso, se desean respetar unos ciertos estándares:
\begin{itemize}
\item {} 
\sphinxAtStartPar
Los routers de acceso a red deben tener siempre la primera IP de la red.

\item {} 
\sphinxAtStartPar
Los routers de distribución (los centrales) deberán tener la primera IP en el punto izquierdo y la última en el derecho.

\end{itemize}


\subsection{Ejercicio 21}
\label{\detokenize{t2_integracion_elementos/ejercicios_subredes_ipv4/ejercicios_dos_router:ejercicio-21}}
\sphinxAtStartPar
Dada la arquitectura de la red de la figura, asignar direcciones IP, máscaras, puertas de enlace y tablas de rutas de manera que haya conectividad entre todos
los nodos de la red. Se desean utilizar las siguientes redes:
\begin{itemize}
\item {} 
\sphinxAtStartPar
Red 22.64.0.0/12 en el área izquierda.

\item {} 
\sphinxAtStartPar
Red 157.105.188.16/28 en el área central.

\item {} 
\sphinxAtStartPar
Red 95.64.0.0/11 en el área derecha

\end{itemize}

\begin{figure}[htbp]
\centering

\noindent\sphinxincludegraphics{{RedDosRouters}.png}
\end{figure}

\sphinxAtStartPar
Aparte de eso, se desean respetar unos ciertos estándares:
\begin{itemize}
\item {} 
\sphinxAtStartPar
Los routers de acceso a red deben tener siempre la primera IP de la red.

\item {} 
\sphinxAtStartPar
Los routers de distribución (los centrales) deberán tener la primera IP en el punto izquierdo y la última en el derecho.

\end{itemize}


\subsection{Solución al ejercicio 1 de enrutamiento}
\label{\detokenize{t2_integracion_elementos/ejercicios_subredes_ipv4/ejercicios_dos_router:solucion-al-ejercicio-1-de-enrutamiento}}
\sphinxAtStartPar
Dada la arquitectura de la red de la figura, asignar direcciones IP, máscaras, puertas de enlace y tablas de rutas de manera que haya conectividad entre todos
los nodos de la red. Se desean utilizar las siguientes redes:
\begin{itemize}
\item {} 
\sphinxAtStartPar
Red 63.192.0.0/11 en el área izquierda.

\item {} 
\sphinxAtStartPar
Red 3.244.0.0/16 en el área central.

\item {} 
\sphinxAtStartPar
Red 81.244.63.64/26 en el área derecha

\end{itemize}

\begin{figure}[htbp]
\centering

\noindent\sphinxincludegraphics{{RedDosRouters}.png}
\end{figure}

\sphinxAtStartPar
Aparte de eso, se desean respetar unos ciertos estándares:
\begin{itemize}
\item {} 
\sphinxAtStartPar
Los routers de acceso a red deben tener siempre la primera IP de la red.

\item {} 
\sphinxAtStartPar
Los routers de distribución (los centrales) deberán tener la primera IP en el punto izquierdo y la última en el derecho.

\end{itemize}

\sphinxAtStartPar
Las direcciones serían estas:
\begin{itemize}
\item {} 
\sphinxAtStartPar
PC 1I: IP 63.192.0.1, máscara 255.224.0.0, gateway 63.223.255.254

\item {} 
\sphinxAtStartPar
PC 2I: IP 63.192.0.2, máscara 255.224.0.0, gateway 63.223.255.254

\item {} 
\sphinxAtStartPar
Router I, interfaz izquierda: IP 63.223.255.254, máscara 255.224.0.0

\item {} 
\sphinxAtStartPar
Router I, interfaz derecha: IP 3.244.0.1, máscara 255.255.0.0

\item {} 
\sphinxAtStartPar
PC 1D: IP 81.244.63.65, máscara 255.255.255.192, gateway 81.244.63.126

\item {} 
\sphinxAtStartPar
PC 2D: IP 81.244.63.66, máscara 255.255.255.192, gateway 81.244.63.126

\item {} 
\sphinxAtStartPar
Router D, interfaz derecha: IP 81.244.63.126, máscara 255.255.255.192

\item {} 
\sphinxAtStartPar
Router D, interfaz izquierda: IP 3.244.255.254, máscara 255.255.0.0

\end{itemize}

\sphinxAtStartPar
La tabla de rutas del Router I debería tener la entrada siguiente:
\begin{itemize}
\item {} 
\sphinxAtStartPar
Red 81.244.63.64/26, máscara 255.255.255.192, siguiente salto: 3.244.255.254

\end{itemize}

\sphinxAtStartPar
La tabla de rutas del Router D debería tener la entrada siguiente:
\begin{itemize}
\item {} 
\sphinxAtStartPar
Red 63.192.0.0/11, máscara 255.224.0.0, siguiente salto 3.244.0.1

\end{itemize}


\subsection{Solución al ejercicio 2 de enrutamiento}
\label{\detokenize{t2_integracion_elementos/ejercicios_subredes_ipv4/ejercicios_dos_router:solucion-al-ejercicio-2-de-enrutamiento}}
\sphinxAtStartPar
Dada la arquitectura de la red de la figura, asignar direcciones IP, máscaras, puertas de enlace y tablas de rutas de manera que haya conectividad entre todos
los nodos de la red. Se desean utilizar las siguientes redes:
\begin{itemize}
\item {} 
\sphinxAtStartPar
Red 36.20.206.68/30 en el área izquierda.

\item {} 
\sphinxAtStartPar
Red 11.0.0.0/13 en el área central.

\item {} 
\sphinxAtStartPar
Red 49.32.0.0/11 en el área derecha

\end{itemize}

\begin{figure}[htbp]
\centering

\noindent\sphinxincludegraphics{{RedDosRouters}.png}
\end{figure}

\sphinxAtStartPar
Aparte de eso, se desean respetar unos ciertos estándares:
\begin{itemize}
\item {} 
\sphinxAtStartPar
Los routers de acceso a red deben tener siempre la primera IP de la red.

\item {} 
\sphinxAtStartPar
Los routers de distribución (los centrales) deberán tener la primera IP en el punto izquierdo y la última en el derecho.

\end{itemize}

\sphinxAtStartPar
Las direcciones serían estas:
\begin{itemize}
\item {} 
\sphinxAtStartPar
PC 1I: IP 36.20.206.69, máscara 255.255.255.252, gateway 36.20.206.70

\item {} 
\sphinxAtStartPar
PC 2I: IP 36.20.206.70, máscara 255.255.255.252, gateway 36.20.206.70

\item {} 
\sphinxAtStartPar
Router I, interfaz izquierda: IP 36.20.206.70, máscara 255.255.255.252

\item {} 
\sphinxAtStartPar
Router I, interfaz derecha: IP 11.0.0.1, máscara 255.248.0.0

\item {} 
\sphinxAtStartPar
PC 1D: IP 49.32.0.1, máscara 255.224.0.0, gateway 49.63.255.254

\item {} 
\sphinxAtStartPar
PC 2D: IP 49.32.0.2, máscara 255.224.0.0, gateway 49.63.255.254

\item {} 
\sphinxAtStartPar
Router D, interfaz derecha: IP 49.63.255.254, máscara 255.224.0.0

\item {} 
\sphinxAtStartPar
Router D, interfaz izquierda: IP 11.7.255.254, máscara 255.248.0.0

\end{itemize}

\sphinxAtStartPar
La tabla de rutas del Router I debería tener la entrada siguiente:
\begin{itemize}
\item {} 
\sphinxAtStartPar
Red 49.32.0.0/11, máscara 255.224.0.0, siguiente salto: 11.7.255.254

\end{itemize}

\sphinxAtStartPar
La tabla de rutas del Router D debería tener la entrada siguiente:
\begin{itemize}
\item {} 
\sphinxAtStartPar
Red 36.20.206.68/30, máscara 255.255.255.252, siguiente salto 11.0.0.1

\end{itemize}


\subsection{Solución al ejercicio 3 de enrutamiento}
\label{\detokenize{t2_integracion_elementos/ejercicios_subredes_ipv4/ejercicios_dos_router:solucion-al-ejercicio-3-de-enrutamiento}}
\sphinxAtStartPar
Dada la arquitectura de la red de la figura, asignar direcciones IP, máscaras, puertas de enlace y tablas de rutas de manera que haya conectividad entre todos
los nodos de la red. Se desean utilizar las siguientes redes:
\begin{itemize}
\item {} 
\sphinxAtStartPar
Red 26.30.192.0/19 en el área izquierda.

\item {} 
\sphinxAtStartPar
Red 156.92.185.128/25 en el área central.

\item {} 
\sphinxAtStartPar
Red 212.193.128.0/26 en el área derecha

\end{itemize}

\begin{figure}[htbp]
\centering

\noindent\sphinxincludegraphics{{RedDosRouters}.png}
\end{figure}

\sphinxAtStartPar
Aparte de eso, se desean respetar unos ciertos estándares:
\begin{itemize}
\item {} 
\sphinxAtStartPar
Los routers de acceso a red deben tener siempre la primera IP de la red.

\item {} 
\sphinxAtStartPar
Los routers de distribución (los centrales) deberán tener la primera IP en el punto izquierdo y la última en el derecho.

\end{itemize}

\sphinxAtStartPar
Las direcciones serían estas:
\begin{itemize}
\item {} 
\sphinxAtStartPar
PC 1I: IP 26.30.192.1, máscara 255.255.224.0, gateway 26.30.223.254

\item {} 
\sphinxAtStartPar
PC 2I: IP 26.30.192.2, máscara 255.255.224.0, gateway 26.30.223.254

\item {} 
\sphinxAtStartPar
Router I, interfaz izquierda: IP 26.30.223.254, máscara 255.255.224.0

\item {} 
\sphinxAtStartPar
Router I, interfaz derecha: IP 156.92.185.129, máscara 255.255.255.128

\item {} 
\sphinxAtStartPar
PC 1D: IP 212.193.128.1, máscara 255.255.255.192, gateway 212.193.128.62

\item {} 
\sphinxAtStartPar
PC 2D: IP 212.193.128.2, máscara 255.255.255.192, gateway 212.193.128.62

\item {} 
\sphinxAtStartPar
Router D, interfaz derecha: IP 212.193.128.62, máscara 255.255.255.192

\item {} 
\sphinxAtStartPar
Router D, interfaz izquierda: IP 156.92.185.254, máscara 255.255.255.128

\end{itemize}

\sphinxAtStartPar
La tabla de rutas del Router I debería tener la entrada siguiente:
\begin{itemize}
\item {} 
\sphinxAtStartPar
Red 212.193.128.0/26, máscara 255.255.255.192, siguiente salto: 156.92.185.254

\end{itemize}

\sphinxAtStartPar
La tabla de rutas del Router D debería tener la entrada siguiente:
\begin{itemize}
\item {} 
\sphinxAtStartPar
Red 26.30.192.0/19, máscara 255.255.224.0, siguiente salto 156.92.185.129

\end{itemize}


\subsection{Solución al ejercicio 4 de enrutamiento}
\label{\detokenize{t2_integracion_elementos/ejercicios_subredes_ipv4/ejercicios_dos_router:solucion-al-ejercicio-4-de-enrutamiento}}
\sphinxAtStartPar
Dada la arquitectura de la red de la figura, asignar direcciones IP, máscaras, puertas de enlace y tablas de rutas de manera que haya conectividad entre todos
los nodos de la red. Se desean utilizar las siguientes redes:
\begin{itemize}
\item {} 
\sphinxAtStartPar
Red 205.90.58.120/30 en el área izquierda.

\item {} 
\sphinxAtStartPar
Red 91.210.20.0/23 en el área central.

\item {} 
\sphinxAtStartPar
Red 199.39.201.32/28 en el área derecha

\end{itemize}

\begin{figure}[htbp]
\centering

\noindent\sphinxincludegraphics{{RedDosRouters}.png}
\end{figure}

\sphinxAtStartPar
Aparte de eso, se desean respetar unos ciertos estándares:
\begin{itemize}
\item {} 
\sphinxAtStartPar
Los routers de acceso a red deben tener siempre la primera IP de la red.

\item {} 
\sphinxAtStartPar
Los routers de distribución (los centrales) deberán tener la primera IP en el punto izquierdo y la última en el derecho.

\end{itemize}

\sphinxAtStartPar
Las direcciones serían estas:
\begin{itemize}
\item {} 
\sphinxAtStartPar
PC 1I: IP 205.90.58.121, máscara 255.255.255.252, gateway 205.90.58.122

\item {} 
\sphinxAtStartPar
PC 2I: IP 205.90.58.122, máscara 255.255.255.252, gateway 205.90.58.122

\item {} 
\sphinxAtStartPar
Router I, interfaz izquierda: IP 205.90.58.122, máscara 255.255.255.252

\item {} 
\sphinxAtStartPar
Router I, interfaz derecha: IP 91.210.20.1, máscara 255.255.254.0

\item {} 
\sphinxAtStartPar
PC 1D: IP 199.39.201.33, máscara 255.255.255.240, gateway 199.39.201.46

\item {} 
\sphinxAtStartPar
PC 2D: IP 199.39.201.34, máscara 255.255.255.240, gateway 199.39.201.46

\item {} 
\sphinxAtStartPar
Router D, interfaz derecha: IP 199.39.201.46, máscara 255.255.255.240

\item {} 
\sphinxAtStartPar
Router D, interfaz izquierda: IP 91.210.21.254, máscara 255.255.254.0

\end{itemize}

\sphinxAtStartPar
La tabla de rutas del Router I debería tener la entrada siguiente:
\begin{itemize}
\item {} 
\sphinxAtStartPar
Red 199.39.201.32/28, máscara 255.255.255.240, siguiente salto: 91.210.21.254

\end{itemize}

\sphinxAtStartPar
La tabla de rutas del Router D debería tener la entrada siguiente:
\begin{itemize}
\item {} 
\sphinxAtStartPar
Red 205.90.58.120/30, máscara 255.255.255.252, siguiente salto 91.210.20.1

\end{itemize}


\subsection{Solución al ejercicio 5 de enrutamiento}
\label{\detokenize{t2_integracion_elementos/ejercicios_subredes_ipv4/ejercicios_dos_router:solucion-al-ejercicio-5-de-enrutamiento}}
\sphinxAtStartPar
Dada la arquitectura de la red de la figura, asignar direcciones IP, máscaras, puertas de enlace y tablas de rutas de manera que haya conectividad entre todos
los nodos de la red. Se desean utilizar las siguientes redes:
\begin{itemize}
\item {} 
\sphinxAtStartPar
Red 37.82.32.0/22 en el área izquierda.

\item {} 
\sphinxAtStartPar
Red 191.91.0.0/18 en el área central.

\item {} 
\sphinxAtStartPar
Red 212.11.243.192/26 en el área derecha

\end{itemize}

\begin{figure}[htbp]
\centering

\noindent\sphinxincludegraphics{{RedDosRouters}.png}
\end{figure}

\sphinxAtStartPar
Aparte de eso, se desean respetar unos ciertos estándares:
\begin{itemize}
\item {} 
\sphinxAtStartPar
Los routers de acceso a red deben tener siempre la primera IP de la red.

\item {} 
\sphinxAtStartPar
Los routers de distribución (los centrales) deberán tener la primera IP en el punto izquierdo y la última en el derecho.

\end{itemize}

\sphinxAtStartPar
Las direcciones serían estas:
\begin{itemize}
\item {} 
\sphinxAtStartPar
PC 1I: IP 37.82.32.1, máscara 255.255.252.0, gateway 37.82.35.254

\item {} 
\sphinxAtStartPar
PC 2I: IP 37.82.32.2, máscara 255.255.252.0, gateway 37.82.35.254

\item {} 
\sphinxAtStartPar
Router I, interfaz izquierda: IP 37.82.35.254, máscara 255.255.252.0

\item {} 
\sphinxAtStartPar
Router I, interfaz derecha: IP 191.91.0.1, máscara 255.255.192.0

\item {} 
\sphinxAtStartPar
PC 1D: IP 212.11.243.193, máscara 255.255.255.192, gateway 212.11.243.254

\item {} 
\sphinxAtStartPar
PC 2D: IP 212.11.243.194, máscara 255.255.255.192, gateway 212.11.243.254

\item {} 
\sphinxAtStartPar
Router D, interfaz derecha: IP 212.11.243.254, máscara 255.255.255.192

\item {} 
\sphinxAtStartPar
Router D, interfaz izquierda: IP 191.91.63.254, máscara 255.255.192.0

\end{itemize}

\sphinxAtStartPar
La tabla de rutas del Router I debería tener la entrada siguiente:
\begin{itemize}
\item {} 
\sphinxAtStartPar
Red 212.11.243.192/26, máscara 255.255.255.192, siguiente salto: 191.91.63.254

\end{itemize}

\sphinxAtStartPar
La tabla de rutas del Router D debería tener la entrada siguiente:
\begin{itemize}
\item {} 
\sphinxAtStartPar
Red 37.82.32.0/22, máscara 255.255.252.0, siguiente salto 191.91.0.1

\end{itemize}


\subsection{Solución al ejercicio 6 de enrutamiento}
\label{\detokenize{t2_integracion_elementos/ejercicios_subredes_ipv4/ejercicios_dos_router:solucion-al-ejercicio-6-de-enrutamiento}}
\sphinxAtStartPar
Dada la arquitectura de la red de la figura, asignar direcciones IP, máscaras, puertas de enlace y tablas de rutas de manera que haya conectividad entre todos
los nodos de la red. Se desean utilizar las siguientes redes:
\begin{itemize}
\item {} 
\sphinxAtStartPar
Red 115.141.50.180/30 en el área izquierda.

\item {} 
\sphinxAtStartPar
Red 135.78.64.0/18 en el área central.

\item {} 
\sphinxAtStartPar
Red 130.24.143.0/24 en el área derecha

\end{itemize}

\begin{figure}[htbp]
\centering

\noindent\sphinxincludegraphics{{RedDosRouters}.png}
\end{figure}

\sphinxAtStartPar
Aparte de eso, se desean respetar unos ciertos estándares:
\begin{itemize}
\item {} 
\sphinxAtStartPar
Los routers de acceso a red deben tener siempre la primera IP de la red.

\item {} 
\sphinxAtStartPar
Los routers de distribución (los centrales) deberán tener la primera IP en el punto izquierdo y la última en el derecho.

\end{itemize}

\sphinxAtStartPar
Las direcciones serían estas:
\begin{itemize}
\item {} 
\sphinxAtStartPar
PC 1I: IP 115.141.50.181, máscara 255.255.255.252, gateway 115.141.50.182

\item {} 
\sphinxAtStartPar
PC 2I: IP 115.141.50.182, máscara 255.255.255.252, gateway 115.141.50.182

\item {} 
\sphinxAtStartPar
Router I, interfaz izquierda: IP 115.141.50.182, máscara 255.255.255.252

\item {} 
\sphinxAtStartPar
Router I, interfaz derecha: IP 135.78.64.1, máscara 255.255.192.0

\item {} 
\sphinxAtStartPar
PC 1D: IP 130.24.143.1, máscara 255.255.255.0, gateway 130.24.143.254

\item {} 
\sphinxAtStartPar
PC 2D: IP 130.24.143.2, máscara 255.255.255.0, gateway 130.24.143.254

\item {} 
\sphinxAtStartPar
Router D, interfaz derecha: IP 130.24.143.254, máscara 255.255.255.0

\item {} 
\sphinxAtStartPar
Router D, interfaz izquierda: IP 135.78.127.254, máscara 255.255.192.0

\end{itemize}

\sphinxAtStartPar
La tabla de rutas del Router I debería tener la entrada siguiente:
\begin{itemize}
\item {} 
\sphinxAtStartPar
Red 130.24.143.0/24, máscara 255.255.255.0, siguiente salto: 135.78.127.254

\end{itemize}

\sphinxAtStartPar
La tabla de rutas del Router D debería tener la entrada siguiente:
\begin{itemize}
\item {} 
\sphinxAtStartPar
Red 115.141.50.180/30, máscara 255.255.255.252, siguiente salto 135.78.64.1

\end{itemize}


\subsection{Solución al ejercicio 7 de enrutamiento}
\label{\detokenize{t2_integracion_elementos/ejercicios_subredes_ipv4/ejercicios_dos_router:solucion-al-ejercicio-7-de-enrutamiento}}
\sphinxAtStartPar
Dada la arquitectura de la red de la figura, asignar direcciones IP, máscaras, puertas de enlace y tablas de rutas de manera que haya conectividad entre todos
los nodos de la red. Se desean utilizar las siguientes redes:
\begin{itemize}
\item {} 
\sphinxAtStartPar
Red 131.129.160.0/19 en el área izquierda.

\item {} 
\sphinxAtStartPar
Red 213.63.0.96/27 en el área central.

\item {} 
\sphinxAtStartPar
Red 80.208.0.0/12 en el área derecha

\end{itemize}

\begin{figure}[htbp]
\centering

\noindent\sphinxincludegraphics{{RedDosRouters}.png}
\end{figure}

\sphinxAtStartPar
Aparte de eso, se desean respetar unos ciertos estándares:
\begin{itemize}
\item {} 
\sphinxAtStartPar
Los routers de acceso a red deben tener siempre la primera IP de la red.

\item {} 
\sphinxAtStartPar
Los routers de distribución (los centrales) deberán tener la primera IP en el punto izquierdo y la última en el derecho.

\end{itemize}

\sphinxAtStartPar
Las direcciones serían estas:
\begin{itemize}
\item {} 
\sphinxAtStartPar
PC 1I: IP 131.129.160.1, máscara 255.255.224.0, gateway 131.129.191.254

\item {} 
\sphinxAtStartPar
PC 2I: IP 131.129.160.2, máscara 255.255.224.0, gateway 131.129.191.254

\item {} 
\sphinxAtStartPar
Router I, interfaz izquierda: IP 131.129.191.254, máscara 255.255.224.0

\item {} 
\sphinxAtStartPar
Router I, interfaz derecha: IP 213.63.0.97, máscara 255.255.255.224

\item {} 
\sphinxAtStartPar
PC 1D: IP 80.208.0.1, máscara 255.240.0.0, gateway 80.223.255.254

\item {} 
\sphinxAtStartPar
PC 2D: IP 80.208.0.2, máscara 255.240.0.0, gateway 80.223.255.254

\item {} 
\sphinxAtStartPar
Router D, interfaz derecha: IP 80.223.255.254, máscara 255.240.0.0

\item {} 
\sphinxAtStartPar
Router D, interfaz izquierda: IP 213.63.0.126, máscara 255.255.255.224

\end{itemize}

\sphinxAtStartPar
La tabla de rutas del Router I debería tener la entrada siguiente:
\begin{itemize}
\item {} 
\sphinxAtStartPar
Red 80.208.0.0/12, máscara 255.240.0.0, siguiente salto: 213.63.0.126

\end{itemize}

\sphinxAtStartPar
La tabla de rutas del Router D debería tener la entrada siguiente:
\begin{itemize}
\item {} 
\sphinxAtStartPar
Red 131.129.160.0/19, máscara 255.255.224.0, siguiente salto 213.63.0.97

\end{itemize}


\subsection{Solución al ejercicio 8 de enrutamiento}
\label{\detokenize{t2_integracion_elementos/ejercicios_subredes_ipv4/ejercicios_dos_router:solucion-al-ejercicio-8-de-enrutamiento}}
\sphinxAtStartPar
Dada la arquitectura de la red de la figura, asignar direcciones IP, máscaras, puertas de enlace y tablas de rutas de manera que haya conectividad entre todos
los nodos de la red. Se desean utilizar las siguientes redes:
\begin{itemize}
\item {} 
\sphinxAtStartPar
Red 197.122.17.8/29 en el área izquierda.

\item {} 
\sphinxAtStartPar
Red 84.76.0.0/15 en el área central.

\item {} 
\sphinxAtStartPar
Red 58.144.0.0/13 en el área derecha

\end{itemize}

\begin{figure}[htbp]
\centering

\noindent\sphinxincludegraphics{{RedDosRouters}.png}
\end{figure}

\sphinxAtStartPar
Aparte de eso, se desean respetar unos ciertos estándares:
\begin{itemize}
\item {} 
\sphinxAtStartPar
Los routers de acceso a red deben tener siempre la primera IP de la red.

\item {} 
\sphinxAtStartPar
Los routers de distribución (los centrales) deberán tener la primera IP en el punto izquierdo y la última en el derecho.

\end{itemize}

\sphinxAtStartPar
Las direcciones serían estas:
\begin{itemize}
\item {} 
\sphinxAtStartPar
PC 1I: IP 197.122.17.9, máscara 255.255.255.248, gateway 197.122.17.14

\item {} 
\sphinxAtStartPar
PC 2I: IP 197.122.17.10, máscara 255.255.255.248, gateway 197.122.17.14

\item {} 
\sphinxAtStartPar
Router I, interfaz izquierda: IP 197.122.17.14, máscara 255.255.255.248

\item {} 
\sphinxAtStartPar
Router I, interfaz derecha: IP 84.76.0.1, máscara 255.254.0.0

\item {} 
\sphinxAtStartPar
PC 1D: IP 58.144.0.1, máscara 255.248.0.0, gateway 58.151.255.254

\item {} 
\sphinxAtStartPar
PC 2D: IP 58.144.0.2, máscara 255.248.0.0, gateway 58.151.255.254

\item {} 
\sphinxAtStartPar
Router D, interfaz derecha: IP 58.151.255.254, máscara 255.248.0.0

\item {} 
\sphinxAtStartPar
Router D, interfaz izquierda: IP 84.77.255.254, máscara 255.254.0.0

\end{itemize}

\sphinxAtStartPar
La tabla de rutas del Router I debería tener la entrada siguiente:
\begin{itemize}
\item {} 
\sphinxAtStartPar
Red 58.144.0.0/13, máscara 255.248.0.0, siguiente salto: 84.77.255.254

\end{itemize}

\sphinxAtStartPar
La tabla de rutas del Router D debería tener la entrada siguiente:
\begin{itemize}
\item {} 
\sphinxAtStartPar
Red 197.122.17.8/29, máscara 255.255.255.248, siguiente salto 84.76.0.1

\end{itemize}


\subsection{Solución al ejercicio 9 de enrutamiento}
\label{\detokenize{t2_integracion_elementos/ejercicios_subredes_ipv4/ejercicios_dos_router:solucion-al-ejercicio-9-de-enrutamiento}}
\sphinxAtStartPar
Dada la arquitectura de la red de la figura, asignar direcciones IP, máscaras, puertas de enlace y tablas de rutas de manera que haya conectividad entre todos
los nodos de la red. Se desean utilizar las siguientes redes:
\begin{itemize}
\item {} 
\sphinxAtStartPar
Red 161.126.229.152/29 en el área izquierda.

\item {} 
\sphinxAtStartPar
Red 182.100.101.232/30 en el área central.

\item {} 
\sphinxAtStartPar
Red 59.132.0.0/14 en el área derecha

\end{itemize}

\begin{figure}[htbp]
\centering

\noindent\sphinxincludegraphics{{RedDosRouters}.png}
\end{figure}

\sphinxAtStartPar
Aparte de eso, se desean respetar unos ciertos estándares:
\begin{itemize}
\item {} 
\sphinxAtStartPar
Los routers de acceso a red deben tener siempre la primera IP de la red.

\item {} 
\sphinxAtStartPar
Los routers de distribución (los centrales) deberán tener la primera IP en el punto izquierdo y la última en el derecho.

\end{itemize}

\sphinxAtStartPar
Las direcciones serían estas:
\begin{itemize}
\item {} 
\sphinxAtStartPar
PC 1I: IP 161.126.229.153, máscara 255.255.255.248, gateway 161.126.229.158

\item {} 
\sphinxAtStartPar
PC 2I: IP 161.126.229.154, máscara 255.255.255.248, gateway 161.126.229.158

\item {} 
\sphinxAtStartPar
Router I, interfaz izquierda: IP 161.126.229.158, máscara 255.255.255.248

\item {} 
\sphinxAtStartPar
Router I, interfaz derecha: IP 182.100.101.233, máscara 255.255.255.252

\item {} 
\sphinxAtStartPar
PC 1D: IP 59.132.0.1, máscara 255.252.0.0, gateway 59.135.255.254

\item {} 
\sphinxAtStartPar
PC 2D: IP 59.132.0.2, máscara 255.252.0.0, gateway 59.135.255.254

\item {} 
\sphinxAtStartPar
Router D, interfaz derecha: IP 59.135.255.254, máscara 255.252.0.0

\item {} 
\sphinxAtStartPar
Router D, interfaz izquierda: IP 182.100.101.234, máscara 255.255.255.252

\end{itemize}

\sphinxAtStartPar
La tabla de rutas del Router I debería tener la entrada siguiente:
\begin{itemize}
\item {} 
\sphinxAtStartPar
Red 59.132.0.0/14, máscara 255.252.0.0, siguiente salto: 182.100.101.234

\end{itemize}

\sphinxAtStartPar
La tabla de rutas del Router D debería tener la entrada siguiente:
\begin{itemize}
\item {} 
\sphinxAtStartPar
Red 161.126.229.152/29, máscara 255.255.255.248, siguiente salto 182.100.101.233

\end{itemize}


\subsection{Solución al ejercicio 10 de enrutamiento}
\label{\detokenize{t2_integracion_elementos/ejercicios_subredes_ipv4/ejercicios_dos_router:solucion-al-ejercicio-10-de-enrutamiento}}
\sphinxAtStartPar
Dada la arquitectura de la red de la figura, asignar direcciones IP, máscaras, puertas de enlace y tablas de rutas de manera que haya conectividad entre todos
los nodos de la red. Se desean utilizar las siguientes redes:
\begin{itemize}
\item {} 
\sphinxAtStartPar
Red 104.25.128.0/17 en el área izquierda.

\item {} 
\sphinxAtStartPar
Red 187.143.84.192/27 en el área central.

\item {} 
\sphinxAtStartPar
Red 142.92.128.0/17 en el área derecha

\end{itemize}

\begin{figure}[htbp]
\centering

\noindent\sphinxincludegraphics{{RedDosRouters}.png}
\end{figure}

\sphinxAtStartPar
Aparte de eso, se desean respetar unos ciertos estándares:
\begin{itemize}
\item {} 
\sphinxAtStartPar
Los routers de acceso a red deben tener siempre la primera IP de la red.

\item {} 
\sphinxAtStartPar
Los routers de distribución (los centrales) deberán tener la primera IP en el punto izquierdo y la última en el derecho.

\end{itemize}

\sphinxAtStartPar
Las direcciones serían estas:
\begin{itemize}
\item {} 
\sphinxAtStartPar
PC 1I: IP 104.25.128.1, máscara 255.255.128.0, gateway 104.25.255.254

\item {} 
\sphinxAtStartPar
PC 2I: IP 104.25.128.2, máscara 255.255.128.0, gateway 104.25.255.254

\item {} 
\sphinxAtStartPar
Router I, interfaz izquierda: IP 104.25.255.254, máscara 255.255.128.0

\item {} 
\sphinxAtStartPar
Router I, interfaz derecha: IP 187.143.84.193, máscara 255.255.255.224

\item {} 
\sphinxAtStartPar
PC 1D: IP 142.92.128.1, máscara 255.255.128.0, gateway 142.92.255.254

\item {} 
\sphinxAtStartPar
PC 2D: IP 142.92.128.2, máscara 255.255.128.0, gateway 142.92.255.254

\item {} 
\sphinxAtStartPar
Router D, interfaz derecha: IP 142.92.255.254, máscara 255.255.128.0

\item {} 
\sphinxAtStartPar
Router D, interfaz izquierda: IP 187.143.84.222, máscara 255.255.255.224

\end{itemize}

\sphinxAtStartPar
La tabla de rutas del Router I debería tener la entrada siguiente:
\begin{itemize}
\item {} 
\sphinxAtStartPar
Red 142.92.128.0/17, máscara 255.255.128.0, siguiente salto: 187.143.84.222

\end{itemize}

\sphinxAtStartPar
La tabla de rutas del Router D debería tener la entrada siguiente:
\begin{itemize}
\item {} 
\sphinxAtStartPar
Red 104.25.128.0/17, máscara 255.255.128.0, siguiente salto 187.143.84.193

\end{itemize}


\subsection{Solución al ejercicio 11 de enrutamiento}
\label{\detokenize{t2_integracion_elementos/ejercicios_subredes_ipv4/ejercicios_dos_router:solucion-al-ejercicio-11-de-enrutamiento}}
\sphinxAtStartPar
Dada la arquitectura de la red de la figura, asignar direcciones IP, máscaras, puertas de enlace y tablas de rutas de manera que haya conectividad entre todos
los nodos de la red. Se desean utilizar las siguientes redes:
\begin{itemize}
\item {} 
\sphinxAtStartPar
Red 36.64.0.0/10 en el área izquierda.

\item {} 
\sphinxAtStartPar
Red 177.55.8.0/21 en el área central.

\item {} 
\sphinxAtStartPar
Red 187.48.0.0/16 en el área derecha

\end{itemize}

\begin{figure}[htbp]
\centering

\noindent\sphinxincludegraphics{{RedDosRouters}.png}
\end{figure}

\sphinxAtStartPar
Aparte de eso, se desean respetar unos ciertos estándares:
\begin{itemize}
\item {} 
\sphinxAtStartPar
Los routers de acceso a red deben tener siempre la primera IP de la red.

\item {} 
\sphinxAtStartPar
Los routers de distribución (los centrales) deberán tener la primera IP en el punto izquierdo y la última en el derecho.

\end{itemize}

\sphinxAtStartPar
Las direcciones serían estas:
\begin{itemize}
\item {} 
\sphinxAtStartPar
PC 1I: IP 36.64.0.1, máscara 255.192.0.0, gateway 36.127.255.254

\item {} 
\sphinxAtStartPar
PC 2I: IP 36.64.0.2, máscara 255.192.0.0, gateway 36.127.255.254

\item {} 
\sphinxAtStartPar
Router I, interfaz izquierda: IP 36.127.255.254, máscara 255.192.0.0

\item {} 
\sphinxAtStartPar
Router I, interfaz derecha: IP 177.55.8.1, máscara 255.255.248.0

\item {} 
\sphinxAtStartPar
PC 1D: IP 187.48.0.1, máscara 255.255.0.0, gateway 187.48.255.254

\item {} 
\sphinxAtStartPar
PC 2D: IP 187.48.0.2, máscara 255.255.0.0, gateway 187.48.255.254

\item {} 
\sphinxAtStartPar
Router D, interfaz derecha: IP 187.48.255.254, máscara 255.255.0.0

\item {} 
\sphinxAtStartPar
Router D, interfaz izquierda: IP 177.55.15.254, máscara 255.255.248.0

\end{itemize}

\sphinxAtStartPar
La tabla de rutas del Router I debería tener la entrada siguiente:
\begin{itemize}
\item {} 
\sphinxAtStartPar
Red 187.48.0.0/16, máscara 255.255.0.0, siguiente salto: 177.55.15.254

\end{itemize}

\sphinxAtStartPar
La tabla de rutas del Router D debería tener la entrada siguiente:
\begin{itemize}
\item {} 
\sphinxAtStartPar
Red 36.64.0.0/10, máscara 255.192.0.0, siguiente salto 177.55.8.1

\end{itemize}


\subsection{Solución al ejercicio 12 de enrutamiento}
\label{\detokenize{t2_integracion_elementos/ejercicios_subredes_ipv4/ejercicios_dos_router:solucion-al-ejercicio-12-de-enrutamiento}}
\sphinxAtStartPar
Dada la arquitectura de la red de la figura, asignar direcciones IP, máscaras, puertas de enlace y tablas de rutas de manera que haya conectividad entre todos
los nodos de la red. Se desean utilizar las siguientes redes:
\begin{itemize}
\item {} 
\sphinxAtStartPar
Red 19.166.126.128/26 en el área izquierda.

\item {} 
\sphinxAtStartPar
Red 184.147.243.128/25 en el área central.

\item {} 
\sphinxAtStartPar
Red 109.101.192.0/18 en el área derecha

\end{itemize}

\begin{figure}[htbp]
\centering

\noindent\sphinxincludegraphics{{RedDosRouters}.png}
\end{figure}

\sphinxAtStartPar
Aparte de eso, se desean respetar unos ciertos estándares:
\begin{itemize}
\item {} 
\sphinxAtStartPar
Los routers de acceso a red deben tener siempre la primera IP de la red.

\item {} 
\sphinxAtStartPar
Los routers de distribución (los centrales) deberán tener la primera IP en el punto izquierdo y la última en el derecho.

\end{itemize}

\sphinxAtStartPar
Las direcciones serían estas:
\begin{itemize}
\item {} 
\sphinxAtStartPar
PC 1I: IP 19.166.126.129, máscara 255.255.255.192, gateway 19.166.126.190

\item {} 
\sphinxAtStartPar
PC 2I: IP 19.166.126.130, máscara 255.255.255.192, gateway 19.166.126.190

\item {} 
\sphinxAtStartPar
Router I, interfaz izquierda: IP 19.166.126.190, máscara 255.255.255.192

\item {} 
\sphinxAtStartPar
Router I, interfaz derecha: IP 184.147.243.129, máscara 255.255.255.128

\item {} 
\sphinxAtStartPar
PC 1D: IP 109.101.192.1, máscara 255.255.192.0, gateway 109.101.255.254

\item {} 
\sphinxAtStartPar
PC 2D: IP 109.101.192.2, máscara 255.255.192.0, gateway 109.101.255.254

\item {} 
\sphinxAtStartPar
Router D, interfaz derecha: IP 109.101.255.254, máscara 255.255.192.0

\item {} 
\sphinxAtStartPar
Router D, interfaz izquierda: IP 184.147.243.254, máscara 255.255.255.128

\end{itemize}

\sphinxAtStartPar
La tabla de rutas del Router I debería tener la entrada siguiente:
\begin{itemize}
\item {} 
\sphinxAtStartPar
Red 109.101.192.0/18, máscara 255.255.192.0, siguiente salto: 184.147.243.254

\end{itemize}

\sphinxAtStartPar
La tabla de rutas del Router D debería tener la entrada siguiente:
\begin{itemize}
\item {} 
\sphinxAtStartPar
Red 19.166.126.128/26, máscara 255.255.255.192, siguiente salto 184.147.243.129

\end{itemize}


\subsection{Solución al ejercicio 13 de enrutamiento}
\label{\detokenize{t2_integracion_elementos/ejercicios_subredes_ipv4/ejercicios_dos_router:solucion-al-ejercicio-13-de-enrutamiento}}
\sphinxAtStartPar
Dada la arquitectura de la red de la figura, asignar direcciones IP, máscaras, puertas de enlace y tablas de rutas de manera que haya conectividad entre todos
los nodos de la red. Se desean utilizar las siguientes redes:
\begin{itemize}
\item {} 
\sphinxAtStartPar
Red 19.64.0.0/10 en el área izquierda.

\item {} 
\sphinxAtStartPar
Red 152.188.64.0/18 en el área central.

\item {} 
\sphinxAtStartPar
Red 151.170.0.0/16 en el área derecha

\end{itemize}

\begin{figure}[htbp]
\centering

\noindent\sphinxincludegraphics{{RedDosRouters}.png}
\end{figure}

\sphinxAtStartPar
Aparte de eso, se desean respetar unos ciertos estándares:
\begin{itemize}
\item {} 
\sphinxAtStartPar
Los routers de acceso a red deben tener siempre la primera IP de la red.

\item {} 
\sphinxAtStartPar
Los routers de distribución (los centrales) deberán tener la primera IP en el punto izquierdo y la última en el derecho.

\end{itemize}

\sphinxAtStartPar
Las direcciones serían estas:
\begin{itemize}
\item {} 
\sphinxAtStartPar
PC 1I: IP 19.64.0.1, máscara 255.192.0.0, gateway 19.127.255.254

\item {} 
\sphinxAtStartPar
PC 2I: IP 19.64.0.2, máscara 255.192.0.0, gateway 19.127.255.254

\item {} 
\sphinxAtStartPar
Router I, interfaz izquierda: IP 19.127.255.254, máscara 255.192.0.0

\item {} 
\sphinxAtStartPar
Router I, interfaz derecha: IP 152.188.64.1, máscara 255.255.192.0

\item {} 
\sphinxAtStartPar
PC 1D: IP 151.170.0.1, máscara 255.255.0.0, gateway 151.170.255.254

\item {} 
\sphinxAtStartPar
PC 2D: IP 151.170.0.2, máscara 255.255.0.0, gateway 151.170.255.254

\item {} 
\sphinxAtStartPar
Router D, interfaz derecha: IP 151.170.255.254, máscara 255.255.0.0

\item {} 
\sphinxAtStartPar
Router D, interfaz izquierda: IP 152.188.127.254, máscara 255.255.192.0

\end{itemize}

\sphinxAtStartPar
La tabla de rutas del Router I debería tener la entrada siguiente:
\begin{itemize}
\item {} 
\sphinxAtStartPar
Red 151.170.0.0/16, máscara 255.255.0.0, siguiente salto: 152.188.127.254

\end{itemize}

\sphinxAtStartPar
La tabla de rutas del Router D debería tener la entrada siguiente:
\begin{itemize}
\item {} 
\sphinxAtStartPar
Red 19.64.0.0/10, máscara 255.192.0.0, siguiente salto 152.188.64.1

\end{itemize}


\subsection{Solución al ejercicio 14 de enrutamiento}
\label{\detokenize{t2_integracion_elementos/ejercicios_subredes_ipv4/ejercicios_dos_router:solucion-al-ejercicio-14-de-enrutamiento}}
\sphinxAtStartPar
Dada la arquitectura de la red de la figura, asignar direcciones IP, máscaras, puertas de enlace y tablas de rutas de manera que haya conectividad entre todos
los nodos de la red. Se desean utilizar las siguientes redes:
\begin{itemize}
\item {} 
\sphinxAtStartPar
Red 105.5.128.0/17 en el área izquierda.

\item {} 
\sphinxAtStartPar
Red 148.199.0.0/16 en el área central.

\item {} 
\sphinxAtStartPar
Red 76.163.23.88/29 en el área derecha

\end{itemize}

\begin{figure}[htbp]
\centering

\noindent\sphinxincludegraphics{{RedDosRouters}.png}
\end{figure}

\sphinxAtStartPar
Aparte de eso, se desean respetar unos ciertos estándares:
\begin{itemize}
\item {} 
\sphinxAtStartPar
Los routers de acceso a red deben tener siempre la primera IP de la red.

\item {} 
\sphinxAtStartPar
Los routers de distribución (los centrales) deberán tener la primera IP en el punto izquierdo y la última en el derecho.

\end{itemize}

\sphinxAtStartPar
Las direcciones serían estas:
\begin{itemize}
\item {} 
\sphinxAtStartPar
PC 1I: IP 105.5.128.1, máscara 255.255.128.0, gateway 105.5.255.254

\item {} 
\sphinxAtStartPar
PC 2I: IP 105.5.128.2, máscara 255.255.128.0, gateway 105.5.255.254

\item {} 
\sphinxAtStartPar
Router I, interfaz izquierda: IP 105.5.255.254, máscara 255.255.128.0

\item {} 
\sphinxAtStartPar
Router I, interfaz derecha: IP 148.199.0.1, máscara 255.255.0.0

\item {} 
\sphinxAtStartPar
PC 1D: IP 76.163.23.89, máscara 255.255.255.248, gateway 76.163.23.94

\item {} 
\sphinxAtStartPar
PC 2D: IP 76.163.23.90, máscara 255.255.255.248, gateway 76.163.23.94

\item {} 
\sphinxAtStartPar
Router D, interfaz derecha: IP 76.163.23.94, máscara 255.255.255.248

\item {} 
\sphinxAtStartPar
Router D, interfaz izquierda: IP 148.199.255.254, máscara 255.255.0.0

\end{itemize}

\sphinxAtStartPar
La tabla de rutas del Router I debería tener la entrada siguiente:
\begin{itemize}
\item {} 
\sphinxAtStartPar
Red 76.163.23.88/29, máscara 255.255.255.248, siguiente salto: 148.199.255.254

\end{itemize}

\sphinxAtStartPar
La tabla de rutas del Router D debería tener la entrada siguiente:
\begin{itemize}
\item {} 
\sphinxAtStartPar
Red 105.5.128.0/17, máscara 255.255.128.0, siguiente salto 148.199.0.1

\end{itemize}


\subsection{Solución al ejercicio 15 de enrutamiento}
\label{\detokenize{t2_integracion_elementos/ejercicios_subredes_ipv4/ejercicios_dos_router:solucion-al-ejercicio-15-de-enrutamiento}}
\sphinxAtStartPar
Dada la arquitectura de la red de la figura, asignar direcciones IP, máscaras, puertas de enlace y tablas de rutas de manera que haya conectividad entre todos
los nodos de la red. Se desean utilizar las siguientes redes:
\begin{itemize}
\item {} 
\sphinxAtStartPar
Red 89.128.0.0/11 en el área izquierda.

\item {} 
\sphinxAtStartPar
Red 173.190.20.0/22 en el área central.

\item {} 
\sphinxAtStartPar
Red 133.8.96.0/19 en el área derecha

\end{itemize}

\begin{figure}[htbp]
\centering

\noindent\sphinxincludegraphics{{RedDosRouters}.png}
\end{figure}

\sphinxAtStartPar
Aparte de eso, se desean respetar unos ciertos estándares:
\begin{itemize}
\item {} 
\sphinxAtStartPar
Los routers de acceso a red deben tener siempre la primera IP de la red.

\item {} 
\sphinxAtStartPar
Los routers de distribución (los centrales) deberán tener la primera IP en el punto izquierdo y la última en el derecho.

\end{itemize}

\sphinxAtStartPar
Las direcciones serían estas:
\begin{itemize}
\item {} 
\sphinxAtStartPar
PC 1I: IP 89.128.0.1, máscara 255.224.0.0, gateway 89.159.255.254

\item {} 
\sphinxAtStartPar
PC 2I: IP 89.128.0.2, máscara 255.224.0.0, gateway 89.159.255.254

\item {} 
\sphinxAtStartPar
Router I, interfaz izquierda: IP 89.159.255.254, máscara 255.224.0.0

\item {} 
\sphinxAtStartPar
Router I, interfaz derecha: IP 173.190.20.1, máscara 255.255.252.0

\item {} 
\sphinxAtStartPar
PC 1D: IP 133.8.96.1, máscara 255.255.224.0, gateway 133.8.127.254

\item {} 
\sphinxAtStartPar
PC 2D: IP 133.8.96.2, máscara 255.255.224.0, gateway 133.8.127.254

\item {} 
\sphinxAtStartPar
Router D, interfaz derecha: IP 133.8.127.254, máscara 255.255.224.0

\item {} 
\sphinxAtStartPar
Router D, interfaz izquierda: IP 173.190.23.254, máscara 255.255.252.0

\end{itemize}

\sphinxAtStartPar
La tabla de rutas del Router I debería tener la entrada siguiente:
\begin{itemize}
\item {} 
\sphinxAtStartPar
Red 133.8.96.0/19, máscara 255.255.224.0, siguiente salto: 173.190.23.254

\end{itemize}

\sphinxAtStartPar
La tabla de rutas del Router D debería tener la entrada siguiente:
\begin{itemize}
\item {} 
\sphinxAtStartPar
Red 89.128.0.0/11, máscara 255.224.0.0, siguiente salto 173.190.20.1

\end{itemize}


\subsection{Solución al ejercicio 16 de enrutamiento}
\label{\detokenize{t2_integracion_elementos/ejercicios_subredes_ipv4/ejercicios_dos_router:solucion-al-ejercicio-16-de-enrutamiento}}
\sphinxAtStartPar
Dada la arquitectura de la red de la figura, asignar direcciones IP, máscaras, puertas de enlace y tablas de rutas de manera que haya conectividad entre todos
los nodos de la red. Se desean utilizar las siguientes redes:
\begin{itemize}
\item {} 
\sphinxAtStartPar
Red 53.160.0.0/11 en el área izquierda.

\item {} 
\sphinxAtStartPar
Red 84.128.0.0/10 en el área central.

\item {} 
\sphinxAtStartPar
Red 179.37.224.0/20 en el área derecha

\end{itemize}

\begin{figure}[htbp]
\centering

\noindent\sphinxincludegraphics{{RedDosRouters}.png}
\end{figure}

\sphinxAtStartPar
Aparte de eso, se desean respetar unos ciertos estándares:
\begin{itemize}
\item {} 
\sphinxAtStartPar
Los routers de acceso a red deben tener siempre la primera IP de la red.

\item {} 
\sphinxAtStartPar
Los routers de distribución (los centrales) deberán tener la primera IP en el punto izquierdo y la última en el derecho.

\end{itemize}

\sphinxAtStartPar
Las direcciones serían estas:
\begin{itemize}
\item {} 
\sphinxAtStartPar
PC 1I: IP 53.160.0.1, máscara 255.224.0.0, gateway 53.191.255.254

\item {} 
\sphinxAtStartPar
PC 2I: IP 53.160.0.2, máscara 255.224.0.0, gateway 53.191.255.254

\item {} 
\sphinxAtStartPar
Router I, interfaz izquierda: IP 53.191.255.254, máscara 255.224.0.0

\item {} 
\sphinxAtStartPar
Router I, interfaz derecha: IP 84.128.0.1, máscara 255.192.0.0

\item {} 
\sphinxAtStartPar
PC 1D: IP 179.37.224.1, máscara 255.255.240.0, gateway 179.37.239.254

\item {} 
\sphinxAtStartPar
PC 2D: IP 179.37.224.2, máscara 255.255.240.0, gateway 179.37.239.254

\item {} 
\sphinxAtStartPar
Router D, interfaz derecha: IP 179.37.239.254, máscara 255.255.240.0

\item {} 
\sphinxAtStartPar
Router D, interfaz izquierda: IP 84.191.255.254, máscara 255.192.0.0

\end{itemize}

\sphinxAtStartPar
La tabla de rutas del Router I debería tener la entrada siguiente:
\begin{itemize}
\item {} 
\sphinxAtStartPar
Red 179.37.224.0/20, máscara 255.255.240.0, siguiente salto: 84.191.255.254

\end{itemize}

\sphinxAtStartPar
La tabla de rutas del Router D debería tener la entrada siguiente:
\begin{itemize}
\item {} 
\sphinxAtStartPar
Red 53.160.0.0/11, máscara 255.224.0.0, siguiente salto 84.128.0.1

\end{itemize}


\subsection{Solución al ejercicio 17 de enrutamiento}
\label{\detokenize{t2_integracion_elementos/ejercicios_subredes_ipv4/ejercicios_dos_router:solucion-al-ejercicio-17-de-enrutamiento}}
\sphinxAtStartPar
Dada la arquitectura de la red de la figura, asignar direcciones IP, máscaras, puertas de enlace y tablas de rutas de manera que haya conectividad entre todos
los nodos de la red. Se desean utilizar las siguientes redes:
\begin{itemize}
\item {} 
\sphinxAtStartPar
Red 29.221.128.0/18 en el área izquierda.

\item {} 
\sphinxAtStartPar
Red 66.181.248.0/21 en el área central.

\item {} 
\sphinxAtStartPar
Red 209.235.210.160/27 en el área derecha

\end{itemize}

\begin{figure}[htbp]
\centering

\noindent\sphinxincludegraphics{{RedDosRouters}.png}
\end{figure}

\sphinxAtStartPar
Aparte de eso, se desean respetar unos ciertos estándares:
\begin{itemize}
\item {} 
\sphinxAtStartPar
Los routers de acceso a red deben tener siempre la primera IP de la red.

\item {} 
\sphinxAtStartPar
Los routers de distribución (los centrales) deberán tener la primera IP en el punto izquierdo y la última en el derecho.

\end{itemize}

\sphinxAtStartPar
Las direcciones serían estas:
\begin{itemize}
\item {} 
\sphinxAtStartPar
PC 1I: IP 29.221.128.1, máscara 255.255.192.0, gateway 29.221.191.254

\item {} 
\sphinxAtStartPar
PC 2I: IP 29.221.128.2, máscara 255.255.192.0, gateway 29.221.191.254

\item {} 
\sphinxAtStartPar
Router I, interfaz izquierda: IP 29.221.191.254, máscara 255.255.192.0

\item {} 
\sphinxAtStartPar
Router I, interfaz derecha: IP 66.181.248.1, máscara 255.255.248.0

\item {} 
\sphinxAtStartPar
PC 1D: IP 209.235.210.161, máscara 255.255.255.224, gateway 209.235.210.190

\item {} 
\sphinxAtStartPar
PC 2D: IP 209.235.210.162, máscara 255.255.255.224, gateway 209.235.210.190

\item {} 
\sphinxAtStartPar
Router D, interfaz derecha: IP 209.235.210.190, máscara 255.255.255.224

\item {} 
\sphinxAtStartPar
Router D, interfaz izquierda: IP 66.181.255.254, máscara 255.255.248.0

\end{itemize}

\sphinxAtStartPar
La tabla de rutas del Router I debería tener la entrada siguiente:
\begin{itemize}
\item {} 
\sphinxAtStartPar
Red 209.235.210.160/27, máscara 255.255.255.224, siguiente salto: 66.181.255.254

\end{itemize}

\sphinxAtStartPar
La tabla de rutas del Router D debería tener la entrada siguiente:
\begin{itemize}
\item {} 
\sphinxAtStartPar
Red 29.221.128.0/18, máscara 255.255.192.0, siguiente salto 66.181.248.1

\end{itemize}


\subsection{Solución al ejercicio 18 de enrutamiento}
\label{\detokenize{t2_integracion_elementos/ejercicios_subredes_ipv4/ejercicios_dos_router:solucion-al-ejercicio-18-de-enrutamiento}}
\sphinxAtStartPar
Dada la arquitectura de la red de la figura, asignar direcciones IP, máscaras, puertas de enlace y tablas de rutas de manera que haya conectividad entre todos
los nodos de la red. Se desean utilizar las siguientes redes:
\begin{itemize}
\item {} 
\sphinxAtStartPar
Red 207.104.97.0/24 en el área izquierda.

\item {} 
\sphinxAtStartPar
Red 163.202.80.0/22 en el área central.

\item {} 
\sphinxAtStartPar
Red 13.160.0.0/11 en el área derecha

\end{itemize}

\begin{figure}[htbp]
\centering

\noindent\sphinxincludegraphics{{RedDosRouters}.png}
\end{figure}

\sphinxAtStartPar
Aparte de eso, se desean respetar unos ciertos estándares:
\begin{itemize}
\item {} 
\sphinxAtStartPar
Los routers de acceso a red deben tener siempre la primera IP de la red.

\item {} 
\sphinxAtStartPar
Los routers de distribución (los centrales) deberán tener la primera IP en el punto izquierdo y la última en el derecho.

\end{itemize}

\sphinxAtStartPar
Las direcciones serían estas:
\begin{itemize}
\item {} 
\sphinxAtStartPar
PC 1I: IP 207.104.97.1, máscara 255.255.255.0, gateway 207.104.97.254

\item {} 
\sphinxAtStartPar
PC 2I: IP 207.104.97.2, máscara 255.255.255.0, gateway 207.104.97.254

\item {} 
\sphinxAtStartPar
Router I, interfaz izquierda: IP 207.104.97.254, máscara 255.255.255.0

\item {} 
\sphinxAtStartPar
Router I, interfaz derecha: IP 163.202.80.1, máscara 255.255.252.0

\item {} 
\sphinxAtStartPar
PC 1D: IP 13.160.0.1, máscara 255.224.0.0, gateway 13.191.255.254

\item {} 
\sphinxAtStartPar
PC 2D: IP 13.160.0.2, máscara 255.224.0.0, gateway 13.191.255.254

\item {} 
\sphinxAtStartPar
Router D, interfaz derecha: IP 13.191.255.254, máscara 255.224.0.0

\item {} 
\sphinxAtStartPar
Router D, interfaz izquierda: IP 163.202.83.254, máscara 255.255.252.0

\end{itemize}

\sphinxAtStartPar
La tabla de rutas del Router I debería tener la entrada siguiente:
\begin{itemize}
\item {} 
\sphinxAtStartPar
Red 13.160.0.0/11, máscara 255.224.0.0, siguiente salto: 163.202.83.254

\end{itemize}

\sphinxAtStartPar
La tabla de rutas del Router D debería tener la entrada siguiente:
\begin{itemize}
\item {} 
\sphinxAtStartPar
Red 207.104.97.0/24, máscara 255.255.255.0, siguiente salto 163.202.80.1

\end{itemize}


\subsection{Solución al ejercicio 19 de enrutamiento}
\label{\detokenize{t2_integracion_elementos/ejercicios_subredes_ipv4/ejercicios_dos_router:solucion-al-ejercicio-19-de-enrutamiento}}
\sphinxAtStartPar
Dada la arquitectura de la red de la figura, asignar direcciones IP, máscaras, puertas de enlace y tablas de rutas de manera que haya conectividad entre todos
los nodos de la red. Se desean utilizar las siguientes redes:
\begin{itemize}
\item {} 
\sphinxAtStartPar
Red 190.232.0.0/16 en el área izquierda.

\item {} 
\sphinxAtStartPar
Red 17.19.204.192/26 en el área central.

\item {} 
\sphinxAtStartPar
Red 102.160.0.0/12 en el área derecha

\end{itemize}

\begin{figure}[htbp]
\centering

\noindent\sphinxincludegraphics{{RedDosRouters}.png}
\end{figure}

\sphinxAtStartPar
Aparte de eso, se desean respetar unos ciertos estándares:
\begin{itemize}
\item {} 
\sphinxAtStartPar
Los routers de acceso a red deben tener siempre la primera IP de la red.

\item {} 
\sphinxAtStartPar
Los routers de distribución (los centrales) deberán tener la primera IP en el punto izquierdo y la última en el derecho.

\end{itemize}

\sphinxAtStartPar
Las direcciones serían estas:
\begin{itemize}
\item {} 
\sphinxAtStartPar
PC 1I: IP 190.232.0.1, máscara 255.255.0.0, gateway 190.232.255.254

\item {} 
\sphinxAtStartPar
PC 2I: IP 190.232.0.2, máscara 255.255.0.0, gateway 190.232.255.254

\item {} 
\sphinxAtStartPar
Router I, interfaz izquierda: IP 190.232.255.254, máscara 255.255.0.0

\item {} 
\sphinxAtStartPar
Router I, interfaz derecha: IP 17.19.204.193, máscara 255.255.255.192

\item {} 
\sphinxAtStartPar
PC 1D: IP 102.160.0.1, máscara 255.240.0.0, gateway 102.175.255.254

\item {} 
\sphinxAtStartPar
PC 2D: IP 102.160.0.2, máscara 255.240.0.0, gateway 102.175.255.254

\item {} 
\sphinxAtStartPar
Router D, interfaz derecha: IP 102.175.255.254, máscara 255.240.0.0

\item {} 
\sphinxAtStartPar
Router D, interfaz izquierda: IP 17.19.204.254, máscara 255.255.255.192

\end{itemize}

\sphinxAtStartPar
La tabla de rutas del Router I debería tener la entrada siguiente:
\begin{itemize}
\item {} 
\sphinxAtStartPar
Red 102.160.0.0/12, máscara 255.240.0.0, siguiente salto: 17.19.204.254

\end{itemize}

\sphinxAtStartPar
La tabla de rutas del Router D debería tener la entrada siguiente:
\begin{itemize}
\item {} 
\sphinxAtStartPar
Red 190.232.0.0/16, máscara 255.255.0.0, siguiente salto 17.19.204.193

\end{itemize}


\subsection{Solución al ejercicio 20 de enrutamiento}
\label{\detokenize{t2_integracion_elementos/ejercicios_subredes_ipv4/ejercicios_dos_router:solucion-al-ejercicio-20-de-enrutamiento}}
\sphinxAtStartPar
Dada la arquitectura de la red de la figura, asignar direcciones IP, máscaras, puertas de enlace y tablas de rutas de manera que haya conectividad entre todos
los nodos de la red. Se desean utilizar las siguientes redes:
\begin{itemize}
\item {} 
\sphinxAtStartPar
Red 213.223.45.0/24 en el área izquierda.

\item {} 
\sphinxAtStartPar
Red 43.234.135.64/26 en el área central.

\item {} 
\sphinxAtStartPar
Red 223.200.34.224/27 en el área derecha

\end{itemize}

\begin{figure}[htbp]
\centering

\noindent\sphinxincludegraphics{{RedDosRouters}.png}
\end{figure}

\sphinxAtStartPar
Aparte de eso, se desean respetar unos ciertos estándares:
\begin{itemize}
\item {} 
\sphinxAtStartPar
Los routers de acceso a red deben tener siempre la primera IP de la red.

\item {} 
\sphinxAtStartPar
Los routers de distribución (los centrales) deberán tener la primera IP en el punto izquierdo y la última en el derecho.

\end{itemize}

\sphinxAtStartPar
Las direcciones serían estas:
\begin{itemize}
\item {} 
\sphinxAtStartPar
PC 1I: IP 213.223.45.1, máscara 255.255.255.0, gateway 213.223.45.254

\item {} 
\sphinxAtStartPar
PC 2I: IP 213.223.45.2, máscara 255.255.255.0, gateway 213.223.45.254

\item {} 
\sphinxAtStartPar
Router I, interfaz izquierda: IP 213.223.45.254, máscara 255.255.255.0

\item {} 
\sphinxAtStartPar
Router I, interfaz derecha: IP 43.234.135.65, máscara 255.255.255.192

\item {} 
\sphinxAtStartPar
PC 1D: IP 223.200.34.225, máscara 255.255.255.224, gateway 223.200.34.254

\item {} 
\sphinxAtStartPar
PC 2D: IP 223.200.34.226, máscara 255.255.255.224, gateway 223.200.34.254

\item {} 
\sphinxAtStartPar
Router D, interfaz derecha: IP 223.200.34.254, máscara 255.255.255.224

\item {} 
\sphinxAtStartPar
Router D, interfaz izquierda: IP 43.234.135.126, máscara 255.255.255.192

\end{itemize}

\sphinxAtStartPar
La tabla de rutas del Router I debería tener la entrada siguiente:
\begin{itemize}
\item {} 
\sphinxAtStartPar
Red 223.200.34.224/27, máscara 255.255.255.224, siguiente salto: 43.234.135.126

\end{itemize}

\sphinxAtStartPar
La tabla de rutas del Router D debería tener la entrada siguiente:
\begin{itemize}
\item {} 
\sphinxAtStartPar
Red 213.223.45.0/24, máscara 255.255.255.0, siguiente salto 43.234.135.65

\end{itemize}


\subsection{Solución al ejercicio 21 de enrutamiento}
\label{\detokenize{t2_integracion_elementos/ejercicios_subredes_ipv4/ejercicios_dos_router:solucion-al-ejercicio-21-de-enrutamiento}}
\sphinxAtStartPar
Dada la arquitectura de la red de la figura, asignar direcciones IP, máscaras, puertas de enlace y tablas de rutas de manera que haya conectividad entre todos
los nodos de la red. Se desean utilizar las siguientes redes:
\begin{itemize}
\item {} 
\sphinxAtStartPar
Red 22.64.0.0/12 en el área izquierda.

\item {} 
\sphinxAtStartPar
Red 157.105.188.16/28 en el área central.

\item {} 
\sphinxAtStartPar
Red 95.64.0.0/11 en el área derecha

\end{itemize}

\begin{figure}[htbp]
\centering

\noindent\sphinxincludegraphics{{RedDosRouters}.png}
\end{figure}

\sphinxAtStartPar
Aparte de eso, se desean respetar unos ciertos estándares:
\begin{itemize}
\item {} 
\sphinxAtStartPar
Los routers de acceso a red deben tener siempre la primera IP de la red.

\item {} 
\sphinxAtStartPar
Los routers de distribución (los centrales) deberán tener la primera IP en el punto izquierdo y la última en el derecho.

\end{itemize}

\sphinxAtStartPar
Las direcciones serían estas:
\begin{itemize}
\item {} 
\sphinxAtStartPar
PC 1I: IP 22.64.0.1, máscara 255.240.0.0, gateway 22.79.255.254

\item {} 
\sphinxAtStartPar
PC 2I: IP 22.64.0.2, máscara 255.240.0.0, gateway 22.79.255.254

\item {} 
\sphinxAtStartPar
Router I, interfaz izquierda: IP 22.79.255.254, máscara 255.240.0.0

\item {} 
\sphinxAtStartPar
Router I, interfaz derecha: IP 157.105.188.17, máscara 255.255.255.240

\item {} 
\sphinxAtStartPar
PC 1D: IP 95.64.0.1, máscara 255.224.0.0, gateway 95.95.255.254

\item {} 
\sphinxAtStartPar
PC 2D: IP 95.64.0.2, máscara 255.224.0.0, gateway 95.95.255.254

\item {} 
\sphinxAtStartPar
Router D, interfaz derecha: IP 95.95.255.254, máscara 255.224.0.0

\item {} 
\sphinxAtStartPar
Router D, interfaz izquierda: IP 157.105.188.30, máscara 255.255.255.240

\end{itemize}

\sphinxAtStartPar
La tabla de rutas del Router I debería tener la entrada siguiente:
\begin{itemize}
\item {} 
\sphinxAtStartPar
Red 95.64.0.0/11, máscara 255.224.0.0, siguiente salto: 157.105.188.30

\end{itemize}

\sphinxAtStartPar
La tabla de rutas del Router D debería tener la entrada siguiente:
\begin{itemize}
\item {} 
\sphinxAtStartPar
Red 22.64.0.0/12, máscara 255.240.0.0, siguiente salto 157.105.188.17

\end{itemize}


\chapter{Anexo: Ejercicios sobre compresión de direcciones IPv6}
\label{\detokenize{t2_integracion_elementos/ejerciciosipv6/ejercicios_ipv6:anexo-ejercicios-sobre-compresion-de-direcciones-ipv6}}\label{\detokenize{t2_integracion_elementos/ejerciciosipv6/ejercicios_ipv6::doc}}
\sphinxAtStartPar
Comprimir las direcciones IPv6 siguientes según las reglas de compresión del protocolo (las soluciones aparecen al final):


\begin{savenotes}\sphinxatlongtablestart\begin{longtable}[c]{|l|l|}
\sphinxthelongtablecaptionisattop
\caption{Ejercicios propuestos IPv6\strut}\label{\detokenize{t2_integracion_elementos/ejerciciosipv6/ejercicios_ipv6:id1}}\\*[\sphinxlongtablecapskipadjust]
\hline
\sphinxstyletheadfamily 
\sphinxAtStartPar
Num ejercicio
&\sphinxstyletheadfamily 
\sphinxAtStartPar
IPv6
\\
\hline
\endfirsthead

\multicolumn{2}{c}%
{\makebox[0pt]{\sphinxtablecontinued{\tablename\ \thetable{} \textendash{} proviene de la página anterior}}}\\
\hline
\sphinxstyletheadfamily 
\sphinxAtStartPar
Num ejercicio
&\sphinxstyletheadfamily 
\sphinxAtStartPar
IPv6
\\
\hline
\endhead

\hline
\multicolumn{2}{r}{\makebox[0pt][r]{\sphinxtablecontinued{continué en la próxima página}}}\\
\endfoot

\endlastfoot

\sphinxAtStartPar
1
&
\sphinxAtStartPar
\sphinxcode{\sphinxupquote{e9f9:ba67:0000:f4e8:0000:b344:0000:77ce}}
\\
\hline
\sphinxAtStartPar
2
&
\sphinxAtStartPar
\sphinxcode{\sphinxupquote{1105:9002:08f6:d492:0000:810e:6fe2:26e9}}
\\
\hline
\sphinxAtStartPar
3
&
\sphinxAtStartPar
\sphinxcode{\sphinxupquote{0000:7cec:7cf3:0000:8874:0000:4df7:0000}}
\\
\hline
\sphinxAtStartPar
4
&
\sphinxAtStartPar
\sphinxcode{\sphinxupquote{3539:0000:0000:0000:0000:1001:0000:0000}}
\\
\hline
\sphinxAtStartPar
5
&
\sphinxAtStartPar
\sphinxcode{\sphinxupquote{0000:0000:942c:238f:0000:0000:5457:911e}}
\\
\hline
\sphinxAtStartPar
6
&
\sphinxAtStartPar
\sphinxcode{\sphinxupquote{0000:ec1b:1252:bc77:a392:364b:5d89:938b}}
\\
\hline
\sphinxAtStartPar
7
&
\sphinxAtStartPar
\sphinxcode{\sphinxupquote{84c1:79a9:2635:0000:0000:0000:0000:0000}}
\\
\hline
\sphinxAtStartPar
8
&
\sphinxAtStartPar
\sphinxcode{\sphinxupquote{0000:0000:0000:1f29:0348:0000:af6c:9306}}
\\
\hline
\sphinxAtStartPar
9
&
\sphinxAtStartPar
\sphinxcode{\sphinxupquote{3261:0000:77be:4c86:b322:0000:0000:5c8b}}
\\
\hline
\sphinxAtStartPar
10
&
\sphinxAtStartPar
\sphinxcode{\sphinxupquote{2749:0000:0000:0000:0000:03bb:df01:0000}}
\\
\hline
\sphinxAtStartPar
11
&
\sphinxAtStartPar
\sphinxcode{\sphinxupquote{0000:0000:b753:0000:0000:0000:0000:ec7f}}
\\
\hline
\sphinxAtStartPar
12
&
\sphinxAtStartPar
\sphinxcode{\sphinxupquote{0000:dd97:0000:2c00:0000:8ac8:0000:b783}}
\\
\hline
\sphinxAtStartPar
13
&
\sphinxAtStartPar
\sphinxcode{\sphinxupquote{0000:0000:0000:0000:b4aa:12c0:47a0:0000}}
\\
\hline
\sphinxAtStartPar
14
&
\sphinxAtStartPar
\sphinxcode{\sphinxupquote{f310:0000:0000:0000:0000:b63a:0000:0000}}
\\
\hline
\sphinxAtStartPar
15
&
\sphinxAtStartPar
\sphinxcode{\sphinxupquote{0000:0000:412a:0000:0000:2403:0000:3a00}}
\\
\hline
\sphinxAtStartPar
16
&
\sphinxAtStartPar
\sphinxcode{\sphinxupquote{0000:67fa:bd62:c27c:0000:0000:0000:f1af}}
\\
\hline
\sphinxAtStartPar
17
&
\sphinxAtStartPar
\sphinxcode{\sphinxupquote{0000:0000:5211:9028:0000:b9d0:b78b:0000}}
\\
\hline
\sphinxAtStartPar
18
&
\sphinxAtStartPar
\sphinxcode{\sphinxupquote{0000:3d58:0000:aa0a:7371:0000:0000:c0a6}}
\\
\hline
\sphinxAtStartPar
19
&
\sphinxAtStartPar
\sphinxcode{\sphinxupquote{45a2:e709:0000:0000:7373:746b:0000:dc24}}
\\
\hline
\sphinxAtStartPar
20
&
\sphinxAtStartPar
\sphinxcode{\sphinxupquote{9c47:0000:0000:0000:6413:3ed8:0000:0000}}
\\
\hline
\sphinxAtStartPar
21
&
\sphinxAtStartPar
\sphinxcode{\sphinxupquote{d43e:0000:0000:4de7:0000:754c:d79b:0000}}
\\
\hline
\sphinxAtStartPar
22
&
\sphinxAtStartPar
\sphinxcode{\sphinxupquote{3e9f:0000:0000:0000:0000:db5f:0000:0000}}
\\
\hline
\sphinxAtStartPar
23
&
\sphinxAtStartPar
\sphinxcode{\sphinxupquote{dab7:0000:b129:4837:0000:e8bb:cd1d:235c}}
\\
\hline
\sphinxAtStartPar
24
&
\sphinxAtStartPar
\sphinxcode{\sphinxupquote{ec0c:48b6:0000:0000:0000:0000:0000:0000}}
\\
\hline
\sphinxAtStartPar
25
&
\sphinxAtStartPar
\sphinxcode{\sphinxupquote{3633:8915:39f5:0000:0000:0d82:0000:0000}}
\\
\hline
\sphinxAtStartPar
26
&
\sphinxAtStartPar
\sphinxcode{\sphinxupquote{0000:af51:13a0:0000:fc84:f114:9af0:b988}}
\\
\hline
\sphinxAtStartPar
27
&
\sphinxAtStartPar
\sphinxcode{\sphinxupquote{73b6:55f0:0000:0000:0000:0000:0000:b887}}
\\
\hline
\sphinxAtStartPar
28
&
\sphinxAtStartPar
\sphinxcode{\sphinxupquote{0663:0000:4704:3132:0000:2f36:0000:d0ca}}
\\
\hline
\sphinxAtStartPar
29
&
\sphinxAtStartPar
\sphinxcode{\sphinxupquote{0000:0000:f66d:0000:b973:0000:0f5c:0000}}
\\
\hline
\sphinxAtStartPar
30
&
\sphinxAtStartPar
\sphinxcode{\sphinxupquote{cda2:0000:7f62:07fa:c569:0000:ee8f:740c}}
\\
\hline
\sphinxAtStartPar
31
&
\sphinxAtStartPar
\sphinxcode{\sphinxupquote{e75a:0000:f7cc:a6ca:5b28:0000:8d59:0000}}
\\
\hline
\sphinxAtStartPar
32
&
\sphinxAtStartPar
\sphinxcode{\sphinxupquote{c449:ea16:8e11:7d22:0000:0000:0000:5fa8}}
\\
\hline
\sphinxAtStartPar
33
&
\sphinxAtStartPar
\sphinxcode{\sphinxupquote{0000:0000:9a7a:d7a3:1b61:0000:0000:cd22}}
\\
\hline
\sphinxAtStartPar
34
&
\sphinxAtStartPar
\sphinxcode{\sphinxupquote{0000:0000:8b6c:293f:0000:0000:0000:d90c}}
\\
\hline
\sphinxAtStartPar
35
&
\sphinxAtStartPar
\sphinxcode{\sphinxupquote{f8d4:0000:0000:4fd3:0000:0000:1837:0000}}
\\
\hline
\sphinxAtStartPar
36
&
\sphinxAtStartPar
\sphinxcode{\sphinxupquote{0000:fb07:0000:0000:0000:d783:a576:f695}}
\\
\hline
\sphinxAtStartPar
37
&
\sphinxAtStartPar
\sphinxcode{\sphinxupquote{0000:0000:4cc9:fb0c:0000:0000:0000:3bd9}}
\\
\hline
\sphinxAtStartPar
38
&
\sphinxAtStartPar
\sphinxcode{\sphinxupquote{60fe:0000:0000:7c56:0000:0000:c619:0000}}
\\
\hline
\sphinxAtStartPar
39
&
\sphinxAtStartPar
\sphinxcode{\sphinxupquote{ccb3:0000:c821:0000:0000:0000:0000:c74c}}
\\
\hline
\sphinxAtStartPar
40
&
\sphinxAtStartPar
\sphinxcode{\sphinxupquote{0000:0000:0b95:21ea:0000:0000:0000:0000}}
\\
\hline
\sphinxAtStartPar
41
&
\sphinxAtStartPar
\sphinxcode{\sphinxupquote{0000:0000:c009:0000:4f26:0000:affb:53f3}}
\\
\hline
\sphinxAtStartPar
42
&
\sphinxAtStartPar
\sphinxcode{\sphinxupquote{0000:dcd2:71be:0000:734d:2e61:0000:9881}}
\\
\hline
\sphinxAtStartPar
43
&
\sphinxAtStartPar
\sphinxcode{\sphinxupquote{bae4:eb02:0000:f41a:145d:bb47:0000:0000}}
\\
\hline
\sphinxAtStartPar
44
&
\sphinxAtStartPar
\sphinxcode{\sphinxupquote{0000:0000:0000:0000:0000:740c:0000:1741}}
\\
\hline
\sphinxAtStartPar
45
&
\sphinxAtStartPar
\sphinxcode{\sphinxupquote{0000:0000:0000:0000:7cd8:1e15:c90f:ae1b}}
\\
\hline
\sphinxAtStartPar
46
&
\sphinxAtStartPar
\sphinxcode{\sphinxupquote{241c:8bb2:a902:dc92:1333:4bfb:0000:56c3}}
\\
\hline
\sphinxAtStartPar
47
&
\sphinxAtStartPar
\sphinxcode{\sphinxupquote{ec1f:2794:38cd:0000:e5f8:0000:c2cc:898c}}
\\
\hline
\sphinxAtStartPar
48
&
\sphinxAtStartPar
\sphinxcode{\sphinxupquote{86a3:f3ec:0000:0000:0000:c93a:b47c:0000}}
\\
\hline
\sphinxAtStartPar
49
&
\sphinxAtStartPar
\sphinxcode{\sphinxupquote{0000:c570:e19b:681a:0615:0000:0000:0000}}
\\
\hline
\sphinxAtStartPar
50
&
\sphinxAtStartPar
\sphinxcode{\sphinxupquote{0000:dc4e:0000:0000:0000:0000:0000:8ffd}}
\\
\hline
\sphinxAtStartPar
51
&
\sphinxAtStartPar
\sphinxcode{\sphinxupquote{4ecb:68e9:e08d:a371:0000:0000:0000:0000}}
\\
\hline
\sphinxAtStartPar
52
&
\sphinxAtStartPar
\sphinxcode{\sphinxupquote{0000:0000:0000:64b2:0000:7e75:8bb1:ec30}}
\\
\hline
\sphinxAtStartPar
53
&
\sphinxAtStartPar
\sphinxcode{\sphinxupquote{a7e1:6747:0bb4:0000:0000:0000:5bd7:0000}}
\\
\hline
\sphinxAtStartPar
54
&
\sphinxAtStartPar
\sphinxcode{\sphinxupquote{1803:0000:0000:0000:33e5:4828:0000:e00c}}
\\
\hline
\sphinxAtStartPar
55
&
\sphinxAtStartPar
\sphinxcode{\sphinxupquote{0000:e083:0000:df9f:4d92:0000:0000:477d}}
\\
\hline
\sphinxAtStartPar
56
&
\sphinxAtStartPar
\sphinxcode{\sphinxupquote{2001:0000:0000:f007:781f:0000:c2d5:a767}}
\\
\hline
\sphinxAtStartPar
57
&
\sphinxAtStartPar
\sphinxcode{\sphinxupquote{0000:94e8:0000:5a26:a616:b790:0000:d238}}
\\
\hline
\sphinxAtStartPar
58
&
\sphinxAtStartPar
\sphinxcode{\sphinxupquote{0000:1a4b:fd0e:0000:0000:e929:0000:0000}}
\\
\hline
\sphinxAtStartPar
59
&
\sphinxAtStartPar
\sphinxcode{\sphinxupquote{5c77:da9a:3305:39eb:0000:ade2:0750:7450}}
\\
\hline
\sphinxAtStartPar
60
&
\sphinxAtStartPar
\sphinxcode{\sphinxupquote{7d09:0000:9d50:0000:33dc:0000:0000:445b}}
\\
\hline
\sphinxAtStartPar
61
&
\sphinxAtStartPar
\sphinxcode{\sphinxupquote{4b1a:0000:fd0b:f0c5:86be:0000:551c:0000}}
\\
\hline
\sphinxAtStartPar
62
&
\sphinxAtStartPar
\sphinxcode{\sphinxupquote{589e:0000:53c7:93e3:0000:0000:12e9:093c}}
\\
\hline
\sphinxAtStartPar
63
&
\sphinxAtStartPar
\sphinxcode{\sphinxupquote{0000:3616:0000:8509:368f:6ffa:0000:0000}}
\\
\hline
\sphinxAtStartPar
64
&
\sphinxAtStartPar
\sphinxcode{\sphinxupquote{78cc:20e9:00e5:f0a9:eac3:0000:0000:0000}}
\\
\hline
\sphinxAtStartPar
65
&
\sphinxAtStartPar
\sphinxcode{\sphinxupquote{0000:0000:4d06:0000:0000:0000:b1dd:0000}}
\\
\hline
\sphinxAtStartPar
66
&
\sphinxAtStartPar
\sphinxcode{\sphinxupquote{0000:0000:0000:e4bf:0d24:d134:0b2a:100a}}
\\
\hline
\sphinxAtStartPar
67
&
\sphinxAtStartPar
\sphinxcode{\sphinxupquote{0000:0000:0000:0000:0000:0000:9159:8768}}
\\
\hline
\sphinxAtStartPar
68
&
\sphinxAtStartPar
\sphinxcode{\sphinxupquote{0000:0000:0000:d4b0:bea8:0000:abdb:f2f7}}
\\
\hline
\sphinxAtStartPar
69
&
\sphinxAtStartPar
\sphinxcode{\sphinxupquote{dc62:0000:5b76:0000:01e3:77fb:0000:0000}}
\\
\hline
\sphinxAtStartPar
70
&
\sphinxAtStartPar
\sphinxcode{\sphinxupquote{0000:c781:0000:f950:0000:2451:7b8a:0000}}
\\
\hline
\sphinxAtStartPar
71
&
\sphinxAtStartPar
\sphinxcode{\sphinxupquote{4deb:e14c:c9b0:e65c:2265:0000:0000:29b3}}
\\
\hline
\sphinxAtStartPar
72
&
\sphinxAtStartPar
\sphinxcode{\sphinxupquote{0000:3ebe:5d28:0000:7697:0000:0000:1708}}
\\
\hline
\sphinxAtStartPar
73
&
\sphinxAtStartPar
\sphinxcode{\sphinxupquote{0000:0000:0000:f141:0000:215a:0000:0000}}
\\
\hline
\sphinxAtStartPar
74
&
\sphinxAtStartPar
\sphinxcode{\sphinxupquote{e64a:ff9f:eb7c:923b:3a5f:0000:0000:0000}}
\\
\hline
\sphinxAtStartPar
75
&
\sphinxAtStartPar
\sphinxcode{\sphinxupquote{36a8:58be:9b67:be76:66c3:0000:90a5:0000}}
\\
\hline
\sphinxAtStartPar
76
&
\sphinxAtStartPar
\sphinxcode{\sphinxupquote{b096:0000:0000:0000:7291:0000:eefd:0000}}
\\
\hline
\sphinxAtStartPar
77
&
\sphinxAtStartPar
\sphinxcode{\sphinxupquote{0000:0000:036c:0000:e583:d41a:956c:394f}}
\\
\hline
\sphinxAtStartPar
78
&
\sphinxAtStartPar
\sphinxcode{\sphinxupquote{79e4:8350:0000:6d5b:cac2:0000:dd6b:e62a}}
\\
\hline
\sphinxAtStartPar
79
&
\sphinxAtStartPar
\sphinxcode{\sphinxupquote{34f5:0000:2ecb:0000:0000:27c6:0000:0000}}
\\
\hline
\sphinxAtStartPar
80
&
\sphinxAtStartPar
\sphinxcode{\sphinxupquote{b2d2:b338:0000:0000:3160:20bc:f4e6:5878}}
\\
\hline
\sphinxAtStartPar
81
&
\sphinxAtStartPar
\sphinxcode{\sphinxupquote{0000:0000:0000:4ddc:0000:1646:d72c:0000}}
\\
\hline
\sphinxAtStartPar
82
&
\sphinxAtStartPar
\sphinxcode{\sphinxupquote{0000:e0bb:111c:0000:0000:6fd4:0000:8891}}
\\
\hline
\sphinxAtStartPar
83
&
\sphinxAtStartPar
\sphinxcode{\sphinxupquote{c04b:0000:0000:bac7:0000:e028:0000:c8e5}}
\\
\hline
\sphinxAtStartPar
84
&
\sphinxAtStartPar
\sphinxcode{\sphinxupquote{f680:0000:0000:0000:0000:0000:dfa9:0000}}
\\
\hline
\sphinxAtStartPar
85
&
\sphinxAtStartPar
\sphinxcode{\sphinxupquote{e19b:0000:101a:3fcc:ae97:0000:7970:f214}}
\\
\hline
\sphinxAtStartPar
86
&
\sphinxAtStartPar
\sphinxcode{\sphinxupquote{50c1:9e9c:0000:0000:ed17:0000:8e99:0000}}
\\
\hline
\sphinxAtStartPar
87
&
\sphinxAtStartPar
\sphinxcode{\sphinxupquote{0000:ab50:4066:2809:f314:0000:92da:0000}}
\\
\hline
\sphinxAtStartPar
88
&
\sphinxAtStartPar
\sphinxcode{\sphinxupquote{ff48:0000:0000:22e7:9656:0000:0000:0000}}
\\
\hline
\sphinxAtStartPar
89
&
\sphinxAtStartPar
\sphinxcode{\sphinxupquote{0000:0000:0000:0000:0000:0000:14a5:0000}}
\\
\hline
\sphinxAtStartPar
90
&
\sphinxAtStartPar
\sphinxcode{\sphinxupquote{8d13:0000:237a:c4d7:0000:0000:4df0:c8d0}}
\\
\hline
\sphinxAtStartPar
91
&
\sphinxAtStartPar
\sphinxcode{\sphinxupquote{4a7e:caaa:0000:0000:ec08:ce1f:0000:0000}}
\\
\hline
\sphinxAtStartPar
92
&
\sphinxAtStartPar
\sphinxcode{\sphinxupquote{1c21:0000:0000:0000:0000:0000:e5c0:fc84}}
\\
\hline
\sphinxAtStartPar
93
&
\sphinxAtStartPar
\sphinxcode{\sphinxupquote{0000:0000:32da:419f:0000:5b69:dad0:bc58}}
\\
\hline
\sphinxAtStartPar
94
&
\sphinxAtStartPar
\sphinxcode{\sphinxupquote{e73c:b036:3efd:0000:0000:0d87:0000:6197}}
\\
\hline
\sphinxAtStartPar
95
&
\sphinxAtStartPar
\sphinxcode{\sphinxupquote{0000:0000:0000:5bb0:bf99:0000:a21e:0000}}
\\
\hline
\sphinxAtStartPar
96
&
\sphinxAtStartPar
\sphinxcode{\sphinxupquote{0000:9a47:5197:a901:0000:0000:3ac3:39c8}}
\\
\hline
\sphinxAtStartPar
97
&
\sphinxAtStartPar
\sphinxcode{\sphinxupquote{0000:14e3:0000:0000:06d0:e328:20a4:ea05}}
\\
\hline
\sphinxAtStartPar
98
&
\sphinxAtStartPar
\sphinxcode{\sphinxupquote{d9b4:e5de:7478:a8ac:2a19:3ef6:a970:0000}}
\\
\hline
\sphinxAtStartPar
99
&
\sphinxAtStartPar
\sphinxcode{\sphinxupquote{dcd1:0000:a0df:0000:0000:f58a:0000:f323}}
\\
\hline
\sphinxAtStartPar
100
&
\sphinxAtStartPar
\sphinxcode{\sphinxupquote{0000:021d:64f1:df12:e8ac:0000:489f:75a0}}
\\
\hline
\end{longtable}\sphinxatlongtableend\end{savenotes}


\section{Soluciones a compresión de direcciones IPv6}
\label{\detokenize{t2_integracion_elementos/ejerciciosipv6/ejercicios_ipv6:soluciones-a-compresion-de-direcciones-ipv6}}
\sphinxAtStartPar
A continuación se muestran las soluciones a los ejercicios propuestos:


\begin{savenotes}\sphinxatlongtablestart\begin{longtable}[c]{|l|l|l|}
\sphinxthelongtablecaptionisattop
\caption{Ejercicios resueltos IPv6\strut}\label{\detokenize{t2_integracion_elementos/ejerciciosipv6/ejercicios_ipv6:id2}}\\*[\sphinxlongtablecapskipadjust]
\hline
\sphinxstyletheadfamily 
\sphinxAtStartPar
Num ejercicio
&\sphinxstyletheadfamily 
\sphinxAtStartPar
IPv6
&\sphinxstyletheadfamily 
\sphinxAtStartPar
Comprimida
\\
\hline
\endfirsthead

\multicolumn{3}{c}%
{\makebox[0pt]{\sphinxtablecontinued{\tablename\ \thetable{} \textendash{} proviene de la página anterior}}}\\
\hline
\sphinxstyletheadfamily 
\sphinxAtStartPar
Num ejercicio
&\sphinxstyletheadfamily 
\sphinxAtStartPar
IPv6
&\sphinxstyletheadfamily 
\sphinxAtStartPar
Comprimida
\\
\hline
\endhead

\hline
\multicolumn{3}{r}{\makebox[0pt][r]{\sphinxtablecontinued{continué en la próxima página}}}\\
\endfoot

\endlastfoot

\sphinxAtStartPar
1
&
\sphinxAtStartPar
\sphinxcode{\sphinxupquote{e9f9:ba67:0000:f4e8:0000:b344:0000:77ce}}
&
\sphinxAtStartPar
\sphinxcode{\sphinxupquote{e9f9:ba67:0:f4e8:0:b344:0:77ce}}
\\
\hline
\sphinxAtStartPar
2
&
\sphinxAtStartPar
\sphinxcode{\sphinxupquote{1105:9002:08f6:d492:0000:810e:6fe2:26e9}}
&
\sphinxAtStartPar
\sphinxcode{\sphinxupquote{1105:9002:8f6:d492:0:810e:6fe2:26e9}}
\\
\hline
\sphinxAtStartPar
3
&
\sphinxAtStartPar
\sphinxcode{\sphinxupquote{0000:7cec:7cf3:0000:8874:0000:4df7:0000}}
&
\sphinxAtStartPar
\sphinxcode{\sphinxupquote{0:7cec:7cf3:0:8874:0:4df7:0}}
\\
\hline
\sphinxAtStartPar
4
&
\sphinxAtStartPar
\sphinxcode{\sphinxupquote{3539:0000:0000:0000:0000:1001:0000:0000}}
&
\sphinxAtStartPar
\sphinxcode{\sphinxupquote{3539::1001:0:0}}
\\
\hline
\sphinxAtStartPar
5
&
\sphinxAtStartPar
\sphinxcode{\sphinxupquote{0000:0000:942c:238f:0000:0000:5457:911e}}
&
\sphinxAtStartPar
\sphinxcode{\sphinxupquote{::942c:238f:0:0:5457:911e}}
\\
\hline
\sphinxAtStartPar
6
&
\sphinxAtStartPar
\sphinxcode{\sphinxupquote{0000:ec1b:1252:bc77:a392:364b:5d89:938b}}
&
\sphinxAtStartPar
\sphinxcode{\sphinxupquote{0:ec1b:1252:bc77:a392:364b:5d89:938b}}
\\
\hline
\sphinxAtStartPar
7
&
\sphinxAtStartPar
\sphinxcode{\sphinxupquote{84c1:79a9:2635:0000:0000:0000:0000:0000}}
&
\sphinxAtStartPar
\sphinxcode{\sphinxupquote{84c1:79a9:2635::}}
\\
\hline
\sphinxAtStartPar
8
&
\sphinxAtStartPar
\sphinxcode{\sphinxupquote{0000:0000:0000:1f29:0348:0000:af6c:9306}}
&
\sphinxAtStartPar
\sphinxcode{\sphinxupquote{::1f29:348:0:af6c:9306}}
\\
\hline
\sphinxAtStartPar
9
&
\sphinxAtStartPar
\sphinxcode{\sphinxupquote{3261:0000:77be:4c86:b322:0000:0000:5c8b}}
&
\sphinxAtStartPar
\sphinxcode{\sphinxupquote{3261:0:77be:4c86:b322::5c8b}}
\\
\hline
\sphinxAtStartPar
10
&
\sphinxAtStartPar
\sphinxcode{\sphinxupquote{2749:0000:0000:0000:0000:03bb:df01:0000}}
&
\sphinxAtStartPar
\sphinxcode{\sphinxupquote{2749::3bb:df01:0}}
\\
\hline
\sphinxAtStartPar
11
&
\sphinxAtStartPar
\sphinxcode{\sphinxupquote{0000:0000:b753:0000:0000:0000:0000:ec7f}}
&
\sphinxAtStartPar
\sphinxcode{\sphinxupquote{0:0:b753::ec7f}}
\\
\hline
\sphinxAtStartPar
12
&
\sphinxAtStartPar
\sphinxcode{\sphinxupquote{0000:dd97:0000:2c00:0000:8ac8:0000:b783}}
&
\sphinxAtStartPar
\sphinxcode{\sphinxupquote{0:dd97:0:2c00:0:8ac8:0:b783}}
\\
\hline
\sphinxAtStartPar
13
&
\sphinxAtStartPar
\sphinxcode{\sphinxupquote{0000:0000:0000:0000:b4aa:12c0:47a0:0000}}
&
\sphinxAtStartPar
\sphinxcode{\sphinxupquote{::b4aa:12c0:47a0:0}}
\\
\hline
\sphinxAtStartPar
14
&
\sphinxAtStartPar
\sphinxcode{\sphinxupquote{f310:0000:0000:0000:0000:b63a:0000:0000}}
&
\sphinxAtStartPar
\sphinxcode{\sphinxupquote{f310::b63a:0:0}}
\\
\hline
\sphinxAtStartPar
15
&
\sphinxAtStartPar
\sphinxcode{\sphinxupquote{0000:0000:412a:0000:0000:2403:0000:3a00}}
&
\sphinxAtStartPar
\sphinxcode{\sphinxupquote{::412a:0:0:2403:0:3a00}}
\\
\hline
\sphinxAtStartPar
16
&
\sphinxAtStartPar
\sphinxcode{\sphinxupquote{0000:67fa:bd62:c27c:0000:0000:0000:f1af}}
&
\sphinxAtStartPar
\sphinxcode{\sphinxupquote{0:67fa:bd62:c27c::f1af}}
\\
\hline
\sphinxAtStartPar
17
&
\sphinxAtStartPar
\sphinxcode{\sphinxupquote{0000:0000:5211:9028:0000:b9d0:b78b:0000}}
&
\sphinxAtStartPar
\sphinxcode{\sphinxupquote{::5211:9028:0:b9d0:b78b:0}}
\\
\hline
\sphinxAtStartPar
18
&
\sphinxAtStartPar
\sphinxcode{\sphinxupquote{0000:3d58:0000:aa0a:7371:0000:0000:c0a6}}
&
\sphinxAtStartPar
\sphinxcode{\sphinxupquote{0:3d58:0:aa0a:7371::c0a6}}
\\
\hline
\sphinxAtStartPar
19
&
\sphinxAtStartPar
\sphinxcode{\sphinxupquote{45a2:e709:0000:0000:7373:746b:0000:dc24}}
&
\sphinxAtStartPar
\sphinxcode{\sphinxupquote{45a2:e709::7373:746b:0:dc24}}
\\
\hline
\sphinxAtStartPar
20
&
\sphinxAtStartPar
\sphinxcode{\sphinxupquote{9c47:0000:0000:0000:6413:3ed8:0000:0000}}
&
\sphinxAtStartPar
\sphinxcode{\sphinxupquote{9c47::6413:3ed8:0:0}}
\\
\hline
\sphinxAtStartPar
21
&
\sphinxAtStartPar
\sphinxcode{\sphinxupquote{d43e:0000:0000:4de7:0000:754c:d79b:0000}}
&
\sphinxAtStartPar
\sphinxcode{\sphinxupquote{d43e::4de7:0:754c:d79b:0}}
\\
\hline
\sphinxAtStartPar
22
&
\sphinxAtStartPar
\sphinxcode{\sphinxupquote{3e9f:0000:0000:0000:0000:db5f:0000:0000}}
&
\sphinxAtStartPar
\sphinxcode{\sphinxupquote{3e9f::db5f:0:0}}
\\
\hline
\sphinxAtStartPar
23
&
\sphinxAtStartPar
\sphinxcode{\sphinxupquote{dab7:0000:b129:4837:0000:e8bb:cd1d:235c}}
&
\sphinxAtStartPar
\sphinxcode{\sphinxupquote{dab7:0:b129:4837:0:e8bb:cd1d:235c}}
\\
\hline
\sphinxAtStartPar
24
&
\sphinxAtStartPar
\sphinxcode{\sphinxupquote{ec0c:48b6:0000:0000:0000:0000:0000:0000}}
&
\sphinxAtStartPar
\sphinxcode{\sphinxupquote{ec0c:48b6::}}
\\
\hline
\sphinxAtStartPar
25
&
\sphinxAtStartPar
\sphinxcode{\sphinxupquote{3633:8915:39f5:0000:0000:0d82:0000:0000}}
&
\sphinxAtStartPar
\sphinxcode{\sphinxupquote{3633:8915:39f5::d82:0:0}}
\\
\hline
\sphinxAtStartPar
26
&
\sphinxAtStartPar
\sphinxcode{\sphinxupquote{0000:af51:13a0:0000:fc84:f114:9af0:b988}}
&
\sphinxAtStartPar
\sphinxcode{\sphinxupquote{0:af51:13a0:0:fc84:f114:9af0:b988}}
\\
\hline
\sphinxAtStartPar
27
&
\sphinxAtStartPar
\sphinxcode{\sphinxupquote{73b6:55f0:0000:0000:0000:0000:0000:b887}}
&
\sphinxAtStartPar
\sphinxcode{\sphinxupquote{73b6:55f0::b887}}
\\
\hline
\sphinxAtStartPar
28
&
\sphinxAtStartPar
\sphinxcode{\sphinxupquote{0663:0000:4704:3132:0000:2f36:0000:d0ca}}
&
\sphinxAtStartPar
\sphinxcode{\sphinxupquote{663:0:4704:3132:0:2f36:0:d0ca}}
\\
\hline
\sphinxAtStartPar
29
&
\sphinxAtStartPar
\sphinxcode{\sphinxupquote{0000:0000:f66d:0000:b973:0000:0f5c:0000}}
&
\sphinxAtStartPar
\sphinxcode{\sphinxupquote{::f66d:0:b973:0:f5c:0}}
\\
\hline
\sphinxAtStartPar
30
&
\sphinxAtStartPar
\sphinxcode{\sphinxupquote{cda2:0000:7f62:07fa:c569:0000:ee8f:740c}}
&
\sphinxAtStartPar
\sphinxcode{\sphinxupquote{cda2:0:7f62:7fa:c569:0:ee8f:740c}}
\\
\hline
\sphinxAtStartPar
31
&
\sphinxAtStartPar
\sphinxcode{\sphinxupquote{e75a:0000:f7cc:a6ca:5b28:0000:8d59:0000}}
&
\sphinxAtStartPar
\sphinxcode{\sphinxupquote{e75a:0:f7cc:a6ca:5b28:0:8d59:0}}
\\
\hline
\sphinxAtStartPar
32
&
\sphinxAtStartPar
\sphinxcode{\sphinxupquote{c449:ea16:8e11:7d22:0000:0000:0000:5fa8}}
&
\sphinxAtStartPar
\sphinxcode{\sphinxupquote{c449:ea16:8e11:7d22::5fa8}}
\\
\hline
\sphinxAtStartPar
33
&
\sphinxAtStartPar
\sphinxcode{\sphinxupquote{0000:0000:9a7a:d7a3:1b61:0000:0000:cd22}}
&
\sphinxAtStartPar
\sphinxcode{\sphinxupquote{::9a7a:d7a3:1b61:0:0:cd22}}
\\
\hline
\sphinxAtStartPar
34
&
\sphinxAtStartPar
\sphinxcode{\sphinxupquote{0000:0000:8b6c:293f:0000:0000:0000:d90c}}
&
\sphinxAtStartPar
\sphinxcode{\sphinxupquote{0:0:8b6c:293f::d90c}}
\\
\hline
\sphinxAtStartPar
35
&
\sphinxAtStartPar
\sphinxcode{\sphinxupquote{f8d4:0000:0000:4fd3:0000:0000:1837:0000}}
&
\sphinxAtStartPar
\sphinxcode{\sphinxupquote{f8d4::4fd3:0:0:1837:0}}
\\
\hline
\sphinxAtStartPar
36
&
\sphinxAtStartPar
\sphinxcode{\sphinxupquote{0000:fb07:0000:0000:0000:d783:a576:f695}}
&
\sphinxAtStartPar
\sphinxcode{\sphinxupquote{0:fb07::d783:a576:f695}}
\\
\hline
\sphinxAtStartPar
37
&
\sphinxAtStartPar
\sphinxcode{\sphinxupquote{0000:0000:4cc9:fb0c:0000:0000:0000:3bd9}}
&
\sphinxAtStartPar
\sphinxcode{\sphinxupquote{0:0:4cc9:fb0c::3bd9}}
\\
\hline
\sphinxAtStartPar
38
&
\sphinxAtStartPar
\sphinxcode{\sphinxupquote{60fe:0000:0000:7c56:0000:0000:c619:0000}}
&
\sphinxAtStartPar
\sphinxcode{\sphinxupquote{60fe::7c56:0:0:c619:0}}
\\
\hline
\sphinxAtStartPar
39
&
\sphinxAtStartPar
\sphinxcode{\sphinxupquote{ccb3:0000:c821:0000:0000:0000:0000:c74c}}
&
\sphinxAtStartPar
\sphinxcode{\sphinxupquote{ccb3:0:c821::c74c}}
\\
\hline
\sphinxAtStartPar
40
&
\sphinxAtStartPar
\sphinxcode{\sphinxupquote{0000:0000:0b95:21ea:0000:0000:0000:0000}}
&
\sphinxAtStartPar
\sphinxcode{\sphinxupquote{0:0:b95:21ea::}}
\\
\hline
\sphinxAtStartPar
41
&
\sphinxAtStartPar
\sphinxcode{\sphinxupquote{0000:0000:c009:0000:4f26:0000:affb:53f3}}
&
\sphinxAtStartPar
\sphinxcode{\sphinxupquote{::c009:0:4f26:0:affb:53f3}}
\\
\hline
\sphinxAtStartPar
42
&
\sphinxAtStartPar
\sphinxcode{\sphinxupquote{0000:dcd2:71be:0000:734d:2e61:0000:9881}}
&
\sphinxAtStartPar
\sphinxcode{\sphinxupquote{0:dcd2:71be:0:734d:2e61:0:9881}}
\\
\hline
\sphinxAtStartPar
43
&
\sphinxAtStartPar
\sphinxcode{\sphinxupquote{bae4:eb02:0000:f41a:145d:bb47:0000:0000}}
&
\sphinxAtStartPar
\sphinxcode{\sphinxupquote{bae4:eb02:0:f41a:145d:bb47::}}
\\
\hline
\sphinxAtStartPar
44
&
\sphinxAtStartPar
\sphinxcode{\sphinxupquote{0000:0000:0000:0000:0000:740c:0000:1741}}
&
\sphinxAtStartPar
\sphinxcode{\sphinxupquote{::740c:0:1741}}
\\
\hline
\sphinxAtStartPar
45
&
\sphinxAtStartPar
\sphinxcode{\sphinxupquote{0000:0000:0000:0000:7cd8:1e15:c90f:ae1b}}
&
\sphinxAtStartPar
\sphinxcode{\sphinxupquote{::7cd8:1e15:c90f:ae1b}}
\\
\hline
\sphinxAtStartPar
46
&
\sphinxAtStartPar
\sphinxcode{\sphinxupquote{241c:8bb2:a902:dc92:1333:4bfb:0000:56c3}}
&
\sphinxAtStartPar
\sphinxcode{\sphinxupquote{241c:8bb2:a902:dc92:1333:4bfb:0:56c3}}
\\
\hline
\sphinxAtStartPar
47
&
\sphinxAtStartPar
\sphinxcode{\sphinxupquote{ec1f:2794:38cd:0000:e5f8:0000:c2cc:898c}}
&
\sphinxAtStartPar
\sphinxcode{\sphinxupquote{ec1f:2794:38cd:0:e5f8:0:c2cc:898c}}
\\
\hline
\sphinxAtStartPar
48
&
\sphinxAtStartPar
\sphinxcode{\sphinxupquote{86a3:f3ec:0000:0000:0000:c93a:b47c:0000}}
&
\sphinxAtStartPar
\sphinxcode{\sphinxupquote{86a3:f3ec::c93a:b47c:0}}
\\
\hline
\sphinxAtStartPar
49
&
\sphinxAtStartPar
\sphinxcode{\sphinxupquote{0000:c570:e19b:681a:0615:0000:0000:0000}}
&
\sphinxAtStartPar
\sphinxcode{\sphinxupquote{0:c570:e19b:681a:615::}}
\\
\hline
\sphinxAtStartPar
50
&
\sphinxAtStartPar
\sphinxcode{\sphinxupquote{0000:dc4e:0000:0000:0000:0000:0000:8ffd}}
&
\sphinxAtStartPar
\sphinxcode{\sphinxupquote{0:dc4e::8ffd}}
\\
\hline
\sphinxAtStartPar
51
&
\sphinxAtStartPar
\sphinxcode{\sphinxupquote{4ecb:68e9:e08d:a371:0000:0000:0000:0000}}
&
\sphinxAtStartPar
\sphinxcode{\sphinxupquote{4ecb:68e9:e08d:a371::}}
\\
\hline
\sphinxAtStartPar
52
&
\sphinxAtStartPar
\sphinxcode{\sphinxupquote{0000:0000:0000:64b2:0000:7e75:8bb1:ec30}}
&
\sphinxAtStartPar
\sphinxcode{\sphinxupquote{::64b2:0:7e75:8bb1:ec30}}
\\
\hline
\sphinxAtStartPar
53
&
\sphinxAtStartPar
\sphinxcode{\sphinxupquote{a7e1:6747:0bb4:0000:0000:0000:5bd7:0000}}
&
\sphinxAtStartPar
\sphinxcode{\sphinxupquote{a7e1:6747:bb4::5bd7:0}}
\\
\hline
\sphinxAtStartPar
54
&
\sphinxAtStartPar
\sphinxcode{\sphinxupquote{1803:0000:0000:0000:33e5:4828:0000:e00c}}
&
\sphinxAtStartPar
\sphinxcode{\sphinxupquote{1803::33e5:4828:0:e00c}}
\\
\hline
\sphinxAtStartPar
55
&
\sphinxAtStartPar
\sphinxcode{\sphinxupquote{0000:e083:0000:df9f:4d92:0000:0000:477d}}
&
\sphinxAtStartPar
\sphinxcode{\sphinxupquote{0:e083:0:df9f:4d92::477d}}
\\
\hline
\sphinxAtStartPar
56
&
\sphinxAtStartPar
\sphinxcode{\sphinxupquote{2001:0000:0000:f007:781f:0000:c2d5:a767}}
&
\sphinxAtStartPar
\sphinxcode{\sphinxupquote{2001::f007:781f:0:c2d5:a767}}
\\
\hline
\sphinxAtStartPar
57
&
\sphinxAtStartPar
\sphinxcode{\sphinxupquote{0000:94e8:0000:5a26:a616:b790:0000:d238}}
&
\sphinxAtStartPar
\sphinxcode{\sphinxupquote{0:94e8:0:5a26:a616:b790:0:d238}}
\\
\hline
\sphinxAtStartPar
58
&
\sphinxAtStartPar
\sphinxcode{\sphinxupquote{0000:1a4b:fd0e:0000:0000:e929:0000:0000}}
&
\sphinxAtStartPar
\sphinxcode{\sphinxupquote{0:1a4b:fd0e::e929:0:0}}
\\
\hline
\sphinxAtStartPar
59
&
\sphinxAtStartPar
\sphinxcode{\sphinxupquote{5c77:da9a:3305:39eb:0000:ade2:0750:7450}}
&
\sphinxAtStartPar
\sphinxcode{\sphinxupquote{5c77:da9a:3305:39eb:0:ade2:750:7450}}
\\
\hline
\sphinxAtStartPar
60
&
\sphinxAtStartPar
\sphinxcode{\sphinxupquote{7d09:0000:9d50:0000:33dc:0000:0000:445b}}
&
\sphinxAtStartPar
\sphinxcode{\sphinxupquote{7d09:0:9d50:0:33dc::445b}}
\\
\hline
\sphinxAtStartPar
61
&
\sphinxAtStartPar
\sphinxcode{\sphinxupquote{4b1a:0000:fd0b:f0c5:86be:0000:551c:0000}}
&
\sphinxAtStartPar
\sphinxcode{\sphinxupquote{4b1a:0:fd0b:f0c5:86be:0:551c:0}}
\\
\hline
\sphinxAtStartPar
62
&
\sphinxAtStartPar
\sphinxcode{\sphinxupquote{589e:0000:53c7:93e3:0000:0000:12e9:093c}}
&
\sphinxAtStartPar
\sphinxcode{\sphinxupquote{589e:0:53c7:93e3::12e9:93c}}
\\
\hline
\sphinxAtStartPar
63
&
\sphinxAtStartPar
\sphinxcode{\sphinxupquote{0000:3616:0000:8509:368f:6ffa:0000:0000}}
&
\sphinxAtStartPar
\sphinxcode{\sphinxupquote{0:3616:0:8509:368f:6ffa::}}
\\
\hline
\sphinxAtStartPar
64
&
\sphinxAtStartPar
\sphinxcode{\sphinxupquote{78cc:20e9:00e5:f0a9:eac3:0000:0000:0000}}
&
\sphinxAtStartPar
\sphinxcode{\sphinxupquote{78cc:20e9:e5:f0a9:eac3::}}
\\
\hline
\sphinxAtStartPar
65
&
\sphinxAtStartPar
\sphinxcode{\sphinxupquote{0000:0000:4d06:0000:0000:0000:b1dd:0000}}
&
\sphinxAtStartPar
\sphinxcode{\sphinxupquote{0:0:4d06::b1dd:0}}
\\
\hline
\sphinxAtStartPar
66
&
\sphinxAtStartPar
\sphinxcode{\sphinxupquote{0000:0000:0000:e4bf:0d24:d134:0b2a:100a}}
&
\sphinxAtStartPar
\sphinxcode{\sphinxupquote{::e4bf:d24:d134:b2a:100a}}
\\
\hline
\sphinxAtStartPar
67
&
\sphinxAtStartPar
\sphinxcode{\sphinxupquote{0000:0000:0000:0000:0000:0000:9159:8768}}
&
\sphinxAtStartPar
\sphinxcode{\sphinxupquote{::9159:8768}}
\\
\hline
\sphinxAtStartPar
68
&
\sphinxAtStartPar
\sphinxcode{\sphinxupquote{0000:0000:0000:d4b0:bea8:0000:abdb:f2f7}}
&
\sphinxAtStartPar
\sphinxcode{\sphinxupquote{::d4b0:bea8:0:abdb:f2f7}}
\\
\hline
\sphinxAtStartPar
69
&
\sphinxAtStartPar
\sphinxcode{\sphinxupquote{dc62:0000:5b76:0000:01e3:77fb:0000:0000}}
&
\sphinxAtStartPar
\sphinxcode{\sphinxupquote{dc62:0:5b76:0:1e3:77fb::}}
\\
\hline
\sphinxAtStartPar
70
&
\sphinxAtStartPar
\sphinxcode{\sphinxupquote{0000:c781:0000:f950:0000:2451:7b8a:0000}}
&
\sphinxAtStartPar
\sphinxcode{\sphinxupquote{0:c781:0:f950:0:2451:7b8a:0}}
\\
\hline
\sphinxAtStartPar
71
&
\sphinxAtStartPar
\sphinxcode{\sphinxupquote{4deb:e14c:c9b0:e65c:2265:0000:0000:29b3}}
&
\sphinxAtStartPar
\sphinxcode{\sphinxupquote{4deb:e14c:c9b0:e65c:2265::29b3}}
\\
\hline
\sphinxAtStartPar
72
&
\sphinxAtStartPar
\sphinxcode{\sphinxupquote{0000:3ebe:5d28:0000:7697:0000:0000:1708}}
&
\sphinxAtStartPar
\sphinxcode{\sphinxupquote{0:3ebe:5d28:0:7697::1708}}
\\
\hline
\sphinxAtStartPar
73
&
\sphinxAtStartPar
\sphinxcode{\sphinxupquote{0000:0000:0000:f141:0000:215a:0000:0000}}
&
\sphinxAtStartPar
\sphinxcode{\sphinxupquote{::f141:0:215a:0:0}}
\\
\hline
\sphinxAtStartPar
74
&
\sphinxAtStartPar
\sphinxcode{\sphinxupquote{e64a:ff9f:eb7c:923b:3a5f:0000:0000:0000}}
&
\sphinxAtStartPar
\sphinxcode{\sphinxupquote{e64a:ff9f:eb7c:923b:3a5f::}}
\\
\hline
\sphinxAtStartPar
75
&
\sphinxAtStartPar
\sphinxcode{\sphinxupquote{36a8:58be:9b67:be76:66c3:0000:90a5:0000}}
&
\sphinxAtStartPar
\sphinxcode{\sphinxupquote{36a8:58be:9b67:be76:66c3:0:90a5:0}}
\\
\hline
\sphinxAtStartPar
76
&
\sphinxAtStartPar
\sphinxcode{\sphinxupquote{b096:0000:0000:0000:7291:0000:eefd:0000}}
&
\sphinxAtStartPar
\sphinxcode{\sphinxupquote{b096::7291:0:eefd:0}}
\\
\hline
\sphinxAtStartPar
77
&
\sphinxAtStartPar
\sphinxcode{\sphinxupquote{0000:0000:036c:0000:e583:d41a:956c:394f}}
&
\sphinxAtStartPar
\sphinxcode{\sphinxupquote{::36c:0:e583:d41a:956c:394f}}
\\
\hline
\sphinxAtStartPar
78
&
\sphinxAtStartPar
\sphinxcode{\sphinxupquote{79e4:8350:0000:6d5b:cac2:0000:dd6b:e62a}}
&
\sphinxAtStartPar
\sphinxcode{\sphinxupquote{79e4:8350:0:6d5b:cac2:0:dd6b:e62a}}
\\
\hline
\sphinxAtStartPar
79
&
\sphinxAtStartPar
\sphinxcode{\sphinxupquote{34f5:0000:2ecb:0000:0000:27c6:0000:0000}}
&
\sphinxAtStartPar
\sphinxcode{\sphinxupquote{34f5:0:2ecb::27c6:0:0}}
\\
\hline
\sphinxAtStartPar
80
&
\sphinxAtStartPar
\sphinxcode{\sphinxupquote{b2d2:b338:0000:0000:3160:20bc:f4e6:5878}}
&
\sphinxAtStartPar
\sphinxcode{\sphinxupquote{b2d2:b338::3160:20bc:f4e6:5878}}
\\
\hline
\sphinxAtStartPar
81
&
\sphinxAtStartPar
\sphinxcode{\sphinxupquote{0000:0000:0000:4ddc:0000:1646:d72c:0000}}
&
\sphinxAtStartPar
\sphinxcode{\sphinxupquote{::4ddc:0:1646:d72c:0}}
\\
\hline
\sphinxAtStartPar
82
&
\sphinxAtStartPar
\sphinxcode{\sphinxupquote{0000:e0bb:111c:0000:0000:6fd4:0000:8891}}
&
\sphinxAtStartPar
\sphinxcode{\sphinxupquote{0:e0bb:111c::6fd4:0:8891}}
\\
\hline
\sphinxAtStartPar
83
&
\sphinxAtStartPar
\sphinxcode{\sphinxupquote{c04b:0000:0000:bac7:0000:e028:0000:c8e5}}
&
\sphinxAtStartPar
\sphinxcode{\sphinxupquote{c04b::bac7:0:e028:0:c8e5}}
\\
\hline
\sphinxAtStartPar
84
&
\sphinxAtStartPar
\sphinxcode{\sphinxupquote{f680:0000:0000:0000:0000:0000:dfa9:0000}}
&
\sphinxAtStartPar
\sphinxcode{\sphinxupquote{f680::dfa9:0}}
\\
\hline
\sphinxAtStartPar
85
&
\sphinxAtStartPar
\sphinxcode{\sphinxupquote{e19b:0000:101a:3fcc:ae97:0000:7970:f214}}
&
\sphinxAtStartPar
\sphinxcode{\sphinxupquote{e19b:0:101a:3fcc:ae97:0:7970:f214}}
\\
\hline
\sphinxAtStartPar
86
&
\sphinxAtStartPar
\sphinxcode{\sphinxupquote{50c1:9e9c:0000:0000:ed17:0000:8e99:0000}}
&
\sphinxAtStartPar
\sphinxcode{\sphinxupquote{50c1:9e9c::ed17:0:8e99:0}}
\\
\hline
\sphinxAtStartPar
87
&
\sphinxAtStartPar
\sphinxcode{\sphinxupquote{0000:ab50:4066:2809:f314:0000:92da:0000}}
&
\sphinxAtStartPar
\sphinxcode{\sphinxupquote{0:ab50:4066:2809:f314:0:92da:0}}
\\
\hline
\sphinxAtStartPar
88
&
\sphinxAtStartPar
\sphinxcode{\sphinxupquote{ff48:0000:0000:22e7:9656:0000:0000:0000}}
&
\sphinxAtStartPar
\sphinxcode{\sphinxupquote{ff48:0:0:22e7:9656::}}
\\
\hline
\sphinxAtStartPar
89
&
\sphinxAtStartPar
\sphinxcode{\sphinxupquote{0000:0000:0000:0000:0000:0000:14a5:0000}}
&
\sphinxAtStartPar
\sphinxcode{\sphinxupquote{::14a5:0}}
\\
\hline
\sphinxAtStartPar
90
&
\sphinxAtStartPar
\sphinxcode{\sphinxupquote{8d13:0000:237a:c4d7:0000:0000:4df0:c8d0}}
&
\sphinxAtStartPar
\sphinxcode{\sphinxupquote{8d13:0:237a:c4d7::4df0:c8d0}}
\\
\hline
\sphinxAtStartPar
91
&
\sphinxAtStartPar
\sphinxcode{\sphinxupquote{4a7e:caaa:0000:0000:ec08:ce1f:0000:0000}}
&
\sphinxAtStartPar
\sphinxcode{\sphinxupquote{4a7e:caaa::ec08:ce1f:0:0}}
\\
\hline
\sphinxAtStartPar
92
&
\sphinxAtStartPar
\sphinxcode{\sphinxupquote{1c21:0000:0000:0000:0000:0000:e5c0:fc84}}
&
\sphinxAtStartPar
\sphinxcode{\sphinxupquote{1c21::e5c0:fc84}}
\\
\hline
\sphinxAtStartPar
93
&
\sphinxAtStartPar
\sphinxcode{\sphinxupquote{0000:0000:32da:419f:0000:5b69:dad0:bc58}}
&
\sphinxAtStartPar
\sphinxcode{\sphinxupquote{::32da:419f:0:5b69:dad0:bc58}}
\\
\hline
\sphinxAtStartPar
94
&
\sphinxAtStartPar
\sphinxcode{\sphinxupquote{e73c:b036:3efd:0000:0000:0d87:0000:6197}}
&
\sphinxAtStartPar
\sphinxcode{\sphinxupquote{e73c:b036:3efd::d87:0:6197}}
\\
\hline
\sphinxAtStartPar
95
&
\sphinxAtStartPar
\sphinxcode{\sphinxupquote{0000:0000:0000:5bb0:bf99:0000:a21e:0000}}
&
\sphinxAtStartPar
\sphinxcode{\sphinxupquote{::5bb0:bf99:0:a21e:0}}
\\
\hline
\sphinxAtStartPar
96
&
\sphinxAtStartPar
\sphinxcode{\sphinxupquote{0000:9a47:5197:a901:0000:0000:3ac3:39c8}}
&
\sphinxAtStartPar
\sphinxcode{\sphinxupquote{0:9a47:5197:a901::3ac3:39c8}}
\\
\hline
\sphinxAtStartPar
97
&
\sphinxAtStartPar
\sphinxcode{\sphinxupquote{0000:14e3:0000:0000:06d0:e328:20a4:ea05}}
&
\sphinxAtStartPar
\sphinxcode{\sphinxupquote{0:14e3::6d0:e328:20a4:ea05}}
\\
\hline
\sphinxAtStartPar
98
&
\sphinxAtStartPar
\sphinxcode{\sphinxupquote{d9b4:e5de:7478:a8ac:2a19:3ef6:a970:0000}}
&
\sphinxAtStartPar
\sphinxcode{\sphinxupquote{d9b4:e5de:7478:a8ac:2a19:3ef6:a970:0}}
\\
\hline
\sphinxAtStartPar
99
&
\sphinxAtStartPar
\sphinxcode{\sphinxupquote{dcd1:0000:a0df:0000:0000:f58a:0000:f323}}
&
\sphinxAtStartPar
\sphinxcode{\sphinxupquote{dcd1:0:a0df::f58a:0:f323}}
\\
\hline
\sphinxAtStartPar
100
&
\sphinxAtStartPar
\sphinxcode{\sphinxupquote{0000:021d:64f1:df12:e8ac:0000:489f:75a0}}
&
\sphinxAtStartPar
\sphinxcode{\sphinxupquote{0:21d:64f1:df12:e8ac:0:489f:75a0}}
\\
\hline
\end{longtable}\sphinxatlongtableend\end{savenotes}


\chapter{Anexo: ejercicios sobre clasificación de direcciones IPv6}
\label{\detokenize{t2_integracion_elementos/ejercicios_ips/clasificacion_ipv6:anexo-ejercicios-sobre-clasificacion-de-direcciones-ipv6}}\label{\detokenize{t2_integracion_elementos/ejercicios_ips/clasificacion_ipv6::doc}}
\sphinxAtStartPar
Dadas las siguientes direcciones IPv6 indica de qué tipo son:
\begin{enumerate}
\sphinxsetlistlabels{\arabic}{enumi}{enumii}{}{.}%
\item {} 
\sphinxAtStartPar
\sphinxcode{\sphinxupquote{feab:e7b8:6626:0:a16e:0:6efe:995f}}

\item {} 
\sphinxAtStartPar
\sphinxcode{\sphinxupquote{fdf6:22e5:e01:0:9af6::}}

\item {} 
\sphinxAtStartPar
\sphinxcode{\sphinxupquote{fd53:a56e:0:950d:4bc::3c92}}

\item {} 
\sphinxAtStartPar
\sphinxcode{\sphinxupquote{ff22:a6f1::933c:ff7b:f150}}

\item {} 
\sphinxAtStartPar
\sphinxcode{\sphinxupquote{fea3:0:d97a::eac:bbb1:c90b}}

\item {} 
\sphinxAtStartPar
\sphinxcode{\sphinxupquote{fd67:0:3456:e82b:4617:138d:936:3834}}

\item {} 
\sphinxAtStartPar
\sphinxcode{\sphinxupquote{fdfb:1935::25cf:4987:0:0}}

\item {} 
\sphinxAtStartPar
\sphinxcode{\sphinxupquote{2277:2700:2b57:0:4d38::}}

\item {} 
\sphinxAtStartPar
\sphinxcode{\sphinxupquote{ffad:bdd2:3e17:0:98f7:7f6b::}}

\item {} 
\sphinxAtStartPar
\sphinxcode{\sphinxupquote{3415::}}

\item {} 
\sphinxAtStartPar
\sphinxcode{\sphinxupquote{ff6d:0:6a25:df12:88d:5fa6::}}

\item {} 
\sphinxAtStartPar
\sphinxcode{\sphinxupquote{ff22::71ac:0:aca}}

\item {} 
\sphinxAtStartPar
\sphinxcode{\sphinxupquote{fc0a:0:cbf3:70ae:e72c:0:c4db:0}}

\item {} 
\sphinxAtStartPar
\sphinxcode{\sphinxupquote{26bc:0:49e::5f84:d8c5:0}}

\item {} 
\sphinxAtStartPar
\sphinxcode{\sphinxupquote{fe90:af2c:c2d6:195e:bc86:0:4f08:f7ca}}

\item {} 
\sphinxAtStartPar
\sphinxcode{\sphinxupquote{2537:5b85:8e74:b4d0:b2fc::}}

\item {} 
\sphinxAtStartPar
\sphinxcode{\sphinxupquote{fe96::b88b:32f8:0:d025:0}}

\item {} 
\sphinxAtStartPar
\sphinxcode{\sphinxupquote{3170::98dc:d2c0:c886:0:0}}

\item {} 
\sphinxAtStartPar
\sphinxcode{\sphinxupquote{29ae:0:243:0:d814:1b69:f171:0}}

\item {} 
\sphinxAtStartPar
\sphinxcode{\sphinxupquote{274d:d1e0:19f2:0:fa26:fb94:529e:1378}}

\item {} 
\sphinxAtStartPar
\sphinxcode{\sphinxupquote{3c9e:6c30::71e3:0:0:8f68}}

\item {} 
\sphinxAtStartPar
\sphinxcode{\sphinxupquote{fd9b::a80f}}

\item {} 
\sphinxAtStartPar
\sphinxcode{\sphinxupquote{feb6:0:6c6d:c6dd:454c:4ea:d71e:cee}}

\item {} 
\sphinxAtStartPar
\sphinxcode{\sphinxupquote{fea1:e171:60e7:d8e7:4d43::df9}}

\item {} 
\sphinxAtStartPar
\sphinxcode{\sphinxupquote{fc98::67d0:0:68a5:0:0}}

\item {} 
\sphinxAtStartPar
\sphinxcode{\sphinxupquote{fea6:791a:c569:86a1:bf71:adcb:11:fe15}}

\item {} 
\sphinxAtStartPar
\sphinxcode{\sphinxupquote{fdfd:9440:8179:7eef:0:a606:0:1fe7}}

\item {} 
\sphinxAtStartPar
\sphinxcode{\sphinxupquote{3b2f:90fa:0:bb9e:13b:ab79::}}

\item {} 
\sphinxAtStartPar
\sphinxcode{\sphinxupquote{fea9:1415:0:e7bc:0:fbcd::}}

\item {} 
\sphinxAtStartPar
\sphinxcode{\sphinxupquote{ff5e:0:336f:0:7f53::}}

\item {} 
\sphinxAtStartPar
\sphinxcode{\sphinxupquote{fe83:dd6c:0:9584:3367:b654:75b3:0}}

\item {} 
\sphinxAtStartPar
\sphinxcode{\sphinxupquote{fc58:0:bc51:10a3::}}

\item {} 
\sphinxAtStartPar
\sphinxcode{\sphinxupquote{ff3a:e8a3::3c07:5a1e:0}}

\item {} 
\sphinxAtStartPar
\sphinxcode{\sphinxupquote{feae::2230:0:e8b}}

\item {} 
\sphinxAtStartPar
\sphinxcode{\sphinxupquote{ffcd:0:6ecc:9718:0:c20e::}}

\item {} 
\sphinxAtStartPar
\sphinxcode{\sphinxupquote{ff9e:0:246f:0:ed59::7fb4}}

\item {} 
\sphinxAtStartPar
\sphinxcode{\sphinxupquote{ff68:0:e756:95e1::67f1:0}}

\item {} 
\sphinxAtStartPar
\sphinxcode{\sphinxupquote{fc5e:d122:7da3:896e:626c::}}

\item {} 
\sphinxAtStartPar
\sphinxcode{\sphinxupquote{3bc1:0:ab12:85bf:2274::}}

\item {} 
\sphinxAtStartPar
\sphinxcode{\sphinxupquote{ff09:7ffd:d25c:2e26:aeac:3a45:0:f1e}}

\item {} 
\sphinxAtStartPar
\sphinxcode{\sphinxupquote{fd03:0:5db4:d114:2972:0:c484:0}}

\item {} 
\sphinxAtStartPar
\sphinxcode{\sphinxupquote{feac:cfe9::e3c9:0}}

\item {} 
\sphinxAtStartPar
\sphinxcode{\sphinxupquote{ff05::9c60:0:0:4b4a}}

\item {} 
\sphinxAtStartPar
\sphinxcode{\sphinxupquote{2499:0:aeac:0:12e:cbdf:7b95:0}}

\item {} 
\sphinxAtStartPar
\sphinxcode{\sphinxupquote{ff9f::5cc0}}

\item {} 
\sphinxAtStartPar
\sphinxcode{\sphinxupquote{fdc5:289e:0:c613::7add:6ea5}}

\item {} 
\sphinxAtStartPar
\sphinxcode{\sphinxupquote{ff79:8c::ece7:0:0}}

\item {} 
\sphinxAtStartPar
\sphinxcode{\sphinxupquote{fd16:8393:2506::9bb4:75be:0}}

\item {} 
\sphinxAtStartPar
\sphinxcode{\sphinxupquote{ff9c:875b:3f71:0:299e::}}

\item {} 
\sphinxAtStartPar
\sphinxcode{\sphinxupquote{fddd::1d58:0:bfa5:f060:0}}

\end{enumerate}


\section{Soluciones a la clasificaciones de direcciones IPv6}
\label{\detokenize{t2_integracion_elementos/ejercicios_ips/clasificacion_ipv6:soluciones-a-la-clasificaciones-de-direcciones-ipv6}}\begin{enumerate}
\sphinxsetlistlabels{\arabic}{enumi}{enumii}{}{.}%
\item {} 
\sphinxAtStartPar
\sphinxcode{\sphinxupquote{feab:e7b8:6626:0:a16e:0:6efe:995f}} es de tipo unicast local en enlace

\item {} 
\sphinxAtStartPar
\sphinxcode{\sphinxupquote{fdf6:22e5:e01:0:9af6::}} es de tipo unicast local único

\item {} 
\sphinxAtStartPar
\sphinxcode{\sphinxupquote{fd53:a56e:0:950d:4bc::3c92}} es de tipo unicast local único

\item {} 
\sphinxAtStartPar
\sphinxcode{\sphinxupquote{ff22:a6f1::933c:ff7b:f150}} es de tipo multicast

\item {} 
\sphinxAtStartPar
\sphinxcode{\sphinxupquote{fea3:0:d97a::eac:bbb1:c90b}} es de tipo unicast local en enlace

\item {} 
\sphinxAtStartPar
\sphinxcode{\sphinxupquote{fd67:0:3456:e82b:4617:138d:936:3834}} es de tipo unicast local único

\item {} 
\sphinxAtStartPar
\sphinxcode{\sphinxupquote{fdfb:1935::25cf:4987:0:0}} es de tipo unicast local único

\item {} 
\sphinxAtStartPar
\sphinxcode{\sphinxupquote{2277:2700:2b57:0:4d38::}} es de tipo unicast global

\item {} 
\sphinxAtStartPar
\sphinxcode{\sphinxupquote{ffad:bdd2:3e17:0:98f7:7f6b::}} es de tipo multicast

\item {} 
\sphinxAtStartPar
\sphinxcode{\sphinxupquote{3415::}} es de tipo unicast global

\item {} 
\sphinxAtStartPar
\sphinxcode{\sphinxupquote{ff6d:0:6a25:df12:88d:5fa6::}} es de tipo multicast

\item {} 
\sphinxAtStartPar
\sphinxcode{\sphinxupquote{ff22::71ac:0:aca}} es de tipo multicast

\item {} 
\sphinxAtStartPar
\sphinxcode{\sphinxupquote{fc0a:0:cbf3:70ae:e72c:0:c4db:0}} es de tipo unicast local único

\item {} 
\sphinxAtStartPar
\sphinxcode{\sphinxupquote{26bc:0:49e::5f84:d8c5:0}} es de tipo unicast global

\item {} 
\sphinxAtStartPar
\sphinxcode{\sphinxupquote{fe90:af2c:c2d6:195e:bc86:0:4f08:f7ca}} es de tipo unicast local en enlace

\item {} 
\sphinxAtStartPar
\sphinxcode{\sphinxupquote{2537:5b85:8e74:b4d0:b2fc::}} es de tipo unicast global

\item {} 
\sphinxAtStartPar
\sphinxcode{\sphinxupquote{fe96::b88b:32f8:0:d025:0}} es de tipo unicast local en enlace

\item {} 
\sphinxAtStartPar
\sphinxcode{\sphinxupquote{3170::98dc:d2c0:c886:0:0}} es de tipo unicast global

\item {} 
\sphinxAtStartPar
\sphinxcode{\sphinxupquote{29ae:0:243:0:d814:1b69:f171:0}} es de tipo unicast global

\item {} 
\sphinxAtStartPar
\sphinxcode{\sphinxupquote{274d:d1e0:19f2:0:fa26:fb94:529e:1378}} es de tipo unicast global

\item {} 
\sphinxAtStartPar
\sphinxcode{\sphinxupquote{3c9e:6c30::71e3:0:0:8f68}} es de tipo unicast global

\item {} 
\sphinxAtStartPar
\sphinxcode{\sphinxupquote{fd9b::a80f}} es de tipo unicast local único

\item {} 
\sphinxAtStartPar
\sphinxcode{\sphinxupquote{feb6:0:6c6d:c6dd:454c:4ea:d71e:cee}} es de tipo unicast local en enlace

\item {} 
\sphinxAtStartPar
\sphinxcode{\sphinxupquote{fea1:e171:60e7:d8e7:4d43::df9}} es de tipo unicast local en enlace

\item {} 
\sphinxAtStartPar
\sphinxcode{\sphinxupquote{fc98::67d0:0:68a5:0:0}} es de tipo unicast local único

\item {} 
\sphinxAtStartPar
\sphinxcode{\sphinxupquote{fea6:791a:c569:86a1:bf71:adcb:11:fe15}} es de tipo unicast local en enlace

\item {} 
\sphinxAtStartPar
\sphinxcode{\sphinxupquote{fdfd:9440:8179:7eef:0:a606:0:1fe7}} es de tipo unicast local único

\item {} 
\sphinxAtStartPar
\sphinxcode{\sphinxupquote{3b2f:90fa:0:bb9e:13b:ab79::}} es de tipo unicast global

\item {} 
\sphinxAtStartPar
\sphinxcode{\sphinxupquote{fea9:1415:0:e7bc:0:fbcd::}} es de tipo unicast local en enlace

\item {} 
\sphinxAtStartPar
\sphinxcode{\sphinxupquote{ff5e:0:336f:0:7f53::}} es de tipo multicast

\item {} 
\sphinxAtStartPar
\sphinxcode{\sphinxupquote{fe83:dd6c:0:9584:3367:b654:75b3:0}} es de tipo unicast local en enlace

\item {} 
\sphinxAtStartPar
\sphinxcode{\sphinxupquote{fc58:0:bc51:10a3::}} es de tipo unicast local único

\item {} 
\sphinxAtStartPar
\sphinxcode{\sphinxupquote{ff3a:e8a3::3c07:5a1e:0}} es de tipo multicast

\item {} 
\sphinxAtStartPar
\sphinxcode{\sphinxupquote{feae::2230:0:e8b}} es de tipo unicast local en enlace

\item {} 
\sphinxAtStartPar
\sphinxcode{\sphinxupquote{ffcd:0:6ecc:9718:0:c20e::}} es de tipo multicast

\item {} 
\sphinxAtStartPar
\sphinxcode{\sphinxupquote{ff9e:0:246f:0:ed59::7fb4}} es de tipo multicast

\item {} 
\sphinxAtStartPar
\sphinxcode{\sphinxupquote{ff68:0:e756:95e1::67f1:0}} es de tipo multicast

\item {} 
\sphinxAtStartPar
\sphinxcode{\sphinxupquote{fc5e:d122:7da3:896e:626c::}} es de tipo unicast local único

\item {} 
\sphinxAtStartPar
\sphinxcode{\sphinxupquote{3bc1:0:ab12:85bf:2274::}} es de tipo unicast global

\item {} 
\sphinxAtStartPar
\sphinxcode{\sphinxupquote{ff09:7ffd:d25c:2e26:aeac:3a45:0:f1e}} es de tipo multicast

\item {} 
\sphinxAtStartPar
\sphinxcode{\sphinxupquote{fd03:0:5db4:d114:2972:0:c484:0}} es de tipo unicast local único

\item {} 
\sphinxAtStartPar
\sphinxcode{\sphinxupquote{feac:cfe9::e3c9:0}} es de tipo unicast local en enlace

\item {} 
\sphinxAtStartPar
\sphinxcode{\sphinxupquote{ff05::9c60:0:0:4b4a}} es de tipo multicast

\item {} 
\sphinxAtStartPar
\sphinxcode{\sphinxupquote{2499:0:aeac:0:12e:cbdf:7b95:0}} es de tipo unicast global

\item {} 
\sphinxAtStartPar
\sphinxcode{\sphinxupquote{ff9f::5cc0}} es de tipo multicast

\item {} 
\sphinxAtStartPar
\sphinxcode{\sphinxupquote{fdc5:289e:0:c613::7add:6ea5}} es de tipo unicast local único

\item {} 
\sphinxAtStartPar
\sphinxcode{\sphinxupquote{ff79:8c::ece7:0:0}} es de tipo multicast

\item {} 
\sphinxAtStartPar
\sphinxcode{\sphinxupquote{fd16:8393:2506::9bb4:75be:0}} es de tipo unicast local único

\item {} 
\sphinxAtStartPar
\sphinxcode{\sphinxupquote{ff9c:875b:3f71:0:299e::}} es de tipo multicast

\item {} 
\sphinxAtStartPar
\sphinxcode{\sphinxupquote{fddd::1d58:0:bfa5:f060:0}} es de tipo unicast local único

\end{enumerate}



\renewcommand{\indexname}{Índice}
\printindex
\end{document}